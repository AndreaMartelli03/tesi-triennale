Nel 1951 L. Lusternik e A. Fet hanno dimostrato in \cite{lusternik1951variational} che in ogni varietà Riemanniana chiusa (cioè compatta e senza bordo) esiste una geodetica chiusa. Già nel 1898 J. Hadamard in \cite{hadamard1898surfaces} aveva dimostrato che, in una varietà non semplicemente connessa, ogni curva chiusa è omotopa a una geodetica chiusa, che corrisponde al punto di minimo dell'energia nella relativa classe di omotopia. Se la varietà è semplicemente connessa, non è più possibile trovare una geodetica chiusa solo minimizzando l'energia: nel 1917 G. Birkhoff in \cite{birkhoff1917dynamical} ha usato un procedimento di minmax per mostrare l'esistenza di una geodetica chiusa su una qualunque superficie compatta di genere 0. 

Il primo capitolo è una rapida presentazione delle geodetiche su una varietà Riemanniana, l'oggetto del teorema di Lusternik-Fet. 

Il secondo capitolo introduce lo spazio \(H^1(\Sp^1,M)\) delle curve chiuse con energia finita, su cui verranno applicate le procedure variazionali per dimostrare il teorema di Lusternik-Fet. Per semplicità, l'approccio è estrinseco: questo renderà molto chiaro l'utilizzo di tutti gli strumenti variazionali.

Il terzo capitolo presenta il prototipo del procedimento di minmax: il teorema di passo montano, dovuto ad A. Ambrosetti e P. Rabinowitz \cite{ambrosetti1973dual}. Nella seconda sezione declineremo il teorema nel caso particolare dello spazio \(H^1(\Sp^1,M)\) e del funzionale energia. 

Nel quarto capitolo viene dimostrato il teorema di Lusternik-Fet. La prima sezione tratta il caso in cui la varietà non è semplicemente connessa, utilizzando i metodi diretti del calcolo delle variazioni per minimizzare l'energia nelle classi di omotopia non banali delle curve chiuse. La seconda sezione tratta il caso in cui la varietà è semplicemente connessa con il teorema di passo montano. La terza sezione è una rapida lista di alcuni risultati di esistenza di geodetiche chiuse e ipersuperfici minime chiuse, che sono venuti dopo il Teorema di Lusternik-Fet.