%%%% FRONTESPIZIO %%%%%
\begin{titlepage}
	%%%% LOGO %%%%%
	\begin{figure}[h]
		\centering
%		\includegraphics{logo.png}
		\includegraphics{image1.jpeg}
	\end{figure}
	
	
	%%%% Corso di Laurea %%%%%% 
	\setlength{\parskip}{-12pt} % Modificare per spaziare diversamente le righe
	\noindent\rule{\textwidth}{.5pt}
	
	\begin{center}
		\fontsize{14pt}{18pt}\selectfont{DIPARTIMENTO DI MATEMATICA}\\
		\vspace{0.25cm}
		\fontsize{14pt}{18pt}\selectfont {Corso di Laurea  in Matematica}
	\end{center}
%	\noindent\rule{\textwidth}{.4pt}
	%%%%%%%%%%%%%%%
	
	\vspace{3 cm} 
	
	%%%% Titolo %%%%%%
	\begin{center}
		{\fontsize{20}{30}\selectfont Il teorema di Lusternik-Fet sull'esistenza di geodetiche chiuse \par} %}
	
	\end{center}
	%%%%%%%%%%%%%%%
	\vspace{3 cm}
	
	\begin{tabular}{p{0.5\textwidth} p{0.5\textwidth}}
		\fontsize{14pt}{22pt}\selectfont {Supervisore:} & \fontsize{14pt}{22pt}\selectfont{Candidato:} \\
		\fontsize{16pt}{30pt}\selectfont {Alessandro Carlotto} & \fontsize{16pt}{30pt}\selectfont {Andrea Martelli} \\
	\end{tabular}
	
%	\begin{flushleft}
%		\rule{7.9cm}{.4pt}\\[0.1cm]
%		{\textit{\large Supervisore}}:\\
%		\large  Alessandro Carlotto \\
%	\end{flushleft}
	
%	\begin{flushright}
%		\rule{7.9cm}{.4pt}\\[0.1cm]
%		{\textit{\large Candidato}}:\\
%		\large Andrea Martelli\\
%	\end{flushright}
	
%	\rule{7.9cm}{.4pt}\\[0.1cm]


\vfill % Manda il resto a fondo pagina

%%%% Anno accademico %%%%%
\setlength{\parskip}{-18pt} % Modificare per spaziare diversamente le righe
%\noindent\rule{\textwidth}{.4pt}
\begin{center}
	\fontsize{14pt}{18pt}\selectfont{Anno Accademico 2023/24}
\end{center}
\noindent\rule{\textwidth}{0.5pt}
\setlength{\parskip}{0pt}

\end{titlepage}