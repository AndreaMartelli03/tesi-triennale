%INSTABILITA' GEODETICA SFERA
\begin{figure}[h]
	\centering
	\begin{tikzpicture}[decoration={markings, mark=at position 0.5 with {\arrow{>}}}]
		%\draw [help lines] (-3,-4) grid (10,4);
		
		% DOMINIO
		%linee
		\draw [thick] (0,0) circle (2);
		\draw[thick] (-2,0) arc (180:360: 2 and 0.55);
		\draw[decorate] (-2,0) arc (180:360: 2 and 0.55);
		\draw [thick, dashed] (2,0) arc (0:180: 2 and 0.55);
		\draw[thin,->] (0,0) -- (0,3);
		\draw[thin,->] (0,0) -- (3,0);
		\draw[thin,->] (0,0) -- (-2.3,-1.5);
		\draw [thick](-1.908,0.6) arc(180:360:1.908 and 0.3);
		\draw [decorate](-1.908,0.6) arc(180:360:1.908 and 0.3);
		\draw [thick,dashed](1.908,0.6) arc(0:180:1.908 and 0.3);
		
		%nomi
		\node at(2,2){\(\Sp^{2}\)};
		\node at (-2.2,-1.8) {\(x\)};
		\node at (3,-0.3){\(y\)};
		\node at (-0.4,2.8){\(z\)};
		\node at  (0,-1) {\(\gamma_0\)};
		\node at  (-0.5,0) {\(\gamma_s\)};
		
	\end{tikzpicture}
	
	\caption{La funzione \(s \mapsto E(\gamma_s)\) non ha un minimo locale per \(s=0\).}
	
	\label{fig: instabilità geodetica}
	
\end{figure}