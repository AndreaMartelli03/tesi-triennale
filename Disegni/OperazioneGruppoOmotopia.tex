	\begin{figure}[h]
	\centering
	\begin{tikzpicture}[decoration={markings, mark=at position 0.5 with {\arrow{<}}}]
		\draw[thick](-2,-2) rectangle (2,2);
		\draw[thick](-1,-1) rectangle (1,1);
		\draw[thin](-1,-1) to (-2,-2);
		\draw[thin](1,-1) to (2,-2);
		\draw[thin](-1,1) to (-2,2);
		\draw[thin](1,1) to (2,2);
		\draw[thin](-1,0) to (-2,0);
		\draw[thin](0,-1) to (0,-2);
		\draw[thin](1,0) to (2,0);
		\draw[thin](0,1) to (0,2);
		\draw[thin](-0.5,-1) to (-1,-2);
		\draw[thin](0.5,-1) to (1,-2);
		\draw[thin](1,-0.5) to (2,-1);
		\draw[thin](1,0.5) to (2,1);
		\draw[thin](0.5,1) to (1,2);
		\draw[thin](-0.5,1) to (-1,2);
		\draw[thin](-1,0.5) to (-2,1);
		\draw[thin](-1,-0.5) to (-2,-1);
		\draw[thin,decorate](-1,-1) to (-2,-2);
		\draw[thin,decorate](1,-1) to (2,-2);
		\draw[thin,decorate](-1,1) to (-2,2);
		\draw[thin,decorate](1,1) to (2,2);
		\draw[thin,decorate](-1,0) to (-2,0);
		\draw[thin,decorate](0,-1) to (0,-2);
		\draw[thin,decorate](1,0) to (2,0);
		\draw[thin,decorate](0,1) to (0,2);
		\draw[thin,decorate](-0.5,-1) to (-1,-2);
		\draw[thin,decorate](0.5,-1) to (1,-2);
		\draw[thin,decorate](1,-0.5) to (2,-1);
		\draw[thin,decorate](1,0.5) to (2,1);
		\draw[thin,decorate](0.5,1) to (1,2);
		\draw[thin,decorate](-0.5,1) to (-1,2);
		\draw[thin,decorate](-1,0.5) to (-2,1);
		\draw[thin,decorate](-1,-0.5) to (-2,-1);
		
		\node at (-0.7,-0.05){\(x_1\)};
		\node at (-2.3,-0.05){\(x_0\)};
		\node at (0.7,0.05){\(x_1\)};
		\node at (2.3,0.05){\(x_0\)};
		\node at (0,-0.7){\(x_1\)};
		\node at (0,-2.3){\(x_0\)};
		\node at (0,0.7){\(x_1\)};
		\node at (0,2.3){\(x_0\)};
		\node at (-1.5, 0.3){\(\gamma\)};
		\node at (0,0){\(f\)};
		
	\end{tikzpicture}
	
	\caption{Identificando \(\Sp^n = I^n/ \de I^n\), con \(I=[0,1]\), possiamo lavorare sull'ipercubo \(I^n\). Dunque la figura definisce, per ogni \(f : (\Sp^n,e_n) \to (X,x_1)\), una mappa continua \(\gamma f: (\Sp^n,e_n) \to (X,x_0)\) e possiamo definire \(\pi_n\gamma[f] \coloneq [\gamma f]\).}
	\label{fig: cambio punto base}
\end{figure}