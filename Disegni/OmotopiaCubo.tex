%OMOTOPIA CUBO
	\begin{figure}[h]
	\centering
	\begin{tikzpicture}[scale=0.7, decoration={markings, mark=at position 0.5 with {\arrow{>}}}]
		%Disegno cubo
		\draw [thick,postaction={decorate}] (0,0) -- (5,0);
		\draw [thick,postaction={decorate}] (0,5) -- (5,5);
		\draw [thick, dashed ,postaction={decorate}] (3,2) -- (8,2);
		\draw [thick ,postaction={decorate}] (3,7) -- (8,7);
		\draw [thick] (0,0)--(0,5);
		\draw [thick] (5,0)--(5,5);
		\draw [thick] (8,2)--(8,7);
		\draw [thick, dashed] (3,2)--(3,7);
		\draw [thick] (0,5)--(3,7);
		\draw [thick] (5,5)--(8,7);
		\draw [thick] (5,0)--(8,2);
		\draw [thick, dashed] (0,0)--(3,2);
		
		%coloro facce e freccine
		\fill[lightgray,opacity=0.15](0,0)--(3,2)--(3,7)--(0,5);
		\fill[lightgray,opacity=0.15](5,0)--(8,2)--(8,7)--(5,5);
		\draw [use Hobby shortcut,->] (-1,3)..(0,2)..(2,2);
		\draw [dashed,->](8,5.5) to [out=185, in=60] (7,4.8);
		\draw (8.8,5.3)to[out=150,in=5](8,5.5);
		
		%nomi
		\node at (2.5,-0.5) {\(\gamma_0\)};
		\node at (2.5,5.5) {\(\widetilde{\gamma_0}\)};
		\node at (3+2.5,2-0.5) {\(\gamma_1\)};
		\node at (3+2.5,2+5.5) {\(\widetilde{\gamma_1}\)};
		\node at (-1,3.5){\(K_0\)};			
		\node at (9.2,5) {\(K_1\)};
		
		
		
	\end{tikzpicture}
	
	\caption{L'omotopia \(K\) tra \(H\) e \(\widetilde{H}\) induce delle omotopie tra \(\gamma_i\) e \(\widetilde{\gamma_i}\), per \(i=0,1\). Dato che la coordinata \(s\) parametrizza la circonferenza \(\Sp^1\), i lati grigi sono identificati.}
	
	\label{fig: cubo di omotopie}
	
\end{figure}