\documentclass[italian,12pt,a4paper]{book}
\usepackage[centering]{geometry}
\usepackage[T1]{fontenc}
\usepackage[utf8]{inputenc} % Ensure proper encoding
\usepackage{lmodern} % Use Latin Modern fonts which provide more sizes
\usepackage{babel}

%----- Math ---------------------------------------------
\usepackage{amsmath}         
\usepackage{mathtools}       
\usepackage{amssymb}   
\usepackage{amsthm}         

%----- Design ------------------------------------------- 
\usepackage{lastpage}                              
\usepackage{enumitem}

%----- Intestazioni ---------------------------------------
\usepackage{fancyhdr}
\pagestyle{fancy}
\fancyhf{} % Clear all headers and footers
\fancyhead[LE,RO]{\nouppercase{\leftmark}} % Header left on even pages, right on odd pages
\fancyfoot[C]{\thepage} % Footer center, page number
\setlength{\headheight}{14.5pt}

% Pacchetto titlesec per personalizzare i titoli dei capitoli
\usepackage{titlesec}

% Comando per settare l'intestazione correttamente per l'introduzione
\newcommand{\chapterwithintroduction}[1]{
	\chapter*{#1}
	\addcontentsline{toc}{chapter}{#1}
	\markboth{#1}{}
}

%----- Pacchetti Disegno --------------------------------
\usepackage{tikz}
\usepackage{tikz-cd}
\usepackage{graphicx}
\usepackage{thmtools}
\usepackage{xcolor}
\usepackage[all]{xy}        
\usetikzlibrary{decorations.markings}
\usetikzlibrary{hobby}
\usepackage{pgfplots}
\pgfplotsset{compat=1.18}

%----- Symbols -----------------------------------------
\providecommand{\R}{\mathbb{R}}
\providecommand{\N}{\mathbb{N}}
\providecommand{\Z}{\mathbb{Z}}
\providecommand{\Sp}{\mathbb{S}}
\providecommand{\D}{\mathbb{D}}
\providecommand{\E}{\mathbb{E}}
\providecommand{\F}{\mathbb{F}}
\providecommand{\T}{\mathcal{T}}
\providecommand{\de}{\partial}
\providecommand{\ve}{\varepsilon}
\providecommand{\vphi}{\varphi}
\providecommand{\dif}{\mathrm{d}}
\providecommand{\integral}[3]{\int_{#1} #2 \mathrm{d}#3}
\providecommand{\id}{\mathrm{id}}
\providecommand{\sff}{\mathrm{I\!I}}
\newcommand{\mapsfrom}{\mathrel{\reflectbox{\ensuremath{\mapsto}}}}

%----- MathOperators -----------------------------------
\DeclareMathOperator{\grad}{grad}

%----- Hyperref ------------------------------------------
\usepackage{nameref}
\usepackage{csquotes}
\usepackage{biblatex}
\usepackage{hyperref}
\hypersetup{
	colorlinks=true,
	linkcolor=black, % Color for normal internal links
	filecolor=black, % Color for file links
	urlcolor=black, % Color for external links
	citecolor=black, % Color for citations
	pdftitle={Il teorema di Lusternik-Fet sull'esistenza di geodetiche chiuse},
	pdfpagemode=FullScreen,
}
\addbibresource{BiblioTesi.bib}

% -----Ambienti matematici-------------------------------
\theoremstyle{plain}
\newtheorem{teo}{Teorema}[chapter]
\newtheorem{prop}[teo]{Proposizione}
\newtheorem{cor}[teo]{Corollario}
\newtheorem{lemma}[teo]{Lemma}
\newtheorem*{teoLF}{Teorema~\ref{teo: Lusternik-Fet}}
\newtheorem*{prop*A}{Proposizione~\ref{prop: esiste f omotopicamente non banale}}
\newtheorem*{conget}{Congettura}
\theoremstyle{definition}
\newtheorem{defi}[teo]{Definizione}
\theoremstyle{remark}
\newtheorem{oss}[teo]{Osservazione}
\newtheorem{es}[teo]{Esempio}

\numberwithin{figure}{chapter}

\setcounter{tocdepth}{1}

% Definizione di un nuovo driver per nascondere URL e DOI
\AtEveryBibitem{%
	\clearfield{url} % Nasconde l'URL
	\clearfield{doi} % Nasconde il DOI
	\clearfield{isbn} % Nasconde ISBN
	\clearfield{issn} % Nasconde ISSN
	\clearfield{note} % Nasconde le note
}

\title{Il teorema di Lusternik-Fet sull'esistenza di geodetiche chiuse}
\author{Andrea Martelli}
\begin{document}
	
	%%%% FRONTESPIZIO %%%%%
\begin{titlepage}
	%%%% LOGO %%%%%
	\begin{figure}[h]
		\centering
%		\includegraphics{logo.png}
		\includegraphics{image1.jpeg}
	\end{figure}
	
	
	%%%% Corso di Laurea %%%%%% 
	\setlength{\parskip}{-12pt} % Modificare per spaziare diversamente le righe
	\noindent\rule{\textwidth}{.5pt}
	
	\begin{center}
		\fontsize{14pt}{18pt}\selectfont{DIPARTIMENTO DI MATEMATICA}\\
		\vspace{0.25cm}
		\fontsize{14pt}{18pt}\selectfont {Corso di Laurea  in Matematica}
	\end{center}
%	\noindent\rule{\textwidth}{.4pt}
	%%%%%%%%%%%%%%%
	
	\vspace{3 cm} 
	
	%%%% Titolo %%%%%%
	\begin{center}
		{\fontsize{20}{30}\selectfont Il teorema di Lusternik-Fet sull'esistenza di geodetiche chiuse \par} %}
	
	\end{center}
	%%%%%%%%%%%%%%%
	\vspace{3 cm}
	
	\begin{tabular}{p{0.5\textwidth} p{0.5\textwidth}}
		\fontsize{14pt}{22pt}\selectfont {Supervisore:} & \fontsize{14pt}{22pt}\selectfont{Candidato:} \\
		\fontsize{16pt}{30pt}\selectfont {Alessandro Carlotto} & \fontsize{16pt}{30pt}\selectfont {Andrea Martelli} \\
	\end{tabular}
	
%	\begin{flushleft}
%		\rule{7.9cm}{.4pt}\\[0.1cm]
%		{\textit{\large Supervisore}}:\\
%		\large  Alessandro Carlotto \\
%	\end{flushleft}
	
%	\begin{flushright}
%		\rule{7.9cm}{.4pt}\\[0.1cm]
%		{\textit{\large Candidato}}:\\
%		\large Andrea Martelli\\
%	\end{flushright}
	
%	\rule{7.9cm}{.4pt}\\[0.1cm]


\vfill % Manda il resto a fondo pagina

%%%% Anno accademico %%%%%
\setlength{\parskip}{-18pt} % Modificare per spaziare diversamente le righe
%\noindent\rule{\textwidth}{.4pt}
\begin{center}
	\fontsize{14pt}{18pt}\selectfont{Anno Accademico 2023/24}
\end{center}
\noindent\rule{\textwidth}{0.5pt}
\setlength{\parskip}{0pt}

\end{titlepage}
	
		% Dedica
\clearpage
\thispagestyle{empty}
\vspace*{4cm}
\begin{flushright}
	\textit{Ad Aldo e SMAG,}\\
	\textit{anche per tutti i libri di matematica che mi hanno regalato.}
\end{flushright}

%\clearpage

% Ringraziamenti
%\chapter*{Ringraziamenti}
%\addcontentsline{toc}{chapter}{Ringraziamenti}
%Ringrazio le persone che credono in me, e che me lo ricordano sempre. Ringrazio le persone che mi sono state vicine, anche quando sono sembrato lontano: sono state essenziali. Ringrazio la mia famiglia, perché mi sprona a sognare. 
	
	\tableofcontents
	
	\chapterwithintroduction{Introduzione}
	%INTRODUZIONE

Il metodo diretto del calcolo delle variazioni è una tecnica molto utile per dimostrare l'esistenza di punti di minimo o di massimo di un funzionale, ma non permette di individuare i punti di ‘‘sella''. In queste situazioni può risultare una strategia efficace utilizzare i metodi di min-max, di cui il prototipo è il teorema di passo montano di A. Ambrosetti e P. Rabinowitz \cite{ambrosetti1973dual}.

Questo elaborato indaga un esempio del genere presentando una dimostrazione, tratta da \cite{klingenberg1995riemannian}, del teorema di L. Lusternik e A. Fet \cite{lusternik1951variational}:
\begin{teoLF}[Lusternik-Fet, 1951]
	Ogni varietà Riemanniana chiusa ammette una geodetica chiusa non costante. 
\end{teoLF}
Se la varietà non è semplicemente connessa è possibile applicare il metodo diretto del calcolo delle variazioni per trovare delle geodetiche come punti di minimo locale del funzionale energia (cf. Sezione~\ref{sez: non semplicemente connesso}); se la varietà è semplicemente connessa adotteremo dei metodi di min-max (cf. Sezione~\ref{sez: semplicemente connesso}). 

Il primo caso, nel quale è possibile minimizzare localmente l'energia, era già stato risolto nel 1898 da J. Hadamard in \cite{hadamard1898surfaces}. Il secondo, molto più delicato, è stato risolto per la prima volta da G.D. Birkhoff nel 1917 nel caso particolare delle superfici (a meno di omeomorfismi, l'unica superficie chiusa e semplicemente connessa è la sfera): la dimostrazione è uno dei primi esempi di procedimenti di min-max. Infine, nel 1951 Lusternik e Fet hanno dimostrato il teorema per tutte le varietà chiuse. 

Con l'intento di rendere più chiaro l'utilizzo di tutti gli strumenti variazionali, ho scelto un approccio estrinseco, cioè ho assunto quasi sempre che la varietà sia una sottovarietà di uno spazio euclideo piatto di dimensione alta. Trattando solo varietà differenziabili  \(C^\infty\), in virtù del teorema di embedding isometrico liscio di Nash \cite{nash1956imbedding}, questa non è un'ipotesi restrittiva. Tuttavia, è possibile seguire un approccio completamente intrinseco (si veda \cite{klingenberg1995riemannian} oppure \cite{klingenberg2012lectures}).

Il primo capitolo è una rapida introduzione alle geodetiche di una varietà Riemanniana, basata soprattutto su \cite{milnor1963morse}, con una parentesi sulle sottovarietà Riemanniane nella terza sezione, tratta da \cite{lee1997riemannian}. L'intenzione è di fissare il linguaggio geometrico per la dimostrazione del teorema di Lusternik-Fet.

Il secondo capitolo introduce lo spazio e il funzionale adatto ad affrontare la dimostrazione del teorema di Lusternik-Fet utilizzando le opportune tecniche di calcolo delle variazioni: lo spazio \(H^1(\Sp^1,M)\) delle curve chiuse di classe \(H^1\) e il funzionale energia \(E\).

Il terzo capitolo è dedicato al teorema di passo montano: la prima sezione su spazi di Hilbert, la seconda sullo spazio \(H^1(\Sp^1,M)\) con funzionale energia. 

Le prime due sezioni del quarto capitolo trattano la dimostrazione del teorema di Lusternik-Fet. L'ultima sezione è una rapida lista di alcuni risultati di esistenza di geodetiche e ipersuperfici minime, associati al teorema di Lusternik-Fet sia per il contenuto geometrico che per le tecniche di min-max usate nelle dimostrazioni.
	
	\chapter{Geodetiche su varietà Riemanniane}
	Questo capitolo è una rapida presentazione di alcuni concetti di geoemtria differenziale: più nello specifico di geometria Riemanniana, con prevalente riferimento a \cite[Part~II]{milnor1963morse} e \cite[Capitoli~6-7]{abate2011geometria}.

Sia \(M\) una varietà differenziabile liscia di dimensione \(n\). Nel seguito, con \textit{liscia} si intende \(C^\infty\). Inoltre assumiamo sempre che \(M\) sia connessa e senza bordo. Denotiamo:
\begin{itemize}[label=]
	\item \(C^\infty(M)\) lo spazio delle funzioni \(f:M \to \R\) lisce;
	\item \(T_pM\) lo spazio tangente a \(M\) in \(p\);
	\item \(\pi: TM \to M\) il fibrato tangente;
	\item \(\T(M)\) lo spazio dei campi vettoriali lisci su \(M\).
\end{itemize}

\begin{defi}
	Una varietà compatta e senza bordo si dice \textit{chiusa}. 
\end{defi}

\section{Connessioni e derivate covarianti}
	
	Una \textit{connessione} su una varietà è un modo di derivare un campo vettoriale lungo una qualunque direzione in maniera ‘‘chiusa'', cioè in modo tale che il risultato sia un vettore tangente. Quanto segue è tratto principalmente da \cite[Chapter~8]{milnor1963morse} e \cite[Sezione~6.1]{abate2011geometria}.
	
	\begin{defi}
		Una \textit{connessione affine} su un punto \(p \in M\) è una mappa bilineare
		\begin{align*}
			\nabla^p:T_pM \times \T(M) &\to T_pM \\
			(X_p,Y) &\mapsto \nabla^p_{X_p}Y = \nabla_{X_p}Y
		\end{align*}
		che soddisfi
		\[
		\nabla_{X_p} fY = (X_pf) Y + f(p)\nabla_{X_p}Y 
		\]
		per ogni \(X_p \in T_pM, \ f \in C^\infty(M), \ Y \in \T(M)\). Il vettore \(\nabla_{X_p}Y\) è detto \textit{derivata covariante} di \(Y\) lungo \(X_p\).
		
		Una \textit{connessione affine globale} (o in breve \textit{connessione}) su \(M\) è una mappa che associa a ogni \(p \in M\) una connessione affine \(\nabla^p\) su \(p\) tale che l'applicazione
		\begin{align*}
			\nabla: \T(M) \times \T(M) &\to \T(M)\\
			(X,Y) &\mapsto \nabla_X Y,
		\end{align*}
		definita da \((\nabla_XY)_p \coloneq \nabla^p_{X_p}Y\), sia ben definita, cioè \(\nabla_X Y \in \T(M)\).
	\end{defi}
	\begin{oss}
		Una connessione su \(M\) soddisfa le seguenti proprietà:
		\begin{enumerate}[label=(\roman*)]
			\item \(\nabla:\T(M) \times \T(M) \to \T(M)\) è bilineare;
			\item \(\nabla_{fX}Y = f \nabla_XY\) per ogni \(f \in C^\infty(M)\), \(X,Y \in \T(M)\);
			\item \(\nabla_X(fY)= (Xf)Y+f\nabla_XY\) per ogni \(f \in C^\infty(M)\), \(X,Y \in \T(M)\).
		\end{enumerate}
		
		Se \((u^1,\dots , u^n)\) sono coordinate su un aperto \(U \subset M\), una connessione \(\nabla\) su \(U\) è univocamente determinata da \(n^3\) funzioni \(\Gamma_{ij}^k\), dette \textit{simboli di Christoffel}. Infatti, sia \(\de_k = \de/\de u^k \in \T(U)\) il frame della carta e scriviamo ogni campo \(X \in \T(U)\) come
		\[
		X= \sum_k  x^k \de_k
		\]
		per delle funzioni \(x^k \in C^\infty(U)\). In particolare, per ogni \(i,j\),
		\[
		\nabla_{\de_i} \de_j = \sum_k \Gamma_{ij}^k\de_k
		\]
		e quindi, usando le proprietà di connessione, per ogni \(X,Y \in \T(U)\) si ottiene che
		\begin{equation*}
			\nabla_X Y = \sum_{k}\left(Xy^k+\sum_{i,j}\Gamma_{ij}^kx^iy^j\right) \de_k.
		\end{equation*}
		
		Inoltre, date \(n^3\) funzioni \(\Gamma^k_{ij}\), questa formula definisce una connessione su \(U\). Per i dettagli, si può vedere \cite[Osservazione~6.1.9]{abate2011geometria}
	\end{oss}
	
	\begin{es}[Connessione piatta]\label{es: connessione patta}
		In \(\R^n\) possiamo definire la connessione data dalla derivata direzionale in ogni componente
		\[
			\nabla_ X Y \coloneq \sum_k (Xy^k) \de_k
		\]
		cioè con tutti i simboli di Christoffel nulli. Questa connessione è detta \textit{connessione piatta}. 
	\end{es}

	\begin{defi}\label{def: curva}
		Una \textit{curva parametrizzata} (o semplicemente \textit{curva}) è una mappa liscia \(\gamma: J \to M\), con \(J\) una varietà (differenziabile) 1-dimensionale, quindi \(J=\R\) (o un intervallo aperto) oppure \(J=\Sp^1 = \R/\Z\).  Talvolta prenderemo \(J=[a,b]\), intendendo che \(\gamma\) è la restrizione di una curva definita su un intervallo aperto che contiene \([a,b]\). Se \(J= \Sp^1\), diremo che \(\gamma\) è una \textit{curva chiusa}. 
	\end{defi}
	Una curva chiusa \(\gamma:\Sp^1 \to M\) si solleva in modo unico a una curva definita su \(\R\), denotata ancora \(\gamma:\R \to M\), che è 1-periodica, ovvero \(\gamma(t)=\gamma(t+1)\) per ogni \(t \in \R\).
	
	\begin{defi}
		Un \textit{campo vettoriale lungo una curva} \(\gamma:J \to M\) è una funzione
		\begin{align*}
			V: J &\to TM \\
			t &\mapsto V_t
		\end{align*}
		tale che \(V_t \in T_{\gamma(t)}M\) e che \(Vf \in C^\infty(J)\) per ogni \(f \in C^\infty(M)\), dove \(Vf(t):=V_t f\) per ogni \(t \in J\). Lo spazio vettoriale dei campi lungo \(\gamma\) è denotato \(\T(\gamma)\). 
		
		Il campo vettoriale \textit{velocità} (o \textit{vettore tangente}) di una curva \(\gamma:\R \to M\) è 
		\[
		\dot \gamma = \frac{\dif \gamma}{\dif t} \coloneq \dif \gamma \frac{\dif}{\dif t} \in \T(\gamma)
		\]
		dove
		\[
		\dif \gamma:T\R \to TM
		\]
		è il differenziale di \(\gamma\) e \(\dif/\dif t\) è il campo vettoriale standard di \(\R\). Se \(\gamma\) è chiusa, allora la velocità della curva solevata è 1-periodica, e quindi passa al quoziente, definendo la velocità della curva \(\dot \gamma:\Sp^1 \to TM\) sulla circonferenza \(\Sp^1\). 
	\end{defi}
	
	\begin{lemma}
		Sia \(\nabla\) una connessione su \(M\) e \(\gamma:\R \to M\) una curva. Allora esiste un unico operatore lineare
		\begin{align*}
			D_\gamma: \T(\gamma) \to \T(\gamma)
		\end{align*}
		tale che
		\begin{enumerate}[label=(\roman*)]
			\item per ogni \(f \in C^\infty(\R)\) e \(V \in \T(\gamma)\),
			\[
			D_\gamma (fV) =\dot f V + fD_\gamma V;
			\]
			dove \(\dot f = \dif f / \dif t\);
			\item se \(V \in \T(\gamma)\) è indotto da \(\widetilde{V} \in \T(M)\), cioè 
			\[
			V_t = \widetilde{V}_{\gamma(t)} \qquad \forall t \in \R,
			\]
			allora
			\[
			\left(D_\gamma V\right)_t = \nabla_{\dot{\gamma}(t)}\widetilde{V} \qquad \forall t \in \R.
			\]
		\end{enumerate}
	\end{lemma}
	\begin{proof} L'unicità segue dalla (ii), che in coordinate dà la formula 
		\begin{equation*}
			D_\gamma V = \sum_k \left(\dot v^k + \sum_{i,j}(\Gamma_{ij}^k \circ \gamma) \dot u^i v^j \right) (\de_k)_{\gamma}.
		\end{equation*}
		per \(V = \sum_k v^k (\de_k)_\gamma\). L'esistenza si ottiene definendo \(D_\gamma\) con tale formula. Per i dettagli si può vedere \cite[Lemma~8.1]{milnor1963morse} oppure \cite[Proposizione~6.1.12]{abate2011geometria}
	\end{proof}
	
	\begin{defi}
		L'operatore \(D_\gamma\) è detto \textit{derivata covariante lungo} \(\gamma\). 
	\end{defi}
	
	\begin{defi}
		Una connessione \(\nabla\) su \(M\) è detta \textit{simmetrica} se per ogni \(X,Y \in \T(M)\)
		\[
		\nabla_X Y - \nabla_YX = [X,Y],
		\]
		dove \([X,Y]\) denota le \textit{Lie-brackets} di \(X\) e \(Y\), definite da
		\[
		[X,Y] f \coloneq X(Yf) - Y(Xf) \qquad \forall f \in C^\infty(M).
		\]
	\end{defi}
	\begin{oss}
		Sia \(\nabla\) una connessione su \(M\). In coordinate, siccome \([\de_i,\de_j]=0\), se \(\nabla\) è simmetrica allora
		\[
		\Gamma_{ij}^k= \Gamma_{ji}^k.
		\]
		Viceversa, se \(\Gamma_{ij}^k = \Gamma_{ji}^k\) per ogni scelta di coordinate, si verifica facilente che \(\nabla\) è simmetrica.
	\end{oss}
	\begin{es}
		La connessione piatta definita nell'Esempio~\ref{es: connessione patta} è simmetrica. 
	\end{es}
	
	
	
	\section{Varietà Riemanniane}
	
	La geometria Riemanniana studia le varietà differenziabili in cui è possibile misurare lunghezza e angoli, e di conseguenza in cui è possibile quantificare la nozione di curvatura. Tuttavia la nozione di curvatura va oltre lo scopo del capitolo, pertanto in questa sezione viene data solo la definizione di metrica Riemanniana e connessione di Levi-Civita, seguendo soprattutto \cite[Chapter~8]{milnor1963morse} e \cite[Sezioni~6.5-6]{abate2011geometria}.
	
	\begin{defi}
		Una \textit{metrica Riemanniana} su una varietà \(M\) è una mappa \(g\) che a ogni punto \(p \in M\) associa un prodotto scalare \(g_p : T_p M \times T_pM \to \R\) e tale che, per ogni scelta di coordinate \((u^1,\dots, u^n)\) su \(U\subset M\), le funzioni
		\begin{align*}
			g_{ij}:U &\to \R \\
			p &\mapsto g_p((\de_i)_p,(\de_j)_p)
		\end{align*}
		siano \(C^\infty\). La coppia \((M,g)\) è detta \textit{varietà Riemanniana}. Talvolta denoteremo \(g_p = \langle \cdot,\cdot \rangle_p\) e \(| \cdot |_p\) la norma su \(T_pM\) indotta da \(g_p\). 
	\end{defi}
	
	\begin{es}[Metrica piatta]\label{es: metrica piatta}
		In \(\R^n\) il prodotto scalare standard induce una metrica Riemanniana, detta \textit{metrica piatta}. I coefficienti metrici sono
		\[
			g_{ij} \equiv \delta_{ij} = \begin{cases}
				1 & \text{ se } i = j \\
				0 & \text{ se } i \neq j
			\end{cases}
		\]
	\end{es}
	
	\begin{lemma}\label{lemma: frame ortonormale}
		Per ogni \(p \in M\) esiste un frame ortonormale locale, cioè esiste un intorno \(U\) e dei campi \(E_1,\dots,E_n \in \T(U)\) tali che in ogni punto di \(U\) formino una base ortonormale dello spazio tangente. 
	\end{lemma}
	\begin{proof}
		È sufficiente prendere una carta locale che contenga \(p\) e applicare il procedimento di ortonormalizzazione di Gram-Schmidt al frame \(\de_1, \dots,\de_n\). 
	\end{proof}
	
	Grazie alla metrica Riemanniana, è possibile definire la lunghezza di una curva. Diremo che una mappa continua \(\gamma:[a,b] \to M\) è una curva \textit{curva liscia regolare a tratti} se esiste una partizione \(a=t_0<t_1< \dots < t_k=b\) di \([a,b]\) tale che per ogni \(i \in \{1,\dots, k\}\) la restrizione \(\gamma|_{[t_{i-1},t_i]}\) sia una curva liscia (cf. Definizione~\ref{def: curva}) con vettore tangente sempre non nullo. Denotiamo con \(C^\infty_{piec}([a,b],M)\) l'insieme di tali curve. La \textit{lunghezza} di una curva \(\gamma \in C^\infty_{piec}([a,b],M)\) è 
	\begin{equation}\label{eq: funzionale lunghezza}
	L(\gamma) \coloneq \int_a^b |\dot \gamma(t) |_{\gamma(t)} \ \dif t. 
	\end{equation}
	
	È naturale dunque porre su \(M\) la seguente distanza: per ogni \(p,q \in M\)
	\[
	d(p,q) \coloneq \inf\{L(\gamma) \ | \ \gamma \in C^\infty_{piec}([0,1],M) \text{ t.c. } \gamma(0)= p, \gamma(1)=q\}.
	\]
	È immediato verificare che \(d\) è una distanza su \(M\). Seguirà dalla Proposizione~\ref{prop: proprietà locali exp} che \(d\) induce su \(M\) la stessa topologia che induce la struttura di varietà.
		
	%%distanza indotta dalla metrica

\begin{figure}[h]
	\centering
	\begin{tikzpicture}
		%\draw [help lines] (-4,-2) grid (4,2);
		\fill (-3,-1)circle (2pt);
		\node at (-3.2,-0.7){\(p\)};
		\fill (0,.5) circle (2pt);
		\node at (0,0.8){\(q\)};
		\fill (3,-0.5) circle (2pt);
		\node at (3.2,-0.8){\(r\)};
		\draw[blue, dashed] (0,.5) circle (0.7071);
		\draw[thick] (-3,-1) to[out=20,in=220](-.5,0) to[out=40, in =230](0,.5) to[out=300, in=150] (.5,0) to[out=330, in=160] (3,-0.5);
		\draw[thick, blue] (-.5,0) to[out=40, in=150] (.5,0);

		
	\end{tikzpicture}
	
	\caption{Costruzione per dimostrare la disuguaglianza triangolare}
	
	\label{fig: metrica}
	
\end{figure}
	
	
	Data una curva \(\gamma: \R \to M\) regolare, cioè tale che \(\dot \gamma(t) \neq 0\) per ogni \(t \in \R\), la metrica Riemanniana permette sempre di riparametrizzare in maniera naturale la curva con la \textit{lunghezza d'arco}:
	\[
		s(t) \coloneq \int_0^t |\dot \gamma(u)| \ \dif u, \qquad t \in \R
	\]
	è un diffeomorfismo perché \(\dot s = |\dot \gamma| \neq 0\). La curva \(\gamma \circ s^{-1}\) è la \textit{riparametrizzazione in lunghezza d'arco} di \(\gamma\). Notare che \(\gamma\)  è parametrizzata in lunghezza d'arco, cioè \(s = \id\), se e solo se \(|\dot \gamma | \equiv 1\).
	
	
	Passiamo allo studio della relazione tra metrica Riemanniana e connessione. Rendiamo precisa la nozione di compatibilità tra queste due strutture.
	\begin{defi}
		Una connessione \(\nabla\) su \(M\) si dice \textit{compatibile} con la metrica \(g\) se per ogni curva \(\gamma:\R \to M\) e per ogni coppia di campi \(V,W \in \T(\gamma)\) vale 
		\[
		\frac{\dif}{\dif t} \langle V,W\rangle_\gamma = \left\langle D_\gamma V, W \right\rangle_\gamma + \left\langle V, D_\gamma W \right\rangle_\gamma.
		\]
	\end{defi}

	\begin{teo}[Levi-Civita]
		Ogni varietà Riemanniana \((M,g)\) ammette un'unica connessione simmetrica \(\nabla\) compatibile con la metrica \(g\), detta connessione di Levi-Civita. Inoltre in ogni carta i simboli di Christoffel sono dati da
		\begin{equation}\label{eq: simboli Christoffel di Levi-Civita}
		\Gamma_{ij}^k = \frac{1}{2} \sum_l g^{kl}\left( \frac{\de g_{lj}}{\de u^i} + \frac{\de g_{il}}{\de u^j} - \frac{\de g_{ij}}{\de u^l} \right),
		\end{equation}
		dove \((g^{ij})\) è la matrice inversa di \((g_{ij})\)
	\end{teo}
	\begin{proof}
		L'unicità si ottiene derivando le funzioni \(g_{ij} = g(\de_i,\de_j)\) per ottenere la formula (\ref{eq: simboli Christoffel di Levi-Civita}) in ogni carta. L'esistenza si mostra verificando che i simboli di Christoffel definiti da (\ref{eq: simboli Christoffel di Levi-Civita}) danno una connessione compatibile con la metrica Riemanniana. Per i dettagli si veda \cite[Teorema~6.6.6]{abate2011geometria} oppure \cite[Lemma~8.6]{milnor1963morse}.
	\end{proof}
	\begin{es}
		La connessione di Levi Civita della metrica piatta è la connessione piatta.
	\end{es}

	
\section{Sottovarietà Riemanniane}\label{sez: sottovarietà}
	
	Con l'intento di rendere più semplice il problema dal punto di vista variazionale, nei prossimi capitoli adotteremo spesso un approccio estrinseco, cioè considereremo delle sottovarietà \(M \subset \R^N\): è dunque necessario definire cos'è una sottovarietà Riemanniana ed esaminarne alcune proprietà. Questa sezione è tratta principalmente da \cite[Section~8.1]{lee1997riemannian}. Ricordiamo le seguenti definizioni.
	\begin{defi}
		Siano \(M\) e \(N\) due varietà differenziabili. Una mappa liscia \(j:M \to N\) si dice un \textit{embedding} se \(j\) è un omeomorfismo con l'immagine e per ogni \(p \in M\) il differenziale \(dj_p:T_pM \to T_{j(p)}N\) è iniettivo.
	\end{defi}
	
	\begin{defi}
		Una varietà differenziabile \(S\) si dice una \textit{sottovarietà differenziabile} di \(M\) se \(S \subset M\) e l'inclusione \(i:S \hookrightarrow M\) è un embedding. 
	\end{defi} 
	
	\begin{defi}
		Siano \(M\) una varietà differenziabile, \((N,g)\) una varietà Riemanniana e \(j:M \to N\) un embedding. La \textit{metrica pull-back} su \(M\) è la metrica \(j^*g\) definita da
		\[
		(j^*g)_p(v,w) \coloneq g_{j(p)}(\dif j_pv,\dif j_p w)
		\]
		per ogni \(p \in M\) e \(v,w \in T_pM\). 
	\end{defi}
	Non è difficile verificare che \(j^*g\) sia effettivamente una metrica Riemanniana su \(M\).
	
	\begin{defi}
		Una \textit{sottovarietà Riemanniana} \(M \subset \widetilde{M}\) della varietà Riemanniana \((\widetilde{M},\widetilde{g})\) è una sottovarietà differenziabile dotata della metrica pull-back \(g=i^*\widetilde{g}\) (denotata anche \(\widetilde{g}|_S\)), dove \(i:M \to \widetilde{M}\) è l'inclusione. La varietà \(\widetilde{M}\) è detta \textit{varietà ambiente} e la metrica \(g\) è anche detta \textit{metrica indotta} da \(\widetilde{g}\) su \(M\). 
	\end{defi}
	
	\begin{defi}
		Un embedding \(j:(M,g^M) \to (N,g^N)\) tra varietà Riemanniane si dice \textit{isometrico} se \(g^M = j^*g^N\).
	\end{defi}
	
	Un risultato (profondo) di J. Nash \cite{nash1956imbedding} garantisce che ogni varietà Riemanniana può essere vista come una sottovarietà di \(\R^N\), dotato della metrica piatta, per \(N\) sufficientemente grande.
	
	\begin{teo}[Nash]\label{teo: embedding isometrico di Nash}
		Ogni varietà Riemanniana ammette un embedding isometrico in \(\R^N\), considerato con la metrica piatta. 
	\end{teo}
	Per una dimostrazione si veda \cite[Theorem 3.1.1]{delellis2017masterpieces}.
	
	Sia \((\widetilde{M},\widetilde{g})\) una varietà Riemanniana di dimensione \(N\) e consideriamo una sottovarietà \((M,g)\) di dimensione \(n\). Tutti i simboli tildati si riferiscono a \(\widetilde{M}\), altrimenti sono le restrizioni a \(M\). Siccome \(g\) e \(\widetilde{g}\) coincidono quando sono definiti entrambi, possiamo denotarle entrambe con \(\langle\cdot ,\cdot\rangle\) senza rischio di confusione. 
	
	In maniera naturale, l'insieme
	\[
	T\widetilde{M}|_M \coloneq \coprod_{p \in M} T_p\widetilde{M}
	\]
	è un fibrato vettoriale di rango \(N\) su \(M\) e una sottovarietà di \(T \widetilde{M}\), detto \textit{fibrato tangente ambiente} su \(M\). Inoltre ogni sezione liscia di \(T\widetilde{M}\) si restringe a una sezione liscia di \(T\widetilde{M}|_M\) e viceversa ogni sezione liscia di \(T\widetilde{M}|_M\) si può estendere a una sezione liscia di \(T\widetilde{M}\) (si veda \cite[p. 132]{lee1997riemannian}). Denoteremo con \(\T(\widetilde{M}|_M)\) lo spazio delle sezioni lisce di \(T\widetilde{M}|_M\).
	
	In ogni punto \(p \in M\), lo spazio tangente ambiente \(T_p\widetilde{M}\) si decompone in somma diretta, ovvero
	\[
	T_p\widetilde{M} = T_pM \oplus N_pM
	\]
	dove \(N_pM \coloneq (T_pM)^\perp\) è lo \textit{spazio normale} a \(p\) rispetto a \(\widetilde{g}_p\) su \(T_p\widetilde{M}\). L'insieme
	\[
	NM \coloneq \coprod_{p \in M}N_pM
	\]
	è naturalmente un fibrato vettoriale di rango \(N-n\) su \(M\) e una sottovarietà di \(T\widetilde{M}|_M\), detto \textit{fibrato normale} di \(M\) (si veda \cite[p. 133]{lee1997riemannian}). Denoteremo con \(\mathcal{N}(M)\) lo spazio delle sezioni lisce di \(NM\).
	
	La proiezione ortogonale a ogni punto \(p\) di \(T_p\widetilde{M}\) su \(T_pM\) o su \(N_pM\) fornisce le mappe lisce
	\begin{align*}
		\pi^\parallel : T\widetilde{M}|_M &\to TM \\
		\pi^\perp : T\widetilde{M}|_M &\to NM
	\end{align*}
	dette \textit{proiezione tangenziale} e \textit{normale} rispettivamente. Spesso useremo la notazione
	\begin{align*}
		&X^\parallel = \pi^\parallel (X), &X^\perp = \pi^\perp (X)
	\end{align*}
	per brevità. Ogni punto ammette un frame ortonormale locale \(E_1,\dots,E_N\) di \(\R^N\) tale che \(E_1, \dots, E_n\) sia un frame ortonormale locale di \(M\). In questo caso, se \(X = \sum_i x^iE_i\), 
		\begin{align*}
		&X^\parallel = \sum_{i=1}^n x^iE_i, \\
		&X^\perp = \sum_{j=n+1}^N x^iE_i.
	\end{align*}
	
	C'è una forte relazione tra la connessione di Levi Civita dell'ambiente \(\widetilde{\nabla}\) e la connessione di Levi Civita \(\nabla\) di \(M\), che può essere espressa con la \textit{seconda forma fondamentale}. 
	
	Sia \(A:\T(M) \times \T(\widetilde{M}) \to \mathcal{N}(M)\) la mappa definita da
	\[
	A(X,\widetilde{Y}) \coloneq (\widetilde{\nabla}_X \widetilde{Y})^\perp.
	\]
	per ogni \(X \in \T(M)\) e \(\widetilde{Y} \in \T(\widetilde{M})\). Osserviamo che \(A\) è bilineare perché la connessione è bilineare e la composizione con \(\pi^\perp\) è lineare. Osserviamo che, se \(\widetilde{X}\) estende \(X\) e \(Y\) è la restrizione di \(\widetilde{Y}\), allora
	\[
	A(X,\widetilde{Y})-A(Y,\widetilde{X}) = (\widetilde{\nabla}_X \widetilde{Y}-\widetilde{\nabla}_Y\widetilde{X})^\perp =[X,Y]^\perp = 0.
	\]
	siccome \([X,Y] \in TM\). Poiché \((\widetilde{\nabla}_X \widetilde{Y})_p\) dipende solo da \(X_p\) e \(\widetilde{Y}\), la simmetria implica che \(A(X,\widetilde{Y})_p\) dipende solo da \(X_p\) e \(Y_p\), e in particolare non dipende da come scegliamo $\widetilde{Y}$ per estendere \(Y\). Dunque nel seguito non denotiamo allo stesso modo campi vettoriali ed estensioni su \(T\widetilde{M}|_M\). Possiamo dare la seguente definizione.
	\begin{defi}
		La \textit{seconda forma fondamentale} è una mappa \(\sff\) che associa a ogni \(p\) la forma bilineare \(\sff_p\) definita da
		\[
		\sff(X_p,Y_p) = \sff_p(X_p,Y_p) \coloneq A(X,Y)_p
		\]
		per due qualunque campi \(X\) e \(Y\) che estendono i vettori \(X_p\) e \(Y_p\). 
	\end{defi}
	
	Si noti che \(\sff(X,Y) = A(X,Y) \in \mathcal{N}(M)\). Il seguente teorema dice che la seconda forma fondamentale misura la distanza tra la connessione ambiente \(\widetilde{\nabla}\) e la connessione \(\nabla\) di \(M\). Per una dimostrazione, si veda \cite[Theorem 8.2, Lemma 8.3]{lee1997riemannian}.
	
	\begin{teo}\label{teo: formula Gauss}
		Per ogni \(X,Y \in \T(M)\) vale la \textnormal{formula di Gauss}:
		\begin{equation}\label{eq: formula Gauss}
			\widetilde{\nabla}_{X}Y = \nabla_XY+ \sff(X,Y).
		\end{equation}
		Inoltre, su \(M\), vale l'\textnormal{equazione di Weingarten}:
		\begin{equation}\label{eq: Weingarten}
			\langle \widetilde{\nabla}_{X} N, Y \rangle = - \langle N, \sff(X,Y) \rangle,
		\end{equation}
		per ogni \(X,Y \in \T(M)\), \(N \in \mathcal{N}(M)\). 
	\end{teo}
	

	\begin{oss}
		Sia \(\gamma:\R \to M\) una curva e siano \(\widetilde{D}_\gamma\) e \(D_\gamma\) le derivate covarianti lungo essa rispetto alla connessione ambiente e alla connessione di \(M\) rispettivamente. La formula di Gauss (\ref{eq: formula Gauss}) diventa
		\begin{equation}
			\widetilde{D}_\gamma V  = D_\gamma V + \sff(\dot \gamma, V)
		\end{equation}
		per ogni campo vettoriale \(V\) lungo una curva \(\gamma\). Nel caso in cui \(\widetilde{M}=\R^N\) con la metrica piatta, 
		\begin{equation}\label{eq: formula Gauss curve in M in R^N}
			\dot V  = D_\gamma V + \sff(\dot \gamma, V).
		\end{equation}
		Analogamente, l'equazione di Weingarten diventa
		\begin{equation}\label{eq: Weingarten curve in M in R^N}
			\langle \dot N, V \rangle = - \langle N, \sff(\dot \gamma,V) \rangle 
		\end{equation}
		per ogni campo \(V\) lungo \(\gamma\) e \(N:\R \to NM\) liscia tale che \(N_t \in N_{\gamma(t)}M\). 
	\end{oss}
	

	
	\section{Geodetiche e mappa esponenziale}
	
	Questa sezione, basata principalmente su \cite[Chapter 10]{milnor1963morse}, è una introduzione alle geodetiche, l'oggetto del teorema di Lusternik-Fet, e alle loro proprietà principali. 
	
	Sia \((M,g)\) una varietà Riemanniana. Le geodetiche sono una generalizzazione dei segmenti di retta negli spazi euclidei con metrica piatta. Ci sono due proprietà fondamentali comuni a tutti i segmenti:
	\begin{enumerate}[label=(\arabic*)]
		\item dati due punti, esiste un'unica curva di lunghezza minima che li connette, cioè il segmento che unisce i due punti;
		\item sono le uniche curve che possono essere parametrizzate in modo tale che l'accelerazione (cioè la derivata seconda) sia nulla.
	\end{enumerate}
	La proprietà (1) riguarda in realtà il supporto delle curve, ed è una questione di carattere globale. Invece la proprietà (2) riguarda la parametrizzazione della curva, e ha un carattere locale, nel senso che è una richiesta da fare per ogni valore del parametro. 
	
	Non è possibile generalizzare la nozione di retta a partire dalla proprietà (1), perché ci sono esempi in cui una curva di lunghezza minima che connette due punti non esiste oppure non è unica.
	
	\underline{Esistenza}: si consideri \(\R^2 \setminus\{(0,0)\}\), il piano senza l'origine; presi due punti opposti rispetto all'origine, non esiste una curva di lunghezza minima che li connette (Figura~\ref{fig: esistenza e unicità geodetiche}, sinistra).
	
	\underline{Unicità}: si consideri la sfera \(\Sp^2\); tutti i meridiani sono curve di lunghezza minima tra le curve che connettono il polo nord e il polo sud (Figura~\ref{fig: esistenza e unicità geodetiche}, destra).
	
		
	\begin{figure}[ht]
		\centering
		\begin{tikzpicture}
			%\draw[help lines] (-6,-3) grid (6,3);
			
			%FIGURA PIANO
			\draw[thick] (-6,-2) rectangle (-1,2);
			\node at (-4.7,1.5){\(\R^2 \setminus\{(0,0)\}\)};
			\fill (-5,-1) circle (1.5 pt)
				(-2,1) circle (1.5 pt);
			\draw (-3.5,0) circle (1.5 pt);
			\node at (-5.2,-1.2){\(p\)};
			\node at (-1.8,1.2){\(q\)};
			\draw[dashed](-5,-1) --(-3.6,-0.07);
			\draw[dashed](-3.4,0.07) -- (-2,1);
			\draw[use Hobby shortcut, thick](-5,-1) .. (-3.5 - 0.158,0.475) .. (-2,1);
			
			%FIGURA SFERA
			\draw [thick] (4,0) circle (2);
			\draw (2,0) arc (180:360: 2 and 0.55);
			\draw [dashed] (6,0) arc (0:180: 2 and 0.55);
			\fill (4,2) circle (1.5pt)
				(4,-2) circle (1.5pt);
			\node at (4,2.4){\(N\)};
			\node at (4,-2.4){\(S\)};
			\node at (6,1.5){\(\Sp^2\)};
			\draw[thick] (4,0) ellipse (1.55 and 2);
			\draw[thick] (4,0) ellipse (0.9 and 2);
			\draw[thick] (4,0) ellipse (0.3 and 2);
		\end{tikzpicture}
		\caption{Il problema della curva di lunghezza minima tra quelle che connettono due punti. A sinistra un caso in cui non esiste la soluzione, a destra un caso in cui la soluzione non è unica.}
		\label{fig: esistenza e unicità geodetiche}
	\end{figure}
	
	
	La definizione di geodetica è una generalizzazione della proprietà (2). Sorprendentemente la proprietà (1) viene recuperata parzialmente (cf. Teorema~\ref{teo: minimizzazione lunghezza}). 
	
	\begin{defi}
		Una \textit{geodetica} è una curva \(\gamma: J \to M\) che ha accelerazione nulla, ovvero tale che \(D_\gamma\dot \gamma \equiv 0\).
	\end{defi}
	
	La parametrizzazione della curva è importante nella definizione di geodetica, perché se \(\theta:J \to J'\) è un diffeomorfismo qualunque, allora
	\begin{equation}\label{eq: riparametrizzazione geodetica}
		D_\gamma \frac{\dif}{\dif t} (\gamma \circ \theta) = D_\gamma( \dot \theta \dot \gamma) = \ddot \theta\dot \gamma + \dot \theta D_\gamma \dot \gamma.
	\end{equation}
	Quindi se \(\ddot \theta \neq 0\), $\widetilde{\gamma}$ non è una geodetica anche se \(\gamma\) lo è.
	
	Una conseguenza immediata della definizione è che la norma del vettore tangente è costante, cioè la parametrizzazione è un multiplo della lunghezza d'arco:
	\begin{equation}
	\frac{\dif}{\dif t} |\dot \gamma |^2 = \frac{\dif}{\dif t} \langle \dot \gamma, \dot \gamma \rangle = 2 \left\langle D_\gamma \dot \gamma, \dot \gamma \right\rangle =0.
	\end{equation}
	In altre parole, se \(\gamma\) è una geodetica e \([a,b]\subset I\), la lunghezza dell'arco \(\gamma|_{[a,b]}\) è esattamente \(|\dot \gamma|(b-a)\).
	
	Come volevamo, questa definizione generalizza (2): negli spazi euclidei con metrica piatta, le geodetiche sono tutti e soli i segmenti di rette con parametrizzazione affine.
	
	In coordinate \((u^1, \dots, u^n)\), l'equazione \(D\dot \gamma = 0\) ha la forma
	\begin{equation}\label{eq: equazione geodetiche}
		\ddot u^k +\sum_{i,j} \Gamma_{ij}^k \dot u^i \dot u^j = 0, \qquad k=1,\dots , n,
	\end{equation}
	dove \((u^1(t), \dots, u^n(t))\) è l'espressione in coordinate di \(\gamma\). Ricordiamo che vale il seguente risultato di equazioni differenziali ordinarie (per una dimostrazione si veda \cite[Theorem X a pag. 154 e Thereom XIII a pag. 157]{walter1998ordinary}):
	\begin{teo}\label{teo: esistenza e unicità locale sol pbm Cauchy}
		Sia \(F \in C^\infty(\Omega, \R^n)\) con \(\Omega \subset \R^n \times \R^n\) aperto. Allora per ogni \((\bar{u}, \bar{v}) \in \Omega\) esistono un intorno \(U \subset \Omega\) di \((\bar{u},\bar{v})\) e un numero \(\ve>0\) tale che, per ogni \((u_0, v_0) \in U\), il problema di Cauchy
		\[\begin{cases*}
			\ddot u = F(u,\dot u) \\
			u(0)=u_0 \\
			\dot{u}(0) = v_0\\
		\end{cases*}
		\] 
		ammette un'unica soluzione \(\phi_{(u_0, v_0)}\) definita su \((-\ve,\ve)\). Inoltre, la soluzione è \(C^\infty\) e dipende in maniera \(C^\infty\) dai dati iniziali, cioè il \textit{flusso}
		\begin{align*}
			\phi: U \times (-\ve,\ve) & \to \R \\
			(u_0,v_0,t) & \mapsto \phi_{(u_0,v_0)}(t)
		\end{align*}
		è di classe \(C^\infty\).
	\end{teo}

	
	Applichiamo questo risultato alle equazioni delle geodetiche (\ref{eq: equazione geodetiche}).
	\begin{teo}\label{teo: esistenza mappa exp}
		Per ogni punto \(p_0 \in M\), esiste un intorno \(U\) di \(p_0\) e un numero \(\ve>0\) tale che per ogni \(p \in U\) e \(v \in T_pM\) con \(|v|_p < \ve\) esiste un'unica geodetica
		\[
		\gamma_v: (-2,2) \to M
		\]
		tale che
		\[
		\gamma_v(0)=p, \quad \dot \gamma_v(0) = v.
		\]
		Inoltre, se 
		\[
		V \coloneq \{v \in T_pM : p \in U, \ |v|_p < \ve\},
		\]
		la mappa 
		\begin{align*}
			\gamma: V \times (-2,2) &\to M \\
			(v,t) &\mapsto \gamma_v(t)
		\end{align*}
		è di classe \(C^\infty\). 
	\end{teo}
	\begin{proof}
		Prendiamo delle coordinate \((u^1,\dots,u^n)\) su un aperto \(U\) e consideriamo l'equazione (\ref{eq: equazione geodetiche}). Per il Teorema~\ref{teo: esistenza e unicità locale sol pbm Cauchy}, a meno di restringere \(U\), esistono \(\ve_1,\ve_2>0\) tali che per ogni \(p \in U\) e \(v \in T_pM\) con \(|v|_p < \ve_1\), esista un'unica geodetica 
		\[
		\phi_v:(-2\ve_2,2\ve_2) \to M
		\]
		con \(\phi_v(0)=p\) e \(\dot \phi_v(0)=v\) e, se 
		\[
		V':=\{v \in T_pM : p \in U, \ |v|_p < \ve_1\},
		\]
		la mappa
		\begin{align*}
			\phi:V' \times(-2\ve_2,2\ve_2) &\to M \\
			(v,t) &\mapsto \phi_v(t)
		\end{align*}
		sia di classe \(C^\infty\). Osserviamo ora che se \(\sigma:(-\delta,\delta) \to M\) è una geodetica e \(c \in \R\), allora anche
		\begin{equation*}
			t \mapsto \sigma(ct), \qquad t \in (-\delta/c,\delta/c)
		\end{equation*}
		è una geodetica, con vettore tangente \(c\dot{\sigma}\).
		Dunque, sia \(\ve < \ve_1 \ve_2\) e sia 
		\[
		V:=\{v \in TqM : q \in U, \ |v|_q < \ve\} \subset TM.
		\]
		Notiamo che se \(v \in V\) e \(t \in (-2,2)\), allora \(|v/\ve_2|<\ve_1\), cioè \(v/\ve_2 \in V'\), e \(\ve_2 t \in (-2\ve_2,2\ve_2)\). Dunque possiamo definire
		\begin{align*}
			\gamma: V \times (-2,2) &\to M \\
			(v,t) &\mapsto \gamma_v(t) \coloneq \phi_{v/\ve_2}(\ve_2 t).
		\end{align*}
	\end{proof}
	
		Siano \(p \in M\) e \( v \in T_pM\) tali che esista un'unica geodetica \(\gamma_v:[-1,1] \to M\) con \(\gamma_v(0)=p\) e \(\dot \gamma_v(0)=v\). Allora
		\[
		\exp(v)=\exp_p(v):= \gamma_v(1).
		\]
		Dal Teorema~\ref{teo: esistenza mappa exp}, risulta che la mappa \(\exp\) è definita e di classe \(C^\infty\) su un aperto \(\mathcal{E} \subset TM\) che contiene lo zero \(0_p\) di \(T_pM\) per ogni \(p \in M\). Inoltre \(\mathcal{E}_p = \mathcal{E} \cap T_pM\) è stellato in \(0_p\) e, con la notazione del Teorema~\ref{teo: esistenza mappa exp},
		\[
		\gamma_v(t)= \exp(tv), \qquad \forall t \in [-1,1].
		\]
		\begin{defi}
			La mappa \(\exp: \mathcal{E} \to M\) è detta \textit{mappa esponenziale}. 
		\end{defi}
	
	La mappa esponenziale ha delle ottime proprietà locali. 
	\begin{prop}\label{prop: proprietà locali exp}
		Per ogni \(p_0 \in M\) esistono un intorno \(W\) e un numero \(\ve>0\) tali che per ogni \(p \in W\):
		\begin{enumerate}[label=(\roman*)]
			\item se \(B_\ve(0_p)\) è la palla di \(T_pM\) di raggio \(\ve\) e centrata in \(0_p\), \(\exp_p|_{B_\ve(0_p)}\) è un diffeomorfismo con l'immagine, che contiene \(W\);
			\item per ogni \(q \in W\), esiste un'unica geodetica \(\gamma_{pq}\) che connette \(p\) a \(q\) di lunghezza minore di \(\ve\), ed è della forma
			\[
			t \mapsto \exp_p(tv)
			\]
			per qualche \(v \in B_\ve(0_p)\); inoltre \(\gamma_{pq}\) dipende in maniera \(C^\infty\) da \(p\) e \(q\).
		\end{enumerate}
	\end{prop}
	\begin{proof}
		Sia \(p_0 \in M\) e siano \(U \subset M\) e \(V \subset TM\) come nel Teorema~\ref{teo: esistenza mappa exp}. Definiamo \(F:V \to M \times M\)
		\[
		F(v) \coloneq (\pi(v),\exp(v)), \qquad \forall v \in V,
		\]
		dove \(\pi:TM \to M\) è il fibrato tangente. Osserviamo che \(F\) è non singolare in \(0_{p_0}\). Infatti, il differenziale delle coordinate \((u^1,\dots,u^n) \in C^\infty(U,\R^n)\) induce delle coordinate \((u^1,\dots,u^n,v^1,\dots,v^n)\) su \(U \times TU \subset TM\). Su \(U \times U \subset M \times M\), consideriamo le coordinate prodotto, che denotiamo, per distinguerle, con \((u_1^1,\dots,u_1^n,u_2^1,\dots,u_2^n)\). Allora
		\[
		\dif F \left(\frac{\de}{\de u^i}\right)_{0_{p_0}} = \left(\frac{\de}{\de u_1^i}\right)_{(p_0,p_0)} + \left(\frac{\de}{\de u_2^i}\right)_{(p_0,p_0)}
		\]
		\[
		\dif F \left( \frac{\de}{\de v^i} \right)_{0_{p_0}} = \left(\frac{\de}{\de u_2^i}\right)_{(p_0,p_0)}.
		\]
		Quindi, in coordinate, la matrice Jacobiana di \(F\) in \(0_{p_0}\) ha la forma
		\[\begin{pmatrix}
			I & I \\
			0 & I
		\end{pmatrix}\]
		e quindi è non singolare. Dal teorema della funzione inversa segue che esiste un intorno \(V'\subset V\) di \(0_{p_0}\) tale che \(F|_{V'}\) sia un diffeomorfismo sulla sua immagine. Eventualmente prendendo un intorno più piccolo, possiamo assumere che \(V'\) sia della forma 
		\[
		V' \coloneq \{v \in V \cap T_pM : |v|_p<\ve, \ p \in W\}
		\]
		per qualche intorno \(W \subset M\) di \(p_0\) ed \(\ve>0\). Osserviamo che \(W \times W \subset F(V')\). Allora per ogni \(p \in W\), anche \(F\) ristretta a \(V'\cap T_pM=B_\ve(0_p)\) è un diffeomorfismo sull'immagine, e in particolare lo è la mappa esponenziale. 
		
		Il punto (ii) si dimostra osservando che 
		\[
		\left| \frac{\dif}{\dif t} \exp_p(tv) \right| \equiv |v|_p 
		\]
		e che \(\exp_p|_{B_\ve(0_p)}\) è un diffeomorfismo sull'immagine. 
		
	\end{proof}
	
	\begin{defi}
		Il \textit{raggio di iniettività} di \(M\) in \(p\) è 
		\[
		r(p) \coloneq \sup\{\ve>0 : \exp|_{B_\ve(0_p)} \text{ è un diffeomorfismo sull'immagine}\}.
		\]
		dove \(B_\ve(0_p)\) è la palla di \(T_pM\) di raggio \(\ve\) centrata in \(0_p\). 
		
		Per ogni \(\ve < r(p)\), 
		\[
		B_\ve(p):=\exp_p(B_\ve(0_p)) = \{ q \in M : d(p,q) < \ve\}
		\]
		è detta \textit{palla geodetica} di raggio \(\ve\) centrata in \(p\).
	\end{defi}
	
	\begin{cor}\label{cor: epsilon per compatta}
		Sia \((M,g)\) una varietà Riemanniana chiusa. Allora esiste \(\ve>0\) tale che \(r(p)>\ve\) per ogni \(p \in M\). 
	\end{cor}
	\begin{proof}
		Per ogni \(p \in M\), siano \(W_p\) e \(\ve_p\) come nella Proposizione~\ref{prop: proprietà locali exp}. Osserviamo che per ogni \(q \in W_p\), \(r(q)\geq \ve_p\). Estraiamo un sottoricoprimento finito da \(\{W_p\}_{p \in M}\) e prendiamo il minimo degli \(\ve_p\). 
	\end{proof}
	
	Come anticipato, recuperiamo parzialmente la proprietà (1) dei segmenti di retta in \(\R^n\).
	\begin{teo}\label{teo: minimizzazione lunghezza}
		Sia \(\gamma:[a,b] \to M\) una curva liscia regolare a tratti minimizzante, cioè con lunghezza minore o uguale della lunghezza di una qualunque curva liscia regolare a tratti che connette \(\gamma(a)\) a \(\gamma(b)\). Allora \(\gamma\) è una geodetica.
	\end{teo}
	\begin{proof}
		Si veda ad esempio \cite[Corollary 10.7]{milnor1963morse}.
	\end{proof}
	
	Il Teorema~\ref{teo: minimizzazione lunghezza} caratterizza i minimi del funzionale lunghezza \(L\), definito in (\ref{eq: funzionale lunghezza}), nello spazio delle curve lisce regolari a tratti che connettono due punti fissati \(p,q \in M\). Per provare l'esistenza dei minimi usando tecniche di calcolo delle variazioni, come ad esempio il metodo diretto, è opportuno completare lo spazio. Euristicamente, se fosse \(M = \R^n\), lo spazio naturale su cui estendere \(L\) è 
	\[	
	\{\gamma \in W^{1,1}([a,b],\R^n) \ | \ \gamma(a)=p, \gamma(b)=q\}
	\]
	perché è sufficiente che \(\dot \gamma\) sia di classe \(L^1\).
	
	Tuttavia, di solito è più pratico lavorare sugli spazi di Hilbert, e per questo motivo conviene definire il funzionale energia:
	\[
	E(\gamma) \coloneq \frac{1}{2} \int_a^b |\dot \gamma|^2 \ \dif t,
	\]
	per una curva \(\gamma \in C^\infty_{piec}([a,b],M)\). Lo spazio naturale su cui estendere \(E\) sarà legato allo spazio di Sobolev \(W^{1,2}\), che è uno spazio di Hilbert (cf. \cite[Section~2.3]{klingenberg1995riemannian}). Nel Capitolo 2, solo per il caso delle curve chiuse, definiremo la varietà di Hilbert su cui l'energia è estesa in maniera naturale. 
	
	Immediatamente, usando la disuguaglianza di H\"older, abbiamo che per ogni curva \(\gamma \in C^\infty_{piec}([a,b],M)\),
	\begin{equation}\label{eq: L^2<2E}
		L(\gamma)^2 = \left( \int_a^b |\dot \gamma| \ \dif t\right)^2 \leq (b-a) \int_a^b |\dot \gamma|^2 \ \dif t = 2(b-a) E(\gamma).
	\end{equation}
	con l'uguaglianza se e solo se \(|\dot \gamma|\) è costante, cioè se \(\gamma\) è parametrizzata da un multiplo di lunghezza d'arco. 
	\begin{teo}\label{teo: equiv minimi}
		Sullo spazio delle curve \(\gamma \in C^\infty_{piec}([a,b],M)\) che connettono due punti \(p\) e \(q\), i punti di minimo della lunghezza e i punti di minimo dell'energia coincidono e sono delle geodetiche.  
	\end{teo}
	\begin{proof}
		Se i punti di minimo coincidono, sono geodetiche per il Teorema~\ref{teo: minimizzazione lunghezza}. Senza perdita di generalità, possiamo assumere \([a,b]=[0,1]\).
		
		Sia \(\gamma_0 \in C^\infty_{piec}([a,b],M)\) un punto di minimo per la lunghezza, quindi una geodetica, che è parametrizzata da un multiplo della lunghezza d'arco. Allora per ogni \(\gamma \in C^\infty_{piec}([a,b],M)\) con \(\gamma(a)=p\) e \(\gamma(b)=q\), 
		\[
		E(\gamma_0) = \frac{1}{2}L(\gamma_0)^2 \leq \frac{1}{2}L(\gamma)^2 \leq E(\gamma).
		\]
		
		Viceversa, sia \(\gamma_0\) un punto di minimo per l'energia. Sicuramente \(\gamma_0\) è parametrizzata da un multiplo di lunghezza d'arco. Infatti, sia \(\hat{\gamma}_0:[0,l] \to M\) la riparametrizzazione di \(\gamma_0\) in lunghezza d'arco; possiamo considerare la riparametrizzazione di \(\gamma_0\) definita da
		\[
			\widetilde{\gamma}_0(t) \coloneq \hat{\gamma}_0(lt), \qquad t \in [0,1].
		\]
		Siccome il funzionale lunghezza coincide su tutte le riparametrizzazioni, 
		\[
			L(\gamma_0)^2 \leq 2E(\gamma_0) \leq 2E(\widetilde{\gamma}_0) = L(\widetilde{\gamma}_0)^2 = L(\gamma_0)^2
		\]
		quindi \(\gamma_0\) è parametrizzata in lunghezza d'arco. Sia ora \(\gamma \in C^\infty_{piec}([a,b],M)\) con \(\gamma(a)=p\) e \(\gamma(b)=q\), e consideriamo la riparametrizzazione in multiplo di lunghezza d'arco \(\widetilde{\gamma}\), costruita come appena fatto per \(\gamma_0\).
		Allora
		\[
		L(\gamma_0)^2 = 2E(\gamma_0) \leq 2 E(\widetilde{\gamma}) = L(\widetilde{\gamma})^2 = L(\gamma)^2.
		\]
	\end{proof}
	
	Non tutte le geodetiche sono dei punti di minimo della lunghezza o dell'energia. 
	
	\begin{es}\label{es: geodetiche non minimi}
		Si consideri ad esempio la sfera unitaria \(\Sp^2 \subset \R^3\), di equazione \(x^2+y^2+z^2=1\). Le geodetiche della sfera sono i cerchi massimi, ovvero sono parametrizzazioni a velocità costante dell'interesezione di \(\Sp^2\) e un piano passante per l'origine. Infatti, il vettore accelerazione di una tale curva, pensata come curva in \(\R^3\), è parallelo alla normale della sfera, e quindi la derivata covariante è nulla; inoltre, dato un punto e un vettore tangente, esiste sempre un cerchio massimo che passa per quel punto e con quel vettore tangente. 
		
		Siano \(p,q \in \Sp^2\) due punti non antipodali. Ci sono due geodetiche che connettono \(p\) e \(q\) e che hanno lunghezze diverse; dunque una è minimale rispetto alle curve che connettono \(p\) e \(q\) (sia rispetto al funzionale lunghezza che al funzionale energia, per il Teorema~\ref{teo: equiv minimi}), l'altra no.
		
		\begin{figure}[h]
	\centering
	\begin{tikzpicture}[decoration={markings, mark=at position 0.5 with {\arrow{>}}}]
		
		%linee
		\draw [thick] (0,0) circle (2);
		\draw[thick] (-2,0) arc (180:360: 2 and 0.55);
		\draw[decorate] (-2,0) arc (180:360: 2 and 0.55);
		\draw [thick, dashed] (2,0) arc (0:180: 2 and 0.55);
		\draw[decorate] (-2,0) arc (180:42: 2 and 0.55);
		\draw[thin,->] (0,0) -- (0,3);
		\draw[thin,->] (0,0) -- (3,0);
		\draw[thin,->] (0,0) -- (-2.3,-1.5);
		\fill (-0.777,-0.507) circle (1.5pt)
		(0.777,-0.507) circle (1.5pt);
		
		%nomi
		\node at(2,2){\(\Sp^{2}\)};
		\node at (-2.2,-1.8) {\(x\)};
		\node at (3,-0.3){\(y\)};
		\node at (-0.4,2.8){\(z\)};
		\node at  (0,-1) {\(\gamma_1\)};
		\node at  (-2.4,0) {\(\gamma_2\)};
		\node at (-0.777,-0.8){\(p\)};
		\node at (0.777,-0.8) {\(q\)};
		
	\end{tikzpicture}
	
	\caption{\(L(\gamma_1) < L(\gamma_2)\).}
	
	\label{fig: geodetica non minima}
	
\end{figure}
	\end{es}
	
	L'equazione delle geodetiche \(D\dot \gamma =0\) è l'\textit{equazione di Eulero-Lagrange} del funzionale energia. Infatti, sia \(\gamma_0:[a,b] \to M\) una curva liscia e consideriamo una mappa liscia \(\alpha:(-\ve,\ve) \times [a,b] \to M\) tale che \(\alpha(0,t) = \gamma_0(t)\) e \(\alpha(s,a)=\gamma_0(a)\), \(\alpha(s,b)=\gamma_0(b)\) per ogni \(s \in (-\ve,\ve)\). Denotiamo
	\begin{itemize}
		\item \(\gamma_s(t) = \alpha(s,t)\)
		\item \(W(t) = \frac{\de \alpha(0,t)}{\de s}\).
	\end{itemize}
	Notare che \(W(a)=0\) e \(W(b)=0\). Per ogni mappa \(\alpha\) di questo tipo, supponiamo che
	\[
		\left.\frac{\dif}{\dif s}\right|_{s=0} E(\gamma_s) = 0.
	\]
	Questo equivale a richiedere che \(\gamma_0\) soddisfi un'equazione differenziale. Supponiamo per semplicità che \(M\) sia una sottovarietà di \(\R^N\). 
	\begin{align*}
		0 &= \left.\frac{\dif}{\dif s}\right|_{s=0} E(\gamma_s) = \frac{1}{2}\int_a^b \left[ \frac{\de}{\de s} \left| \frac{\de \alpha(s,t)}{\de t}\right|^2 \right]_{s=0} \dif t \\
		&= \int_a^b \left[ \left\langle \frac{\de^2 \alpha(s,t)}{\de s\de t}, \frac{\de \alpha(s,t)}{\de t} \right\rangle \right]_{s=0} \dif t \\
		& = \int_a^b \left\langle \dot W, \dot \gamma_0 \right\rangle  \ \dif t =  \left.\langle W,\dot \gamma_0 \rangle\right|_a^b - \int_a^b \langle W, \ddot \gamma_0 \rangle  \ \dif t \\
		&= -\int_a^b \langle W, D \dot \gamma_0 \rangle  \ \dif t 
	\end{align*}
	Per il lemma fondamentale del calcolo delle variazioni, \(D \dot \gamma_0 \equiv 0\). Dunque le geodetiche sono i punti critici del funzionale energia.
	
	Calcoliamo invece l'equazione di Eulero-Lagrange associata al funzionale energia. Con le stesse notazioni di sopra,
	\begin{align*}
		0 &= \left.\frac{\dif}{\dif s}\right|_{s=0} L(\gamma_s) = \int_a^b \left[ \frac{\de}{\de s} \left| \frac{\de \alpha(s,t)}{\de t}\right| \right]_{s=0} \dif t \\
		&= \int_a^b \left[ \frac{ \left\langle \frac{\de^2 \alpha}{\de s\de t}, \frac{\de \alpha}{\de t} \right\rangle }{\left| \frac{\de \alpha}{\de t}\right|} \right]_{s=0} \dif t \\
		& = \int_a^b \frac{\langle \dot W, \dot \gamma_0 \rangle}{|\dot \gamma_0|} \ \dif t  = \left.\left\langle W,\frac{\dot \gamma_0}{|\dot \gamma_0|} \right\rangle\right|_a^b - \int_a^b \left\langle W,\frac{\dif}{\dif t} \frac{\dot \gamma_0}{|\dot \gamma_0|} \right\rangle  \ \dif t \\
		& = -\int_a^b \left\langle W , \frac{D\dot \gamma_0}{|\dot \gamma_0|} - \frac{\langle \dot \gamma_0,D\dot \gamma_0\rangle}{|\dot \gamma_0|^3} \dot \gamma_0 \right\rangle \ \dif t.
	\end{align*}
	Per il lemma fondamentale del calcolo delle variazioni, moltiplicando per \(|\dot \gamma_0|\), otteniamo l'equazione
	\begin{equation}\label{eq: EL lunghezza}
		D\dot \gamma_0 - \frac{\langle \dot \gamma_0,D\dot \gamma_0\rangle}{|\dot \gamma_0|^2} \dot \gamma_0 \equiv 0.
	\end{equation}
	Risolvere questa equazione è equivalente a dire che \(D \dot \gamma_0\) è parallelo a \(\dot \gamma_0\). Utilizzando la (\ref{eq: riparametrizzazione geodetica}), è immediato verificare che tutte le riparametrizzazioni di una geodetica sono delle soluzioni di (\ref{eq: EL lunghezza}). Viceversa, supponiamo che \(\gamma_0\) sia una soluzione di (\ref{eq: EL lunghezza}) e riparametrizziamola in multiplo di lunghezza d'arco, ottendendo \(\gamma_1\); in virtù della (\ref{eq: riparametrizzazione geodetica}), otteniamo che \(D\dot \gamma_1\) è ancora parallela a \(\gamma_0\), e quindi anche a \(\dot \gamma_0\). Tuttavia, essendo parametrizzata in multiplo di lunghezza d'arco, \(D\dot\gamma_1\) è ortogonale a \(\dot \gamma_1\), e quindi \(D\dot\gamma_1\equiv 0\). Dunque i punti critici del funzionale lunghezza sono le riparametrizzazioni di geodetiche.
	

	

	
	\chapter{Lo spazio \texorpdfstring{$H^1(\Sp^1,M)$}{H1(S1,M)} e il funzionale energia}
	
	Un'analisi accurata dal punto di vista variazionale del funzionale energia richiede l'introduzione di spazi completi su cui l'energia è definita in maniera naturale. Nell'ottica del teorema di Lusternik-Fet, ci restringiamo fin da subito allo studio dello spazio \(H^1(\Sp^1,M)\) delle curve \textit{chiuse} di classe \(H^1\), vale a dire con una nozione debole di vettore tangente e con energia finita.
	
	Nella prima sezione richiamo alcuni fatti di analisi funzionale, fondamentali per tutto quello che segue. La seconda sezione è una rapidissima introduzione alle varietà di Banach e di Hilbert, che spesso sono definiti ‘‘naturalmente'' a partire da varietà Riemanniane di dimensione finita. La terza sezione tratta la costruzione della varietà di Hilbert \(H^1(\Sp^1,M)\) delle curve chiuse di classe \(H^1\); l'approccio è estrinseco, cioè assumiamo che \(M \subset \R^N\). Infine nella quarta sezione studiamo la regolarità del funzionale energia su \(H^1(\Sp^1,M)\) e caratterizziamo le geodetiche (chiuse) come i suoi punti critici: questo giustifica tutta l'analisi variazionale del Capitolo 3.

\section{Richiami di analisi funzionale}

Richiamiamo qui alcune definizioni e alcuni risultati di analisi funzionale, tratti da \cite{brezis2011functional}, fondamentali per tutto quello che segue. 

Sia \(I\) un intervallo. Per \(1 \leq p \leq \infty\), denotiamo con \(W^{1,p}(I)\) lo spazio di Sobolev \((1,p)\), ovvero lo spazio delle funzioni \(u \in L^p(I)\) per le quali esiste \(\dot u \in L^p(I)\), detta \textit{derivata debole} di \(u\), tale che
\[
\int_I u\dot v \ \dif t = - \int_I \dot u v \ \dif t
\]
per ogni \(v \in C_0^\infty(I)\), cioè di classe \(C^\infty\) e a supporto compatto in \(I\). La derivata debole è unica in \(L^p(I)\) (cf. \cite[p. 203 Remark 3]{brezis2011functional}). Se \(I\) è un intervallo limitato, le inclusioni degli spazi \(L^p\) inducono le inclusioni
\[
	W^{1,p}(I) \subset W^{1,q}(I) \quad  \text{ se }p \geq q.
\]
Quando \(p = 2\), \(H^1(I) = W^{1,2}(I)\) è uno spazio di Hilbert se dotato del prodotto scalare
\begin{align*}
	\langle u,v \rangle_{H^1(I)} \coloneq & \langle u,v \rangle_{L^2(I)} + \langle \dot u, \dot v \rangle_{L^2(I)} &\\
	= & \int_I uv \ \dif t + \int_I \dot u \dot v \ \dif t  & u,v \in H^1(I).
\end{align*}
Se consideriamo funzioni a valori vettoriali \(u: I \to \R^n\), scriviamo \(L^p(I,\R^n)\), \(W^{1,p}(I,\R^n)\) e \(H^1(I,\R^n)\): le definizioni sono le stesse utilizzando il prodtto scalare di \(\R^n\), e in particolare
\[
\langle u, v \rangle_{H^1(I,\R^n)} = \int_I \langle u(t),v(t)\rangle_{\R^n} \ \dif t, \qquad u,v \in H^1(I,\R^n)
\]
Chiaramente, \(L^p(I,\R^n)\) e \(W^{1,p}(I,\R^n)\) sono canonicamente isomorfi al prodotto di \(n\) copie rispettivamente di \(L^p(I)\) e \(W^{1,p}(I)\).

Ricordiamo il seguente risultato (si veda \cite{brezis2011functional}, Theorem 8.2, Theorem 8.8, Theorem 3.17).
\begin{teo}\label{teo: immersione di Sobolev}
	Sia \(I\) un intervallo limitato. Per ogni \(u \in W^{1,1}(I)\) esiste un'unica funzione \(\widetilde{u} \in C^0(\overline{I})\) tale che \(u=\widetilde{u}\) quasi ovunque su \(I\) e
	\[
	\widetilde{u}(t_1)-\widetilde{u}(t_0) = \int_{t_0}^{t_1}\dot u \ \dif t, \qquad \forall t_0,t_1 \in I.
	\]
	Inoltre, se \(p>1\), la mappa iniettiva
	\begin{align*}
		W^{1,p}(I) &\hookrightarrow C^0(\overline{I}) \\
		u &\mapsto \widetilde{u}
	\end{align*}
	è compatta, cioè manda sottoinsiemi limitati in precompatti, e in particolare è continua. 
	
	Se \(1<p<+\infty\) per ogni successione limitata \((u_h)_h \subset W^{1,p}(I)\) esistono una sottosuccessione \((u_{h_k})_k\) e \(u \in W^{1,p}(I)\) tali che
	\begin{itemize}
		\item \(u_{h_k} \rightharpoonup u\) in \(W^{1,p}(I)\);
		\item \(u_{h_k} \to u\) in \(C^0(\overline{I})\).
	\end{itemize}
\end{teo}
Questo risultato si estende in maniera naturale agli spazi \(W^{1,p}(I,\R^n)\). Nel seguito, quando prendiamo \(u \in H^1(I,\R^n)\), in realtà supporremo sempre di aver preso il rappresentante continuo \(\widetilde{u}\). 

Lo spazio \(H^1(\Sp^1,\R^n)\) è la chiusura in \(H^1(\R,\R^n)\) dello spazio delle funzioni \(u \in C^\infty(\R,\R^n)\) 1-periodiche, cioè tali che \(u(1+t) = u(t)\) per ogni \(t \in \R\). Possiamo pensare alle funzioni \(u \in H^1(\Sp^1,\R^n)\) come mappe \(u: \Sp^1 \to \R^n\), dove \(\Sp^1 = [0,1]/\{0,1\}\), o come mappe \(u: \R \to \R^n\) 1-periodiche. Se \(I=(0,1)\), la mappa restrizione
\begin{align*}
	H^1(\Sp^1,\R^n) &\to H^1(I,\R^n) \\
	u &\mapsto u|_I
\end{align*}
è lineare, iniettiva e continua, e l'immagine è un sottospazio chiuso (e quindi anche debolmente chiuso) di \(H^1(I,\R^n)\). Dunque il Teorema~\ref{teo: immersione di Sobolev} si estende naturalmente allo spazio \(H^1(\Sp^1,\R^n)\) :

\begin{teo}\label{teo: immersione Sobolev per S^1}
	L'immersione del Teorema~\ref{teo: immersione di Sobolev} induce l'immersione compatta
	\[
		H^1(\Sp^1,\R^n) \hookrightarrow C^0(\Sp^1,\R^n).
	\]
	In particolare, per ogni successione limitata \((u_h)_h \subset H^1(\Sp^1,\R^n)\) esistono una sottosuccessione \((u_{h_k})_k\) e \(u \in H^1(\Sp^1,\R^n)\) tali che
	\begin{itemize}
		\item \(u_{h_k} \rightharpoonup u\) in \(H^1(\Sp^1,\R^n)\);
		\item \(u_{h_k} \to u\) in \(C^0(\Sp^1,\R^n)\).
	\end{itemize}
\end{teo}

\section{Varietà di Banach e di Hilbert}\label{sez: var Banach/Hilbert}
Questa sezione è dedicata a una rapida introduzione delle varietà di Banach e di Hilbert, basata parzialmente su \cite[Section~6.1]{ambrosetti2007nonlinear}. Consideriamo solo il caso delle sottovarietà di uno spazio di Banach \((\mathbb{E}, \| \cdot\|)\) o di uno spazio di Hilbert \((\mathbb{H}, \langle \cdot, \cdot \rangle)\); per una trattazione completa e generale, si veda \cite{lang2001fundamentals}.  I modelli da tenere presente sono delle varietà di mappe, ad esempio \(C^0(S,M)\) per le varietà di Banach, dove \(S\) è uno spazio topologico compatto e \((M,g)\) è una varietà Riemanniana (di dimensione finita), e \(H^1(\Sp^1,M)\) per le varietà di Hilbert, lo spazio delle curve chiuse di classe \(H^1\) su una varietà Riemanniana chiusa \((M,g)\). La Sezione~\ref{sez: H^1(S,M)} è interamente dedicata a costruire \(H^1(\Sp^1,M)\), mentre per la costruzione della struttura di varietà di Banach di \(C^0(S,M)\) rimandiamo a \cite{eells1958geometry}. 

\begin{defi}
Sia \((\mathbb{E}, \|\cdot \|)\) uno spazio di Banach. Un sottoinsieme \(E \subset \mathbb{E}\) è una \textit{varietà di Banach} (di classe \(C^\infty\)) se esiste un ricoprimento aperto \(\{U_\alpha\}_{\alpha \in \mathcal{A}}\) di \(E\) e una famiglia di spazi di Banach \(\{\E_\alpha\}_{\alpha \in \mathcal{A}}\) e una collezione di mappe \(\psi_\alpha : V_\alpha \to \mathbb{H}\), con \(V_\alpha \subset \E_\alpha\) aperto omeomorfo a una palla aperta di \(\E_\alpha\), tali che siano soddisfatte le seguenti condizioni:
\begin{itemize}
	\item \(\psi_\alpha(V_\alpha) = U_\alpha\) è aperto in \(\mathcal{E}_\alpha\) e \(\psi_\alpha\) è un omeomorfismo da \(V_\alpha\) in \(U_\alpha\);
	\item \(\psi^{-1}_\beta \circ \psi_\alpha: V_\alpha \cap \psi^{-1}_\alpha(U_\beta) \to \psi_\beta(U_\alpha) \cap V_\beta\) è di classe \(C^\infty\) (nel senso di Frechét, cf. \cite[Definition 1.1]{ambrosetti2007nonlinear});
	\item il differenziale \((\dif \psi_\alpha)_x: \E_\alpha \to \mathbb{H}\) sia iniettivo per ogni \(x \in V_\alpha\).
\end{itemize}
Le mappe \(\psi_\alpha\) sono dette \textit{parametrizzazioni locali} e la famiglia \(\{U_\alpha,\psi^{-1}_\alpha\}_{\alpha \in \mathcal{A}}\) è detta \textit{atlante}. 

Se \(\E= \mathbb{H}\) è uno spazio di Hilbert, allora \(E\) si dice una \textit{varietà di Hilbert}.

Una mappa \(f: E' \to E\) tra due varietà di Banach \(E' \subset \mathbb{E}'\) ed \(E\subset \mathbb{E}\) si dice \textit{liscia} oppure \textit{di classe \(C^\infty\)} se, data una parametrizzazione locale  \(\psi'_a:V_a' \subset \E'_a \to \mathbb{E}'\) di \(E'\) e una parametriazzazione locale \(\psi_\alpha:V_\alpha \subset \E_\alpha \to \mathbb{E}\) di \(E\) tale che \(\psi'_a(V_a') \cap f^{-1}(\psi_\alpha(V_\alpha)) \neq \emptyset\), la mappa
\[
	\psi_\alpha^{-1} \circ f \circ \psi_a': V_a' \cap (\psi_a')^{-1}(f^{-1}(\psi_\alpha(V_\alpha))) \to \E_\alpha
\]
è di classe \(C^\infty\) (nel senso di Frechét).
\end{defi}

Chiaramente, ogni sottoinsieme aperto di uno spazio di Banach (Hilbert) è una varietà di Banach (Hilbert). Un esempio importante di varietà di Banach è \(C^0(S,M)\), dove \(S\) è uno spazio topologico compatto e \((M,g)\) è una varietà Riemanniana (si veda \cite{eells1958geometry}). Notare che non abbiamo richiesto la proprietà di connessione, perché molte varietà di Banach (o di Hilbert) definite in maniera naturale a partire da varietà Riemanniane non sono connesse (cf. Sezione~\ref{sez: non semplicemente connesso}). 

\begin{defi}
	Siano \(E \subset \mathbb{E}\) una varietà di Banach e \(u \in E\). Sia \(\psi_\alpha: V_\alpha \subset \E_\alpha \to \mathbb{E}\) una parametrizzazione locale, con \(u=\psi_\alpha(x) \in \psi_\alpha(V_\alpha)\). Lo spazio tangente \(T_uE\) ad \(E\) in \(u\) è l'immagine della mappa lineare, iniettiva e continua \((\dif \psi_\alpha)_x:\E_\alpha \to \mathbb{E}\).
\end{defi}
La definizione non dipende dalla scelta della parametrizzazione. Infatti sia \(\psi_\beta: V_\beta \subset \E_\beta \to \mathbb{E}\) un'altra parametrizzazione locale con \(u \in \psi_\beta(V_\beta)\) e consideriamo il diagramma commutativo
\[
\begin{tikzcd}[row sep=large]
	\mathbb{E}  \arrow[r,<->] & \mathbb{E} \\
	V_\alpha \cap(\psi_\alpha^{-1}(\psi_\beta(V_\beta)) \subset \E_\alpha \arrow[u, "\psi_\alpha"'] \arrow[r, "\psi_\beta^{-1} \circ \psi_\alpha"] & V_\beta \cap \psi_\beta^{-1}(\psi_\alpha(V_\alpha))\subset \E_\beta.   \arrow[u, "\psi_\beta"]
\end{tikzcd}
\]
Poiché \(\psi_\beta^{-1} \circ \psi_\alpha\) è un diffeomorfismo, si ha che \(\mathrm{Im}(\dif \psi_\alpha)_y = \mathrm{Im} (\dif \psi_\beta)_{y'}\) per ogni \(y \in V_\alpha \cap \psi_\alpha^{-1}(\psi_\beta(V_\beta))\) e \(y' = \psi_\beta^{-1} \circ \psi_\alpha(y)\). Inoltre, \(\E_\alpha\) ed \(\E_\beta\) sono isomorfi.

Dalla definizione segue subito che \(T_u E\), con la norma di \(\E\) ristretta, è uno spazio di Banach isomorfo a \(\E_\alpha\), e quindi un sottospazio vettoriale chiuso di \(\mathbb{E}\). Se \(E\) è una varietà di Hilbert, per ogni \(u \in E\) possiamo decomporre \(\mathbb{E}= \mathbb{H}\) nella somma diretta dei due sottospazi vettoriali chiusi \(\mathbb{H} = T_uE \oplus N_uE\), con 
\[
	N_uE \coloneq (T_uE)^\perp
\]
lo \textit{spazio normale} ad \(E\) in \(u\).

\begin{defi}
	Il \textit{fibrato tangente} di \(E\) è l'insieme 
	\[
		TE \coloneq \{(u,v) \in E \times \mathbb{E} \ | \ v \in T_uE\}
	\]
	dotato della proiezione sulla prima componente \(\pi:TE \to E\). 
\end{defi}

Prendendo i differenziali delle parametrizzazioni locali, è immediato osservare che \(TE \subset \mathbb{E} \times \mathbb{E}\) è una varietà di Banach. 

\begin{defi}
	Sia \(f: E' \subset \mathbb{E}' \to E \subset \mathbb{E}\) una mappa liscia tra due varietà di Banach. Il \textit{differenziale} di \(f\) è la mappa \(\dif f: TE' \to TE\) definita dal seguente diagramma commutativo 
	\[
	\begin{tikzcd}[row sep={tiny}, column sep=large]
		TE' \arrow[r,"\dif f"] & TE \\
		\arrow[phantom, "\strut"] & \\
		\arrow[phantom, "\strut"] & \\
		V'_a \times \E'_a \arrow[uuu,"\psi'_a"] \arrow[r,"\dif (\psi_\alpha^{-1} \circ f \circ \psi'_a)"] & V_\alpha \times \E_\alpha \arrow[uuu,"\psi_\alpha"'] \\
		(x,v) \arrow[r, mapsto] & ((\psi_\alpha^{-1} \circ f \circ \psi'_a) (x),\dif (\psi_\alpha^{-1} \circ f \circ \psi'_a) (x) [v])
	\end{tikzcd}
	\]
	per ogni parametrizzazione locale \(\psi'_a:V'_a \subset \E'_a \to \mathbb{E}'\) di \(E'\) e \(\psi_\alpha:V_\alpha \subset \E_\alpha \to \mathbb{E}\) di \(E\) tale che \(\psi'_a(V'_a) \cap f^{-1}(\psi_\alpha(V_\alpha)) \neq \emptyset\).
\end{defi}

Da adesso assumiamo che \(E \subset \mathbb{H}\) sia una varietà di Hilbert. È possibile dotare \(E\) di una metrica \(g\), cioè di un'applicazione che associa a ogni punto \(u \in E\) un prodotto scalare \(g_u\) su \(T_uE\), restringendo il prodotto scalare di \(\mathbb{H}\)
\[
	g_u(X,Y) \coloneq \langle X, Y \rangle \qquad \forall X,Y \in T_uE.
\]

\begin{defi}
	Sia \(f: E \to \R\) un funzionale \(C^\infty\). Il \textit{gradiente} di \(f\) è il campo vettoriale liscio
	\[
		\grad f: E \to TE
	\]
	che a ogni \(u \in E\) associa il vettore \(\grad f(u) \in T_uE\) che rappresenta il funzionale lineare \((\dif f)_u:T_uE \to \R\) rispetto al prodotto scalare \(g_u\):
	\[
		(\dif f)_u[v] = g_u(\grad f(u), v) \qquad \forall v \in T_uE.
	\]
\end{defi}

Come per le sottovarietà Riemanniane, l'ambiente \(\mathbb{H}\) induce un'ulteriore struttura su \(E\), il \textit{fibrato normale}
\[
	NE \coloneq \{(u,v) \in E \times \mathbb{H} \ | \ v \in N_uE\},
\]
che si può vedere facilmente essere una varietà di Hilbert contenuta in \(\mathbb{H}\times \mathbb{H}\). Inoltre, sono naturalmente definite la \textit{proiezione tangenziale} 
\[\pi^\parallel: E \times \mathbb{H} \to TE\]
e la \textit{proiezione normale} 
\[\pi^\perp: E \times \mathbb{H} \to NE.\]
Talvolta, per \(u \in E\) fissato, scriveremo \(\pi^\parallel_u: \mathbb{H} \to T_uE\) e \(\pi^\perp_u:\mathbb{H} \to N_uE\) per le restrizioni a \(\{u\} \times \mathbb{H} \equiv \mathbb{H}\).

\begin{oss}\label{oss: estensioni di funzioni}
	Supponiamo che \(f:E \to \R\) si estenda a un funzionale liscio \(\widetilde{f}: U \to \R\), con \(U\) intorno aperto di \(E\) in \(\mathbb{H}\). Allora \(f\) è un funzionale liscio su \(E\) e il differenziale è la restrizione di \(\dif \widetilde{f}\) a \(TE\). Inoltre, se \(\grad \widetilde{f} \in C^\infty(U,\mathbb{H})\) è il gradiente di \(\widetilde{f}\), allora per ogni \(u \in E\)
	\[
		\grad f(u) = \pi^\parallel (u,\grad \widetilde{f}) .
	\]
\end{oss}

\section{La varietà di Hilbert \(H^1(\Sp^1,M)\)}\label{sez: H^1(S,M)}

Sia \(M \subset \R^N\) una sottovarietà Riemanniana di \(\R^N\) (dotato della metrica piatta). Questa ipotesi non è restrittiva nel senso del Teorema~\ref{teo: embedding isometrico di Nash}: tutto quello che segue può essere dimostrato anche in modo intrinseco, cioè senza assumere un embedding in \(\R^N\) (si veda \cite[Chapter~1]{klingenberg2012lectures} oppure \cite[Section~2.3]{klingenberg1995riemannian}). 

Definiamo lo \textit{spazio delle curve chiuse di classe \(H^1\)} come
\[
	H^1(\Sp^1,M) \coloneq \{ \gamma \in H^1(\Sp^1,\R^N) \ | \ \gamma(t) \in M \ \forall t \in \Sp^1\}
\]
Per ogni \(\gamma \in H^1(\Sp^1,M)\) è definito quasi ovunque il vettore tangente debole \(\dot \gamma \in L^2(I,\R^N)\). È ancora lecito chiamare \(\dot \gamma\) ‘‘vettore tangente''. Infatti, sia \((\gamma_h)_h \subset C^\infty(\Sp^1,M)\) tale che \(\gamma_h \to \gamma \quad \text{in }H^1(\Sp^1,\R^N)\).
Una conseguenza della dimostrazione del teorema di Riesz-Fischer è che, a meno di estrarre una sottosuccessione, \((\gamma_h(t),\dot \gamma_h(t)) \to (\gamma(t),\dot \gamma(t))\) per quasi ogni \(t \in I\) (cf. \cite[Theorem 4.9]{brezis2011functional}). Poiché \(TM \subset \R^N \times \R^N\) è chiuso, \(\dot \gamma(t) \in T_{\gamma(t)}M\) per q.o. \(t \in I\).

\begin{prop}\label{prop: approssimazione con curve lisce}
	Per ogni \(\gamma \in H^1(\Sp^1,M)\) esistono \(\gamma_0 \in C^\infty(I,M)\) e \(X \in H^1(\Sp^1,\R^N)\) tale che \(X(t) \in T_{\gamma_0(t)}M\) per ogni \(t \in \Sp^1\) e
	\[
		\gamma(t) = \exp_{\gamma_0(t)}(X(t)).
	\]
\end{prop}
\begin{proof}
	Dato che \(\Sp^1\) è compatto, come nel Corollario~\ref{cor: epsilon per compatta} possiamo trovare \(\ve >0\) tale che \(r(\gamma(t))> \ve\) per ogni \(t \in \Sp^1\). Poiché \(C^\infty(\Sp^1,M)\) è denso in \(C^0(\Sp^1,M)\), esiste \(\gamma_0 \in C^\infty(\Sp^1,M)\) tale che
	\[
	d(\gamma(t),\gamma_0(t)) < \ve \qquad \forall t \in \Sp^1.
	\]
	In particolare, per ogni \(t \in \Sp^1\) esiste un unico \(X(t) \in T_{\gamma_0(t)}M\) tale che
	\[
	\gamma(t) = \exp_{\gamma_0(t)}(X(t)).
	\]
	Che \(X \in H^1(\Sp^1,\R^N)\) segue dal fatto che la mappa esponenziale è un diffeomorfismo. 
\end{proof}

\begin{oss}\label{oss: H1 varietà di Hilbert}
La Proposizione~\ref{prop: approssimazione con curve lisce} permette di dare ad \(H^1(\Sp^1,M)\) una struttura di varietà di Hilbert. Infatti basta scegliere, al variare di \(\gamma \in C^\infty(\Sp^1,M)\), l'atlante indotto dalle parametrizzazioni locali
\[
\exp_\gamma : \mathcal{B}(\gamma) \to H^1(\Sp^1,\R^N)
\]
con 
\[
	\mathcal{B}(\gamma) \coloneq \left\{X \in H^1(\Sp^1,\R^N) \ \middle| \ 
	\begin{aligned}
		&X(t) \in T_{\gamma(t)}M, \\
		&|X(t)|_{\gamma(t)} < \ve_\gamma
	\end{aligned}
	\quad  \forall t \in \Sp^1
	\right\},
\]
dove \(\ve_\gamma \coloneq \inf_{t \in \Sp^1} r(\gamma(t))\), e
\[
\exp_\gamma(X)(t) = \exp_{\gamma(t)}(X(t))
\]
per ogni \(X \in \mathcal{B}(\gamma)\), \(t \in \Sp^1\). 

Lo spazio tangente a \(H^1(\Sp^1,M)\) in \(\gamma\) è
\[
	T_\gamma H^1(\Sp^1,M) = \{ X \in H^1(\Sp^1,\R^N) \ | \ X(t) \in T_{\gamma(t)}M \ \forall t \in \Sp^1\}.
\]
\end{oss}

\begin{defi}
	La \textit{derivata covariante debole} lungo \(\gamma\) di un campo \(V \in T_\gamma H^1(\Sp^1,M)\) è 
	\[
		D_\gamma V \coloneq \dot V - \sff(\dot \gamma, V),
	\]
	dove \(\sff\) è la seconda forma fondamentale di \(M\). 
\end{defi}
\begin{oss}
	Dalla (\ref{eq: formula Gauss curve in M in R^N}) segue che la derivata covariante debole coincide con quella calssica quando \(V \in \T(\gamma)\). Inoltre vale la seguente formula di integrazione per parti: per ogni \(W \in \T(\gamma)\),
	\[
		\int_{\Sp^1} \langle V, D_\gamma W \rangle_\gamma \ \dif t = - \int_{\Sp^1} \langle D_\gamma V, W \rangle_\gamma \ \dif t.
	\]
\end{oss}

\section{Il funzionale energia}

Su \(H^1(\Sp^1,M)\) è ben definito il \textit{funzionale energia}
\[
	E(\gamma) \coloneq \frac{1}{2}\int_{\Sp^1} |\dot \gamma |^2 \ \dif t, \qquad \gamma \in H^1(\Sp^1,M).
\]

\begin{teo}\label{teo: energia è liscia}
	Il funzionale energia è di classe \(C^\infty\) e, per ogni \(\gamma \in H^1(\Sp^1,M)\),
	\[
	\dif E(\gamma) [W] = \int_{\Sp^1} \langle D_\gamma W, \dot \gamma \rangle \ \dif t, \quad \forall W \in T_\gamma H^1(\Sp^1,M).
	\]
\end{teo}
\begin{proof}
	Consideriamo l'estensione \(\widetilde{E}:H^1(\Sp^1,\R^N) \to \R\) definita da
	\[
	\widetilde{E}(u) \coloneq \frac{1}{2}\|\dot u \|^2_{L^2(I,\R^N)}, \qquad u \in H^1(\Sp^1,\R^N).
	\]
	Osserviamo che \(\widetilde{E}\) è la composizione di due mappe \(C^\infty\), ovvero \(\widetilde{E}=q \circ T\) dove
	\begin{align*}
		T: H^1(\Sp^1,\R^N) &\to L^2(I,\R^N) \\
		u &\mapsto \dot u,
	\end{align*}
	è lineare e continua, e in particolare ha come differenziale la mappa costante \(\dif T(u)=T\), e quindi \(T\) è di classe \(C^\infty\);
	\begin{align*}
		q: L^2(\Sp^1,\R^N) &\to \R \\
		u &\mapsto \frac{1}{2} \| u \|^2_{L^2}.
	\end{align*}
	ha come gradiente \(\nabla q = \id_{L^2(I,\R^N)}\), e quindi \(q \in C^\infty\). Di conseguenza anche \(\widetilde{E}=q \circ T\) è \(C^\infty\) e il differenziale è dato da
	\[
		\dif \widetilde{E}(u)[v] = \langle \dot u, \dot v \rangle_{L^2(I,\R^N)}.
	\]
	
	Per l'Osservazione~\ref{oss: estensioni di funzioni}, \(E= \widetilde{E}|_{H^1(\Sp^1,M)}\) è un funzionale \(C^\infty\) su \(H^1(\Sp^1,M)\) e il differenziale è dato da
	\begin{align*}
		\dif E(\gamma)[W] &= \langle \dot \gamma, \dot W  \rangle_{L^2(I,\R^N)} = \int_{\Sp^1} \langle \dot \gamma, D_\gamma W + \sff(\dot \gamma,W)\rangle \ \dif t \\
		&= \int_{\Sp^1} \langle \dot \gamma, D_\gamma W\rangle  \ \dif t.
	\end{align*}
	
\end{proof}

\begin{oss}\label{oss: soluzioni deboli dell'equazione delle geodetiche}
	I punti critici di \(E\) sono esattamente le soluzioni deboli dell'equazione delle geodetiche o, in altre parole, l'equazione delle geodetiche è l'\textit{equazione di Eulero-Lagrange} associato al funzionale energia. Infatti, se \(\gamma \in C^2(\Sp^1,M) \) è un punto critico di \(E\), allora per ogni \(W \in \T(\gamma) \cap T_\gamma H^1(\Sp^1,M)\)
	\[
	 0=\int_{\Sp^1} \langle D_\gamma W, \dot \gamma \rangle \ \dif t = - \int_{\Sp^1} \langle W, D_\gamma \dot \gamma \rangle \ \dif t
	\]
	e per il lemma fondamentale del calcolo delle variazioni \(D_\gamma \dot \gamma = 0\).
	
\end{oss}

\begin{lemma}\label{lemma: regolarità punti critici di E}
	I punti critici di \(E\) sono contenuti in \(C^\infty(\Sp^1,M)\).
\end{lemma}
\begin{proof}
	Sia \(\gamma \in H^1(\Sp^1,M)\) un punto critico dell'energia. Verifichiamo che \(\gamma\) risolve l'equazione 
	\[
		\ddot \gamma = \sff(\dot \gamma,\dot \gamma)
	\] 
	in senso debole, cioè che per ogni \(v \in C^\infty_0(I,\R^N)\)
	\begin{align*}
		\int_0^1 \langle \dot \gamma, \dot v \rangle\ \dif t = - \int_0^1 \langle \sff(\dot \gamma,\dot\gamma), v \rangle \ \dif t.
	\end{align*}
	Siano \(\pi^\parallel\) e \(\pi^\perp\) la proiezione tangente e la proiezione ortogonale di \(H^1(\Sp^1,M) \subset H^1(\Sp^1,\R^N)\). Scomponendo \(v = \pi^\parallel_{\gamma}(v)+\pi^\perp_{\gamma}(v) = v^\parallel + v^\perp\) e applicando la (\ref{eq: Weingarten curve in M in R^N}) (che vale puntualmente quasi ovunque, indipendentemente dalla regolarità di \(\gamma\)),
	\begin{align*}
		\int_0^1 \langle \dot \gamma, \dot v \rangle \ \dif t &= \int_0^1 \left\langle \dot \gamma, \frac{\dif v^\parallel}{\dif t} \right\rangle \ \dif t +\int_0^1\left \langle \dot \gamma, \frac{\dif v^\perp}{\dif t} \right\rangle \ \dif t  \\
		&=-\int_0^1 \langle \sff(\dot \gamma,\dot \gamma), v^\perp \rangle \ \dif t \\
		&=-\int_0^1 \langle \sff(\dot \gamma,\dot \gamma), v\rangle \ \dif t.
	\end{align*}
	
	Dunque \(\dot \gamma = \sff(\dot \gamma,\dot \gamma)\). Osserviamo che, siccome \(\dot \gamma \in L^2(I,\R^N)\), si ha \(\ddot \gamma = \sff(\dot \gamma,\dot \gamma) \in L^1(I)\). Infatti sia  \(E_1,\dots,E_N\) un frame ortonormale locale di \(\R^N\) definito su \(\gamma((t_0-\ve,t_0+\ve))\) e tale che \(E_1, \dots, E_n\) sia un frame ortonormale locale di \(M\); allora, usando ancora la (\ref{eq: Weingarten curve in M in R^N}),
	\begin{align*}
		\int_{t_0-\ve}^{t_0+\ve} \left| \sff(\dot \gamma, \dot \gamma) \right| \ \dif t &= \int_{t_0-\ve}^{t_0+\ve} \sqrt{\sum_{i=n+1}^N \langle \sff(\dot \gamma, \dot \gamma), E_i \rangle^2} \ \dif t \\
			&\leq \sum_{i=n+1}^N \int_{t_0-\ve}^{t_0+\ve} \left| \langle \sff(\dot \gamma, \dot \gamma), E_i \rangle \right| \ \dif t \\
			&= \sum_{i=n+1}^N \int_{t_0-\ve}^{t_0+\ve} \left| \langle \dot \gamma, \dot E_i \rangle \right| \ \dif t \\
			& \leq \sum_{i=n+1}^N \|\dot \gamma \|_{L^2} \|\dot E_i \|_{L^2} < + \infty.
	\end{align*}
	Siccome \(\gamma(I)\) è compatto, basta sommare questa stima su un numero finito di punti di \(\Sp^1\) per ottenere \(\|\sff(\dot \gamma, \dot \gamma)\|_{L^1(I, \R^N)} < +\infty\).
	
	Dunque \(\dot \gamma \in W^{1,1}(I,\R^N) \hookrightarrow C^0(I,\R^N)\) per il Teorema~\ref{teo: immersione di Sobolev}. Allora anche \(\sff(\dot \gamma,\dot \gamma)\) è continuo, e quindi \(\ddot \gamma \in C^0(I,\R^N)\). Dunque \(\gamma \in C^2(I,\R^N)\) e, in ogni carta, risolve l'equazione delle geodetiche (\ref{eq: equazione geodetiche}), quindi per il Teorema~\ref{teo: esistenza e unicità locale sol pbm Cauchy} \(\gamma \in C^\infty(I,M)\). Per concludere che \(\gamma \in C^\infty(\Sp^1,M)\), basta applicare lo stesso ragionamento alla curva
	\[
		t \mapsto \gamma(t + 1/2).
	\] 
\end{proof}

\begin{teo}\label{teo: punto critico energia se e solo se geodetica}
	Una curva \(\gamma \in H^1(\Sp^1,M)\) è un punto critico dell'energia se e solo se è una geodetica (possibilmente costante).
\end{teo}
\begin{proof}
	Grazie al Lemma~\ref{lemma: regolarità punti critici di E}, i punti critici di \(E\) sono tutte e sole le curve \(\gamma \in C^\infty(\Sp^1,M)\) che soddisfano l'equazione delle geodetiche. Concludiamo con l'Osservazione~\ref{oss: soluzioni deboli dell'equazione delle geodetiche}.
\end{proof}

	
	\chapter{Il teorema di passo montano}
	
In questo capitolo affronteremo uno strumento variazionale per individuare punti critici di funzionali: il teorema di passo montano, di A. Ambrosetti e P. Rabinowitz \cite{ambrosetti1973dual}. Enunceremo una versione leggermente più generale di quella enunciata in \cite[Section.81]{ambrosetti2007nonlinear}: racchiude più situazioni geometriche e avrà una forma più adatta per essere applicata alla dimostrazione del teorema di Lusternik-Fet. Anche se con delle ipotesi leggermente diverse, la dimostrazione è sostanzialmente la stessa ed è basata soprattutto su \cite[Chapter~7]{ambrosetti2007nonlinear}.

\section{Passo montano su spazi di Hilbert}

	
	In questa sezione consideriamo uno spazio di Hilbert \((\E, \langle \cdot, \cdot \rangle)\) e una funzione \(f: \E \to \R\) di cui vogliamo individuare dei valori critici (assunti da punti critici). 
	Assumiamo che \(f\) sia di classe \(C^{1,1}_{loc}\), ovvero di classe \(C^1\) (nel senso di Fréchet, cf. \cite[Definition~1.1]{ambrosetti2007nonlinear}) e che il gradiente \(\grad f\) sia un campo continuo localmente Lipschitziano, cioè che per ogni \(u_0 \in \E\) esista un intorno \(U=U(u_0) \subset \E\) di \(u_0\) e una costante \(L=L(u_0)>0\) tale che
	\[
	\forall u,v \in U, \quad |f(u)-f(v)| \leq L \|u-v\|,
	\]
	dove \(\|\cdot\|\) è la norma indotta dal prodotto scalare \(\langle \cdot,\cdot\rangle\). 
	
	Per enunciare il teorema, abbiamo bisogno di studiare una particolare condizione di compattezza, la condizione di Palais-Smale.
	
	\subsection{Condizione di Palais-Smale}
	
	\begin{defi}
		Una successione \((u_h)_h \subset \E\) si dice una (PS)-successione della funzione \(f\) se la successione \((f(u_h))_h \subset \R\) è limitata e 
		\[
		\grad f (u_h) \to 0.
		\]
		
		La funzione \(f\) si dice di Palais-Smale (o che soddisfa la \textit{condizione di Palais-Smale} (PS)) se ogni sua (PS)-successione ammette una sottosuccessione convergente. 
	\end{defi}
	
	La condizione di Palais-Smale permette di controllare il comportamento del gradiente di \(f\) intorno ai livelli di valori regolari.
	
	\begin{lemma} \label{lemma: PS -> controllo sul gradiente}
		Supponiamo che \(f\) soddisfi (PS) e che \(\lambda \in \R\) sia un valore regolare. Allora esiste \(\delta >0\) tale che per ogni \(u \in \E\)
		\[
		|f(u)-\lambda| \leq \delta \implies \|\grad f(u)\| \geq \delta.
		\]
	\end{lemma}
	\begin{proof}
		Supponiamo per assurdo che per ogni \(h \in \N\) esista \(u_h \in \E\) tale che \(\|\grad f(u_h) \| < 1/h\) e \(|f(u_h)-\lambda | \leq 1/h\). Allora \(u_h\) è una (PS)-successione, e quindi a meno di estrarre una sottosuccessione converge a qualche \(u \in \E\). Per continuità, \(\grad f(u) =0\) e \(f(u)=\lambda\), contraddicendo la regolarità di \(\lambda\). 
	\end{proof}
	
	\subsection{Teorema di passo montano} \label{subs: MP Hilbert}
	
	Consideriamo il flusso \(\phi: \R \times \E \to \E\) associato all'equazione differenziale
	\begin{equation}\label{eq: flusso di f}
		\dot u = \frac{-\grad f(u)}{1+ \|\grad f (u)\|}.
	\end{equation}
	Il flusso è definito globalmente perché il campo vettoriale
	\[
	W(u) \coloneq \frac{-\grad f(u)}{1+ \|\grad f (u)\|}
	\]
	è localmente Lipschitziano e limitato. Infatti, essendo localmente Lipschitziano è garantita l'esistenza e unicità locale della soluzione di (\ref{eq: flusso di f}) dipendente in maniera continua dal dato inziale (il Teorema~\ref{teo: esistenza e unicità locale sol pbm Cauchy} si generalizza a spazi di Banach sostanzialmente con la stessa dimostrazione \cite{ambrosetti1967esistenza}). Supponiamo per assurdo che per il dato iniziale \(u_0 \in \E\) l'intervallo aperto di definizione massimale della soluzione \((a,b)\) sia limitato. Senza perdita di generalità supponiamo che \(b<+\infty\). Per ogni successione \((t_j)_j \subset (a,b)\) tale che \(t_j \to b\), osserviamo che la successione \(\phi(t_j,u_0)\) è di Cauchy:
	\[
	\|\phi(t_j,u_0) - \phi(t_i,u_0) \| \leq  \int_{t_i}^{t_j} \|W(\phi(t,u_0))\| \ \dif t \leq |t_i-t_j|
	\]
	perché \(\|W\| \leq 1\). Dato che \(\E\) è completo,  \(\phi(t_j,u_0) \to v_0\) per qualche \(v_0 \in \E\). Allora possiamo estendere la soluzione considerando la soluzione locale del problema 
	\[
	\begin{cases}
		\dot v = W(v) \\
		v(b) = v_0
	\end{cases}
	\]
	contro la massimalità dell'intervallo \((a,b)\). Quindi ogni soluzione è definita su \(\R\) e il flusso è definito su \(\R \times \E\).
	
	Un'altra osservazione importante è che, siccome \(W\) è parallelo a \(-\grad f\), le orbite di \(\phi\) 
	\[
	\sigma(u_0) \coloneq \{\phi(t,u_0) : t \in \R\}
	\]
	coincidono con le orbite associate al campo \(-\grad f\), cioè coincide con il supporto della soluzione massimale del problema di Cauchy
	\[
	\begin{cases}
		\dot u = -\grad f(u) \\
		u(0) = u_0.
	\end{cases}
	\]
	Questo vuol dire che \(f\) decresce lungo \(\phi\): sarà l'idea chiave della dimostrazione del teorema di passo montano. 
	
	Per ogni \(t \in \R\), denotiamo \(\phi_t : \E\to \E\) la mappa \(u \mapsto \phi(t,u)\).
	
	\begin{defi}
		Una famiglia non vuota \(\mathcal{H}\) di sottoinsiemi di \(\E\) è detta \textit{\(\phi\)-invariante} se 
		\[
		\phi_t(H) \in \mathcal{H} \quad \forall H \in \mathcal{H}, \ t \geq 0.
		\]
	\end{defi}
	
	Possiamo enunciare:
	\begin{teo}[passo montano, Ambrosetti-Rabinowitz, 1973]\label{teo: MP}
		Sia \(\E\) uno spazio di Hilbert e \(f \in C^{1,1}_{loc}(\E)\) una funzione di Palais-Smale. Sia \(\mathcal{H}\) una famiglia \(\phi\)-invariante tale che
		\[
		\lambda \coloneq \inf_{H \in \mathcal{H}} \sup_{u \in H} f(u) \in \R.
		\]
		Allora \(\lambda\) è un valore critico per \(f\). 
	\end{teo}
	
	\begin{lemma}\label{lemma: di deformazione}
		Sia \(f\) di Palais-Smale e supponiamo che \(\lambda\) sia un valore regolare per \(f\). Allora esistono \(\delta>0\) e \(T>0\) tale che per ogni \(u \in \E\)
		\[
		f(u)\leq \lambda +\delta \implies f(\phi_T(u)) \leq \lambda -\delta.
		\]
	\end{lemma}
	\begin{proof}
		Sia \(\delta>0\) dato dal Lemma~\ref{lemma: PS -> controllo sul gradiente}, cioè tale che per ogni \(u \in \E\)
		\[
		|f(u)-\lambda| \leq \delta \implies \|\grad f(u)\| \geq \delta.
		\]
		Per ogni \(u \in \E\), denotiamo con \(\phi_u: \R \to \E\) la mappa \( t \mapsto \phi(t,u) \).
		Poiché
		\[
		\frac{\dif f \circ \phi_u}{\dif t}(t) = \dif f (\phi(t,u))[\dot \phi_u(t)] = - \frac{\|\grad f(\phi(t,u))\|^2}{1+\|\grad f(\phi(t,u))\|},
		\]
		valgono:
		\begin{enumerate}[label=(\roman*)]
			\item la funzione \(f \circ \phi_u\) è decrescente;
			\item per ogni \(a,b \in [0,+\infty)\),
			\[
			f(\phi(b,u))-f(\phi(a,u)) = -\int_a^b \frac{\|\grad f(\phi(t,u))\|^2}{1+\|\grad f(\phi(t,u))\|} \ \dif t.
			\]
		\end{enumerate}
		
		Sia \(c=\delta^2/(1+\delta)\). Osserviamo che per ogni \(u \in \E\)
		\[
		|f(u)-\lambda| \leq \delta \implies \frac{\|\grad f(u)\|^2}{1+\|\grad f(u)\|} \geq c,
		\]
		perché la funzione \(s \mapsto s^2/(1+s)\) è non decrescente su \([0,+\infty)\).
		
		%		\begin{figure}[h]
	\centering
	\begin{tikzpicture}
		\begin{axis}
			[xmin=-0.5, xmax=4.5, ymin=-0.5,ymax=4.5,
			xtick={2}, ytick={4/3},
			xticklabels={\(\delta\)},
			yticklabels={\(c\)},
			xlabel =$s$,
			axis lines = middle]
			\addplot [domain = 0:4, samples = 15, thick, smooth, color = black]{x^2/(x+1)};
			\draw[dotted] (axis cs: 0,4/3) to (axis cs: 2,4/3);
			\draw[dotted] (axis cs: 2,0) to (axis cs: 2,4/3);
		\end{axis}
	\end{tikzpicture} 
	\caption{Grafico di \(s \mapsto s^2/(1+s)\)}  
\end{figure}
		
		Sia \(T= 2\delta/c\) e supponiamo per assurdo che esista \(u \in \E\) tale che \(f(u)\leq \lambda +\delta\) e \(f(\phi_T(u)) > \lambda -\delta\). Da (i) segue che
		\[
		\lambda-\delta < f(\phi(t,u)) \leq \lambda+\delta \quad \forall t \in [0,T].
		\]
		Usando (ii) con \(a=0\) e \(b=T\),
		\begin{align*}
			f(\phi_T(u)) &= f(u) - \int_0^T \frac{\|\grad f(\phi(t,u))\|^2}{1+\|\grad f(\phi(t,u))\|} \ \dif t \\
			& \leq \lambda +\delta - T c \\
			&= \lambda -\delta 
		\end{align*}
		in contraddizione con quanto trovato prima. Dunque per ogni \(u \in \E\)
		\[
		f(u)\leq \lambda +\delta \implies f(\phi_T(u)) \leq \lambda -\delta.
		\]
		
	\end{proof}
	
	\begin{proof}[Dimostrazione del Teorema~\ref{teo: MP}]
		Supponiamo per assurdo che \( \lambda \) sia un valore regolare. Per il Lemma~\ref{lemma: di deformazione} esistono \(\delta>0\) e \(T>0\) tale che per ogni \(u \in \E\)
		\[
		f(u)\leq \lambda +\delta \implies f(\phi_T(u)) \leq \lambda -\delta.
		\]
		Sia \(H \in \mathcal{H}\) tale che \(\sup_H f \leq \lambda + \delta\). Allora 
		\[
		\sup_{u \in \phi_T(H)} f(u) = \sup_{v \in H} f(\phi_T(v)) \leq \lambda -\delta 
		\]
		e siccome \(\phi_T(H) \in \mathcal{H}\), questa è una contraddizione. 
	\end{proof}


\section{Passo montano su \texorpdfstring{$H^1(\Sp^1,M)$}{H1(S1,M)}}\label{sez: MP con energia}

Nella dimostrazione del Lemma~\ref{lemma: di deformazione} e del Teorema~\ref{teo: MP}, non abbiamo mai usato veramente la struttura di spazio vettoriale, ma solo:
\begin{itemize}
	\item risultati di esistenza, unicità e prolungamento del problema di Cauchy associato a una ODE;
	\item regolarità e gradiente della funzione \(f\) di cui cerchiamo i punti critici.
\end{itemize}
Dunque la dimostrazione vale anche per il caso di varietà di Hilbert \(E \subset \mathbb{H}\) e funzionali \(f \in C^\infty(E)\) con gradiente \(\grad f: E \to \mathbb{H}\) localmente Lipschitziano. Vediamo nel dettaglio il caso dello spazio \(H^1(\Sp^1,M)\) e del funzionale energia \(E\) associati a una varietà chiusa \(M \subset \R^N\). Dimostrare la condizione di Palais-Smale per una funzione è in generale un passaggio tecnico non immediato. Assumendo \(M \subset \R^N\), possiamo usare la seconda forma fondamentale (cf. Sezione~\ref{sez: sottovarietà}) per generalizzare la dimostrazione proposta in \cite[Theorem~4.4]{struwe2008variational} per il teorema di Birkhoff sull'esistenza di geodetiche chiuse su una sfera contenuta in \(\R^3\). 

Ricordiamo che \(E\) è esteso naturalmente dal funzionale liscio
\[
	\widetilde{E}(u) = \frac{1}{2} \|\dot u \|_{L^2(I,\R^N)}^2,
\]
che ha gradiente Lispchitziano. Infatti, per ogni \(u,v \in H^1(\Sp^1,\R^N)\),
\begin{align*}
	\| \grad \widetilde{E}(u)-\grad \widetilde{E}(v)\|_{H^1} &= \| \dif \widetilde{E}(u)-\dif \widetilde{E}(v)\|_{(H^1)^*} \\
	&\leq \|\dot u- \dot v\|_{L^2} \\
	&\leq \|u-v\|_{H^1}.
\end{align*}
In particolare anche \(\grad E = \pi^\parallel(\grad \widetilde{E})\) è Lipschitziano.

Posto
\[
	W = - \frac{\grad E}{1 + \|\grad E\|_{H^1}},
\]
possiamo ripetere tutti i ragionamenti dell'inizio della sottosezione~\ref{subs: MP Hilbert} per garantire l'esistenza del flusso 
\[
	\phi:\R \times H^1(\Sp^1,M) \to H^1(\Sp^1,M)
\]
associato all'equazione
\[
	\frac{\dif \alpha}{\dif t}(t) = W(\alpha(t)).
\]
Infatti, basta osservare che le orbite sono contenute in \(H^1(\Sp^1,M)\) perché \(W\) è sempre un vettore tangente a \(H^1(\Sp^1,M)\). Inoltre, siccome \(E\) è di classe \(C^\infty\), anche \(\phi\) è di classe \(C^\infty\).

Denotiamo per ogni \(t \in \R\) con \(\phi_t:H^1(\Sp^1,M) \to H^1(\Sp^1,M)\) la mappa 
\[
\phi_t(u) \coloneq \phi(t,u).
\]

\begin{defi}
	Una famiglia non vuota \(\mathcal{H}\) di sottoinsiemi di \(H^1(\Sp^1,M)\) è detta \textit{\(\phi\)-invariante} se 
	\[
		\phi_t(H) \in \mathcal{H} \quad \forall H \in \mathcal{H}, \ t \in \R.
	\]
\end{defi}

Dobbiamo verificare che \(E\) soddisfi la condizione di Palais-Smale. 
\begin{lemma}
	Sia \((\gamma_h)_h \subset H^1(\Sp^1,M)\) una successione tale che \((E(\gamma_h))_h \subset \R\) sia limitata e 
	\begin{equation*}
		\|\grad E(\gamma_h)\|_{H^1} \to 0.
	\end{equation*}
	Allora esiste una sottosuccessione corvengente. 
\end{lemma}
\begin{proof}
	Siccome \(M\) è compatta in \(\R^N\), è anche limitata, quindi esiste \(c_1>0\) tale che
	\[
		\|\gamma\|_{L^2(\Sp^1,\R^N)} \leq \|\gamma\|_{C^0(\Sp^1,\R^N)} \leq c_1
	\] 
	per ogni \(\gamma \in H^1(\Sp^1,M)\). Siccome \(2E(\gamma) = \|\dot \gamma\|_{H^0}\) e \((E(\gamma_h))_h\) è limitata, esiste \(c_2>0\) tale che per ogni \(h \in \N\)
	\[
		\|\gamma_h\|_{H^1(\Sp^1,\R^N)} \leq c_2.
	\]
	Per il Teorema~\ref{teo: immersione Sobolev per S^1}, esistono una sottosuccessione \((\gamma_{h_k})_k\) e un elemento \(\gamma \in H^1(\Sp^1,\R^N)\) tali che
	\begin{itemize}
		\item \(\dot \gamma_{h_k} \rightharpoonup \dot \gamma\) in \(H^1(I,\R^N)\),
		\item \(\gamma_{h_k} \to \gamma\) in \(C^0(\Sp^1,\R^N)\).
	\end{itemize}
	In particolare \(\gamma_{h_k} \to \gamma\) in \(L^2(\Sp^1,\R^N)\). Inoltre, poiché \(M\) è chiuso in \(\R^N\) e il limite uniforme coincide con il limite puntuale, \(\gamma(t) \in M\) per ogni \(t \in \Sp^1\), e quindi \(\gamma \in H^1(\Sp^1,M)\).
	
	Resta da provare che \(\dot \gamma_{h_k} \to \dot \gamma\) in \(L^2\). Scriviamo \(v_k = \gamma_{h_k}-\gamma\).
	Siano \(\pi^\parallel_\gamma\) e \(\pi^\perp_\gamma\) le proiezioni di \(H^1(\Sp^1,\R^N)\) su \(T_\gamma H^1(\Sp^1,M)\) e su \(N_\gamma H^1(\Sp^1,M)\) rispettivamente. 
	
	%	\begin{figure}[h]
	\centering
		
\begin{tikzpicture}
	
	% Drawing the axes
	\draw[->] (-0.5, 0) -- (5, 0) 
	node[above, align=center] at (5,0.1) {\(\begin{aligned}
			&T_\gamma H^1(\Sp^1,M) \\
			&\cong H^1(\Sp^1,\R^n)
		\end{aligned}\)};
	\draw[->] (0, -0.5) -- (0, 4) 
	node[right, align=center] at (0.1,3.5) {\(\begin{aligned}
			&N_\gamma H^1(\Sp^1,M) \\
			&\cong H^1(\Sp^1,\R^{N-n})
		\end{aligned}\)};
	
	% Adding H^1(\Sp^1,\R^N) in the top right corner
	\node[below right, align=center] at (4.5, 3.5) {\(H^1(\Sp^1,\R^N)\)};
	
	% Point where the vector ends
	\coordinate (v) at (3,2);
	
	% Drawing the vector v
	\draw[->, thick] (0, 0) -- (v) node[midway, above left] {\(v\)};
	
	% Drawing the projection on x
	\coordinate (v_proj_x) at (3, 0);
	\draw[dashed] (v) -- (v_proj_x);
	\draw[->, thick] (0, 0) -- (v_proj_x) node[midway, below] {\(\pi^\parallel_\gamma(v)\)};
	
	% Drawing the projection on y
	\coordinate (v_proj_y) at (0, 2);
	\draw[dashed] (v) -- (v_proj_y);
	\draw[->, thick] (0, 0) -- (v_proj_y) node[midway, left] {\(\pi^\perp_\gamma(v)\)};
	
\end{tikzpicture}
		
	\caption{Le proiezioni \(\pi^\parallel_\gamma\) e \(\pi^\perp_\gamma\).}  
\end{figure}
	
	Consideriamo la successione
	\[
		w_k \coloneq \pi^\parallel_\gamma (v_k) \in T_\gamma H^1(\Sp^1,M).
	\]
	Essendo limitata, a meno di estrarre un'altra sottosuccessione, \(w_k\) converge a \(0\) rispetto alla topologia forte di \(C^0(\Sp^1,\R^N)\) e rispetto alla topologia debole di \(H^1(\Sp^1,\R^N)\). Quindi (si veda \cite[Proposition 3.5(iv)]{brezis2011functional}), 
	\begin{equation}\label{eq: PS1}
		\langle \grad E(\gamma_{h_k}), w_k \rangle_{H^1} \to 0. 
	\end{equation}
	Osserviamo che
	\begin{align}\label{eq: PS2}
		\langle \grad E(\gamma_{h_k}), w_k \rangle_{H^1}  &= \dif E(\gamma_{h_k}) w_k 
		= \int_0^1 \langle \dot \gamma_{h_k}, \dot w_k \rangle \ \dif t \nonumber \\ 
		&= \int_0^1 \langle \dot v_k, \dot w_k \rangle \ \dif t + \int_0^1\langle \dot \gamma, \dot w_k \rangle \ \dif t \nonumber\\
		&= \int_0^1 |\dot v_k |^2 \ \dif t - \int_0^1 \left\langle \dot v_k, \frac{\dif}{\dif t}\pi^\perp_\gamma(v_k) \right\rangle \ \dif t + o(1)
	\end{align}
	con \(o(1) \to 0\) quando \(k \to \infty\) e avendo usato che \(\dot w_k \rightharpoonup 0\) in \(L^2\) e che
	\[
		\dot w_k = \dot v_k - \frac{\dif}{\dif t}\pi^\perp_\gamma(v_k).
	\]
	Fissiamo \(t_0 \in \Sp^1\) e sia \(E_1 , \dots, E_n,E_{n+1}, \dots, E_N\) un frame ortonormale di \(\R^N\) su un intorno di \(\gamma(t_0)\) tale che \(E_1, \dots, E_n\) sia un frame ortonormale di \(M\). Scriviamo 
	\[
		v_k = \sum_{i=1}^N v_k^i (E_i)_\gamma.
	\]
	Poiché \(v_k \to 0\) in \(C^0\) e in \(L^2\), anche 
	\[
	v_k^i = \langle v_k, E_i \rangle_\gamma \to 0 \text{ in } C^0 \text{ e }L^2.
	\]
	Inoltre,  
	\[
		\dot v_k =  \sum_{i=1}^N \dot v_k^i (E_i )_\gamma + \sum_{i=1}^N v^i_k \frac{\dif}{\dif t} (E_i)_\gamma.
	\]
	e quindi, siccome \(\dot v_k \rightharpoonup 0\) in \(L^2\), anche
	\[
		\dot v_k^i = \langle \dot v_k, E_i \rangle - \sum_{j=1}^N v^j_k \left\langle \frac{\dif}{\dif t} (E_j)_\gamma , (E_i)_\gamma \right\rangle  \rightharpoonup 0 \text{ in }L^2.
	\]
	Osservando che
	\[
		\pi^\perp_\gamma (v_k) = \sum_{j=n+1}^N v_k^j (E_j)_\gamma,
	\]
	otteniamo che, se \(\gamma(t_0-\ve,t_0+\ve) \) è contenuto nell'intorno su cui è definito il frame, 
	\begin{align*}
		\int_{t_0-\ve}^{t_0+\ve} \left \langle \dot v_k, \frac{\dif}{\dif t} \pi^\perp_\gamma(v_k) \right\rangle \ \dif t &= \int_{t_0-\ve}^{t_0+\ve} \left\langle \sum_{i=1}^N \dot v^i_k (E_i)_\gamma + \sum_{i=1}^N v^i_k \frac{\dif}{\dif t} (E_i)_\gamma, \right. \\
		& \qquad \qquad \left.\sum_{j=n+1}^N \dot v^j_k (E_j)_\gamma + \sum_{j=n+1}^N v^j_k \frac{\dif}{\dif t} (E_j)_\gamma \right\rangle \ \dif t  \\
		&= 	\int_{t_0-\ve}^{t_0+\ve} \sum_{j=n+1}^N (\dot v^j_k)^2 \ \dif t+ o(1),
	\end{align*}
	e, sommando questo risultato su un numero finito di punti di \(\Sp^1\),
	\begin{equation}\label{eq: PS3}
			\int_{0}^{1} \left \langle \dot v_k, \frac{\dif}{\dif t} \pi^\perp_\gamma(v_k) \right\rangle \ \dif t  =  \int_0^1 |\pi^\perp_\gamma (\dot v_k)|^2 \ \dif t + o(1).
	\end{equation}
	
	Estendiamo la proiezione normale con la mappa
	\begin{align*}
		\pi^\perp : C^0(\Sp^1,M) &\to L(C^0(\Sp^1,\R^N),C^0(\Sp^1,\R^N))\\
		\sigma &\mapsto \pi^\perp_\sigma,
	\end{align*}
	definita da 
	\[
		\pi^\perp_\sigma (v(t)) = \pi^\perp_{\sigma(t)}(v(t)) \in N_{v(t)}M \subset \R^N
	\]
	per \(v \in C^0(\Sp^1,\R^N)\). Osserviamo che \(\pi^\perp\) è continua, perché l'espressione locale dipende dalle mappe
	\[
		\sigma \mapsto (E_i)_\sigma
	\] 
	associate a un frame \(E_1, \dots , E_N\), che sono chiaramente continue. Poiché \(\gamma_{h_k} \to \gamma\) in \(C^0\), anche \(\pi^\perp_{\gamma_{h_k}} \to \pi^\perp_\gamma\) in \(L(C^0(\Sp^1,\R^N),C^0(\Sp^1,\R^N))\). Dato che 
	\[
		\pi^\perp_\gamma(\dot \gamma) = 0 = \pi^\perp_{\gamma_{h_k}}(\dot \gamma_{h_k}),
	\]
	otteniamo
	\begin{align}\label{eq: PS4}
		\int_0^1 |\pi^\perp_\gamma (\dot v_k)|^2 \ \dif t &= \int_0^1 |\pi^\perp_\gamma (\dot \gamma_{h_k}- \dot \gamma)|^2 \ \dif t \nonumber \\
		&= \int_0^1 |(\pi^\perp_\gamma - \pi^\perp_{\gamma_{h_k}})(\dot \gamma_{h_k})|^2 \ \dif t \nonumber \\
		& \leq \|\pi^\perp_\gamma - \pi^\perp_{\gamma_{h_k}}\|^2_{L(C^0,C^0)} \int_0^1|\dot \gamma_{h_k}|^2 \ \dif t \nonumber \\
		& \leq c_2 \|\pi^\perp_\gamma - \pi^\perp_{\gamma_{h_k}}\|^2_{L(C^0,C^0)} \to 0.
	\end{align}
	
	Mettendo insieme (\ref{eq: PS1}), (\ref{eq: PS2}), (\ref{eq: PS3}) e (\ref{eq: PS4}) otteniamo che \(\dot v_k \to 0\) in \(L^2\), e quindi \(\gamma_{h_k} \to \gamma\) in \(H^1\).
\end{proof}

Ripetendo la stessa dimostrazione del Lemma~\ref{lemma: di deformazione} e del Teorema~\ref{teo: MP}, otteniamo il seguente. 
\begin{teo}\label{teo: MP con energia}
	Sia \(M\) una varietà chiusa e \(\mathcal{H}\) una famiglia non vuota di sottoinsiemi di \(H^1(\Sp^1,M)\) \(\phi\)-invariante e tale che
	\begin{equation}\label{eq: MP level}
		\lambda \coloneq \inf_{H \in \mathcal{H}} \sup_{\gamma \in H} E(\gamma) \in \R.
	\end{equation}
	Allora \(\lambda\) è un valore critico, cioè è l'energia di una geodetica chiusa (costante se \(\lambda = 0\)). 
\end{teo}




	
	\chapter{Il teorema di Lusternik-Fet}
	
In questo capitolo dimostriamo il seguente:
\begin{teo}[Lusternik-Fet, 1951]\label{teo: Lusternik-Fet}
	Ogni varietà Riemanniana chiusa ammette una geodetica chiusa non costante. 
\end{teo}
La dimostrazione è adattata da \cite[Theorem~2.4.20]{klingenberg1995riemannian}. Un'altra dimostrazione, più elementare e che non utilizza esplicitamente il teorema di passo montano, può essere trovata in \cite[Theorem~A.1.5]{klingenberg2012lectures}.

Grazie al Teorema~\ref{teo: punto critico energia se e solo se geodetica}, dimostrare l'esistenza di una geodetica chiusa non costante equivale a dimostrare l'esistenza di un punto critico non banale dell'energia in \(H^1(\Sp^1,M)\). La topologia di \(H^1(\Sp^1,M)\) è estremamente diversa nei seguenti due casi, che analizziamo nel dettaglio nelle prossime sezioni.
\begin{enumerate}[label=(\arabic*)]
	\item \(M\) non è semplicemente connessa, cioè il gruppo fondamentale di \(M\) non è banale: si riesce a \textit{minimizzare} l'energia nelle classi di omotopia libera di curve chiuse.
	\item  \(M\) è semplicemente connessa, cioè il gruppo fondamentale di \(M\) è banale: non abbiamo una classe non banale su cui minimizzare. Utilizzando dei risultati sui gruppi di omotopia, ci riconduciamo a lavorare su una sfera e, con un argomento di passo montano, individuiamo un punto critico non banale. 
\end{enumerate}

\section{Caso non semplicemente connesso}\label{sez: non semplicemente connesso}

Sia \((M,g)\) una varietà Riemanniana chiusa. Ricordiamo che abbiamo le seguenti inclusioni
\[
	C^\infty(\Sp^1,M) \hookrightarrow H^1(\Sp^1,M) \hookrightarrow C^0(\Sp^1,M).
\]
indotte da le analoghe in \(\R^N\).

Un cammino in \(C^0(\Sp^1,M)\) è una mappa
\begin{align*}
	H: I &\to C^0(\Sp^1,M) \\
	s &\mapsto H(s)
\end{align*}
che può essere identificata con l'omotopia continua tra le due curve chiuse \(H(0),H(1) \in C^0(\Sp^1,M)\) definita da
\begin{align*}
	\overline{H}: I \times \Sp^1 &\to M \\
	(s,t) &\mapsto H(s)(t).
\end{align*}
Infatti vale
\begin{lemma}\label{lemma: identificazione curve/omotopie}
	Siano \(X\) e \(Y\) due spazi topologici compatti e sia \((Z,d)\) uno spazio metrico. Dotiamo \(C^0(X \times Y, Z)\), \(C^0(Y,Z)\) e \(C^0(X,C^0(Y,Z))\) delle distanze della convergenza uniforme, ovvero
	\begin{align*}
		& d_{C^0(X \times Y, Z)}(f,g) \coloneq \sup_{(x,y) \in X \times Y} d (f(x,y),g(x,y)) & f,g \in C^0(X \times Y, Z), \\
		& d_{C^0(Y, Z)}(f,g) \coloneq \sup_{y \in Y} d(f(y),g(y)) & f,g \in C^0(Y, Z), \\
		& d_{C^0(X,C^0(Y,Z))}(F,G) \coloneq \sup_{x \in X} d_{C^0(Y, Z)} (F(x),G(x)) & F,G \in C^0(X,C^0(Y, Z))
	\end{align*}
	rispettivamente. Allora la mappa 
	\begin{align}\label{eq: omemom isometrico}
		C^0(X \times Y,Z) &\to C^0(X,C^0(Y,Z))\\
		h &\mapsto H \nonumber
	\end{align}
	definita da
	\[
		H(x)(y) \coloneq h(x,y)
	\]
	è ben definita ed è un omeomorfismo isometrico. 
	
	Inoltre, se \(X\), \(Y\) e \(Z\) sono varietà Riemanniane, allora per ogni \(h \in C^0(X \times Y, Z)\), 
	\[
		h \in C^\infty(X \times Y, Z) \quad \iff \quad \begin{cases}
			H \in C^\infty(X,C^0(Y,Z))\\
			H(x) \in C^\infty(Y,Z) \ \forall x \in X
		\end{cases}
	\]
	ricordando che \(C^0(Y,Z)\) è una varietà di Banach (cf. Sezione~\ref{sez: var Banach/Hilbert}).
\end{lemma}
\begin{proof}
	La mappa (\ref{eq: omemom isometrico}) è ben definita e un omeomorfismo per \cite[Proposition~A.13, Proposition~A.14, Proposition~A.16]{hatcher2000algebraic}. Per vedere che è un'isometria, consideriamo \(g,h \in C^0(X\times Y, Z)\) e le loro immagini \(G,H \in C^0(X,C^0(Y,Z))\):
	\begin{align*}
		d_{C^0(X,C^0(Y,Z))}(G,H) &= \sup_{x \in X} d_{C^0(Y, Z)} (G(x),H(x))\\
		&= \sup_{x \in X} \sup_{y \in Y} d(G(x)(y),H(x)(y)) \\
		&= \sup_{(x,y) \in Y} d (g(x,y),h(x,y)) \\
		&= d_{C^0(X \times Y, Z)}(g,h).
	\end{align*}
	
	Siano \(X\), \(Y\) e \(Z\) delle varietà Riemanniane, di dimensione \(n\), \(m\) e \(l\) rispettivamente. Osserviamo che in coordinate (senza rinominare le rappresentazioni locali), se sia \(h\) che \(H\) fossero differenziabili, allora
	\begin{align}\label{eq: de_xh}
		\dif H: \R^n \times \R^n &\to C^0(\R^m,\R^l) \times C^0(\R^m,\R^l) \nonumber \\
		(x,v) & \mapsto (H(x), \de_x h(x, \cdot )[v]),
	\end{align}
	dove \(\de_x h(x,y): \R^n \to \R^l\) è la mappa lineare data dalla matrice \(l \times n\)
	\[
		\left( \frac{\de h(x,y)}{\de x^1}, \dots , \frac{\de h(x,y)}{\de x^n} \right).
	\] 
	Analogamente, 
	\begin{equation}\label{eq: de_yh}
		\dif (H(x))(y)[w] = \de_y h (x,y)[w],
	\end{equation} 
	dove \(\de_y h(x,y): \R^m \to \R^l\) è dato dalla matrice \(l \times m\)
	\[
	\left( \frac{\de h(x,y)}{\de y^1}, \dots , \frac{\de h(x,y)}{\de y^m} \right).
	\]
	In particolare, 
	\begin{align}\label{eq: dh}
		\dif h(x,y)[v,w] &= \de_xh(x,y)[v]+ \de_yh(x,y)[w] \nonumber\\
		&= pr_2(\dif H(x)[v])(y) + \dif (H(x))(y)[w],
	\end{align}
	dove \(pr_2: C^0(\R^m,\R^l) \times C^0(\R^m,\R^l)\) è la proiezione sul secondo fattore.
	
	Supponiamo che \(H\) sia liscia e abbia valori in \(C^\infty(Y,Z)\). Allora \(h\) ha derivate parziali \(\de_xh(x,y)[v] = pr_2(\dif H(x)[v])(y)\) e \(\de_yh(x,y)[w] = \dif (H(x))(y)[w]\), e per il teorema del differenziale totale \(h\) è differenziabile, con differenziale definito dalla formula (\ref{eq: dh}). Siccome \(T(X \times Y) \equiv TX \times TY\), con lo stesso ragionamento otteniamo che anche \(\dif h : TX \times TY \to TZ\) è differenziabile, per cui \(h \in C^2(X \times Y, Z)\). Per induzione, otteniamo \(h \in C^\infty(X \times Y, Z)\). 
	
	Supponiamo che \(h\) sia liscia. Poiché \(\{x\} \times Y \equiv Y\) è una sottovarietà di \(X \times Y\), la restrizione \(h|_{\{x\} \times Y} = H(x) \in C^\infty(Y,Z)\), e quindi vale la (\ref{eq: de_yh}). Inoltre \(H\) risulta differenziabile definendo il differenziale con (\ref{eq: de_xh}). In maniera induttiva analoga a quanto fatto sopra per \(h\), otteniamo che \(H \in C^\infty(X,C^0(Y,Z))\).
\end{proof}


Dunque le componenti connesse di \(C^0(\Sp^1,M)\) sono le classi di omotopia di curve chiuse \(\Sp^1 \to M\). 

Alla fine del XIX secolo, H. Poincaré e J. Hadamard cercavano di trovare un rappresentante geodetico per ogni classe di omotopia di curve chiuse. Nel 1898 J. Hadamard ha dimostrato l'asserto nel caso delle superfici compatte, sostanzialmente minimizzando la lunghezza \cite{hadamard1898surfaces}. Dimostriamo lo stesso risultato, ma con un linguaggio più moderno: utilizzeremo il metodo diretto del Calcolo delle Variazioni minimizzando il funzionale energia, quindi in primis occorre restringere il problema allo spazio \(H^1(\Sp^1,M)\).

Siccome \( H^1(\Sp^1,M) \hookrightarrow C^0(\Sp^1,M)\) è continua, ogni componente connessa di \(H^1(\Sp^1,M)\) è contenuta in un'unica componente connessa di \(C^0(\Sp^1,M)\). Verifichiamo che le componenti connesse di \(H^1(\Sp^1,M)\) sono esattamente quelle di \(C^0(\Sp^1,M)\) intersecate con \(H^1(\Sp^1,M)\). 

\begin{lemma}\label{lemma: omotopie lisce}
	Siano \(\gamma_0, \gamma_1 \in C^\infty(\Sp^1,M)\) due curve chiuse omotope in \(C^0(\Sp^1,M)\). Allora l'omotopia può essere scelta di classe \(C^\infty\), e in particolare \(\gamma_0\) e \(\gamma_1\) sono connesse da un cammino \(\widetilde{H}:I \to C^\infty(\Sp^1,M) \subset C^0(\Sp^1,M)\).
\end{lemma}
\begin{proof}
	Sia \(H: I \times \Sp^1 \to M\) un'omotopia tra \(\gamma_0\) e \(\gamma_1\). Possiamo estendere \(H\) a una mappa \(F: \R \times \Sp^1 \to M\) ponendo
	\[
		F(s,t) \coloneq \begin{cases}
			H(s,t) & \text{ se } 0\leq s \leq 1 \\
			\gamma_0(t) & \text{ se } s <0 \\
			\gamma_1(t) &\text{ se }s>1.
		\end{cases}
	\]
	Osserviamo che \(F\) è liscia sul chiuso \(A = (\R \setminus (0,1)) \times \Sp^1\). Per il Teorema~\ref{teo: approssimazione whitney}, \(F\) è omotopa relativamente ad \(A\) a una mappa liscia \(\widetilde{F}\). Allora \(\widetilde{H} \coloneq \widetilde{F}|_{I \times \Sp^1}\) è l'omotopia liscia voluta. 
\end{proof}

\begin{lemma}\label{lemma: le classi di omotopia sono aperte}
	Per ogni \(\gamma\in H^1(\Sp^1,M)\), l'aperto (cf. Osservazione~\ref{oss: H1 varietà di Hilbert})
	\[
		\mathcal{U}(\gamma) \coloneq \exp_\gamma(\mathcal{B}(\gamma)) = \left\{\exp_\gamma(X) \in H^1(\Sp^1,M) \ \middle| \ 
		\begin{aligned}
			&X \in T_{\gamma}H^1(\Sp^1,M), \\
			&|X(t)|_{\gamma(t)} < \ve_\gamma \; \forall t \in \Sp^1
		\end{aligned}
		\right\},
	\]
	è contenuto nella classe di omotopia di \(\gamma\) in \(H^1(\Sp^1,M)\). In particolare, se \([\gamma]\) è la classe di omotopia di \(\gamma\) in \(C^0(\Sp^1,M)\), allora \([\gamma] \cap H^1(\Sp^1,M)\) è aperto in \(H^1(\Sp^1,M)\). 
\end{lemma}
\begin{proof}
	Sia \(\widetilde{\gamma}= \exp_\gamma(W) \in \mathcal{U}(\gamma)\). Un'omotopia in \(H^1(\Sp^1,M)\) tra \(\gamma\) e \(\widetilde{\gamma}\) è 
	\begin{align}\label{eq: cammino in H^1}
		I \times \Sp^1 &\to M \nonumber \\
		(s,t) &\mapsto \exp_{\gamma(t)}(sW_t). 
	\end{align}
\end{proof}

Siano \(\gamma_0,\gamma_1 \in H^1(\Sp^1,M)\) due curve chiuse omotope in \(C^0(\Sp^1,M)\). Grazie alla Proposizione~\ref{prop: approssimazione con curve lisce}, esistono due curve lisce \(\widetilde{\gamma}_0 \in \mathcal{U}(\gamma_0)\) e \(\widetilde{\gamma}_1 \in \mathcal{U}(\gamma_1)\) che per il Lemma~\ref{lemma: le classi di omotopia sono aperte} sono omotope in \(H^1(\Sp^1,M)\) a \(\gamma_0\) e \(\gamma_1\) rispettivamente. Per il Lemma~\ref{lemma: omotopie lisce}, esiste un'omotopia liscia \(\widetilde{H}\) tra \(\widetilde{\gamma}_0\) e \(\widetilde{\gamma}_1\). In particolare \(\widetilde{H}\) è un cammino in \(C^\infty(\Sp^1,M) \subset H^1(\Sp^1,M)\) che connette \(\widetilde{\gamma}_0\) a \(\widetilde{\gamma}_1\). Siano \(H_0\) e \(H_1\) i cammini definiti da (\ref{eq: cammino in H^1}) relativi a \(\gamma_0\) e \(\widetilde{\gamma}_0\) e a \(\gamma_1\) e \(\widetilde{\gamma}_1\) rispettivamente. Allora il cammino \(H:I \to H^1(\Sp^1,M)\) definito da 
\[
	H(s) \coloneq \begin{cases}
		H_0(3s) & \text{ se } 0 \leq s \leq 1/3 \\
		\widetilde{H}(3s-1) &\text{ se } 1/3 \leq s \leq 2/3 \\
		H_1(3-3s) & \text{ se } 2/3 \leq s \leq 1
	\end{cases}
\]
connette \(\gamma_0\) a \(\gamma_1\). Abbiamo dimostrato la seguente:
\begin{prop}
	Le componenti connesse di \(H^1(\Sp^1,M)\) sono ottenute intersecando quelle di \(C^0(\Sp^1,M)\) con \(H^1(\Sp^1,M)\), cioè, se \(\gamma \in \Gamma \subset H^1(\Sp^1,M)\), allora 
	\[
		\Gamma = [\gamma] \cap H^1(\Sp^1,M),
	\]
	dove \([\gamma]\) è la classe di omotopia di \(\gamma\) in \(C^0(\Sp^1,M)\).
\end{prop}

Se \(M\) non è semplicemente connessa, sicuramente ci sono almeno due componenti connesse di \(C^0(\Sp^1,M)\). Infatti, denotando con \(\pi_0(C^0(\Sp^1,M))\) la famiglia delle classi di omotopia di curve chiuse (o delle componenti connesse di \(C^0(\Sp^1,M)\)), l'azione per coniugio di \(\pi_1(M,p)\) su se stesso e la mappa naturale \(\pi_1(M,p) \to \pi_0(C^0(\Sp^1,M))\) inducono una biezione tra l'insieme delle classi di coniugio di \(\pi_1(M,p)\) e \(\pi_0(C^0(\Sp^1,M))\) (cf. \cite[Proposition~4A.2]{hatcher2000algebraic}). Siccome solo il gruppo banale ha un'unica classe di coniugio, se \(M\) non è semplicemente connessa c'è almeno una componente connessa di \(C^0(\Sp^1,M)\) non banale, cioè che non contiene curve costanti.

\begin{teo}[Hadamard, 1898]\label{teo: minimizzazione energia}
	Sia \((M,g)\) chiusa e non semplicemente connessa. Sia \(\Gamma \subset H^1(\Sp^1,M)\) una componente connessa non banale. Allora esiste una geodetica \(\gamma \in \Gamma\), che corrisponde a un punto di minimo dell'energia in \(\Gamma\).
\end{teo}
\begin{proof}
	Assumiamo che \(M \subset \R^N\), con la metrica indotta da quella piatta di \(\R^N\). Abbiamo le seguenti inclusioni:
	\begin{align*}
		&C^0(\Sp^1,M) \subset C^0(\Sp^1, \R^N) \quad \text{ chiuso},  \\
		&H^1(\Sp^1,M) \subset H^1(\Sp^1, \R^N) \quad \text{ chiuso}.
	\end{align*}
	Inoltre l'energia è estesa ad \(H^1(\Sp^1,\R^N)\) da
	\[
		\widetilde{E}(u) \coloneq \frac{1}{2} \|\dot u\|^2_{L^2} \qquad u \in H^1(\Sp^1, \R^N)
	\]
	che è debolmente semicontinua dal basso. 
	
	Usiamo la strategia del metodo diretto del Calcolo delle Variazioni per minimizzare l'energia. Poniamo
	\[
		\lambda \coloneq \inf_\Gamma E
	\]
	e consideriamo una successione minimizzante \((\gamma_h)_h \subset \Gamma\), cioè tale che
	\[
		E(\gamma_h) \to \lambda.
	\]
	Siccome \(M \subset \R^N\) è compatta, è anche limitata. Allora esiste \(c>0\) tale che 
	\[
		\|\gamma_h(t) \|_{\R^N}^2 \leq c \qquad \forall t \in \Sp^1, \ h \in \N
	\]
	e quindi
	\[
		\|\gamma_h\|_{L^2} \leq \|\gamma_h\|_{C^0} \leq c \qquad \forall h \in \N.
	\]
	Inoltre anche la successione
	\[
		\|\dot \gamma_h \|_{L^2} = \sqrt{2E(\gamma_h)}
	\]
	è limitata perché convergente. Per il Teorema~\ref{teo: immersione Sobolev per S^1}, esiste \(\gamma \in H^1(\Sp^1,\R^N)\) tale che, a meno di estrarre una sottosuccessione,
	\begin{itemize}
		\item \(\gamma_h \rightharpoonup \gamma\) in \(H^1(\Sp^1,\R^N)\),
		\item \(\gamma_h \to \gamma \) in \(C^0(\Sp^1,\R^N)\).
	\end{itemize}
	Per la semicontinuità di \(\widetilde{E}\), 
	\[
		\widetilde{E}(\gamma) = \lambda.
	\]
	Sia \(\overline{\Gamma}\) la componente connessa di \(C^0(\Sp^1,M)\) tale che \(\Gamma= \overline{\Gamma} \cap H^1(\Sp^1,M)\). Osserviamo che la seguente è una catena di chiusi:
	\[
		\overline{\Gamma} \subset C^0(\Sp^1,M) \subset C^0(\Sp^1,\R^N)
	\]
	e quindi in realtà \(\gamma \in \Gamma\). Dunque \(\gamma\) è un punto di minimo per l'energia in \(\Gamma\). Per il Lemma~\ref{lemma: le classi di omotopia sono aperte}, \(\Gamma\) è aperta in \(H^1(\Sp^1,M)\), e quindi \(\gamma\) è un punto di minimo locale per l'energia, e in particolare è un punto critico. Per il Teorema~\ref{teo: punto critico energia se e solo se geodetica}, \(\gamma\) è una geodetica. 
\end{proof}


\section{Caso semplicemente connesso}\label{sez: semplicemente connesso}

Se \(M\) è semplicemente connessa, il Teorema~\ref{teo: minimizzazione energia} non ha significato: non c'è una classe non banale in cui minimizzare l'energia. Il seguente esempio mostra che non è detto che ci siano dei punti di minimo locali non banali dell'energia.
\begin{es}
	Consideriamo la sfera \(M=\Sp^2 \subset \R^3\), definita dall'equazione \(x^2+y^2+z^2=1\). Le geodetiche della sfera sono i cerchi massimi, ovvero sono parametrizzazioni a velocità costante dell'interesezione di \(\Sp^2\) e un piano passante per l'origine (cf. Esempio~\ref{es: geodetiche non minimi}).
	Sia \(\gamma_0 \in H^1(\Sp^1,\Sp^2)\) un punto critico dell'energia, vale a dire una geodetica. A meno di ruotare la sfera, possiamo supporre che 
	\[
		\gamma_0(t) = (\cos(2 k\pi t), \sin(2k \pi t),0) \qquad t \in \R.
	\]
	con \(k \in \Z\). Per una funzione \(u \in C^\infty(\R)\) 1-periodica, consideriamo
	\[
		\gamma_s(t) \coloneq (\cos(2k \pi t)\cos(s u(t)), \sin(2k\pi t)\cos(su(t)),\sin(su(t)))
	\]
	per ogni \(s \in (-\pi,\pi)\), \(t \in \R.\)
	La mappa \(s \mapsto E(\gamma_s)\) è una funzione che ha \(s=0\) come punto critico. Osserviamo che
	\[
		E(\gamma_s) = \frac{1}{2} \int_0^1s^2 \dot u^2(t) + 4k^2\pi^2\cos^2(su(t))\ \dif t 
	\]
	e quindi
	\[
		\left.\frac{\dif^2}{\dif s^2}\right|_{s=0} E(\gamma_s) = \int_0^1 \dot u^2(t) - 4k^2\pi^2\cos(2su(t))u^2(t) \ \dif t.
	\]
	Scegliamo \(u\equiv 1\), e quindi 
	\[
			\left.\frac{\dif^2}{\dif s^2}\right|_{s=0} E(\gamma_s) = -4k^2\pi^2 <0.
	\]
	Dunque \(s=0\) non è un punto di minimo della funzione \(s \mapsto E(\gamma_s)\), cioè \(\gamma_0\) non è un punto di minimo locale dell'energia.
	
	%INSTABILITA' GEODETICA SFERA
\begin{figure}[h]
	\centering
	\begin{tikzpicture}[decoration={markings, mark=at position 0.5 with {\arrow{>}}}]
		%\draw [help lines] (-3,-4) grid (10,4);
		
		% DOMINIO
		%linee
		\draw [thick] (0,0) circle (2);
		\draw[thick] (-2,0) arc (180:360: 2 and 0.55);
		\draw[decorate] (-2,0) arc (180:360: 2 and 0.55);
		\draw [thick, dashed] (2,0) arc (0:180: 2 and 0.55);
		\draw[thin,->] (0,0) -- (0,3);
		\draw[thin,->] (0,0) -- (3,0);
		\draw[thin,->] (0,0) -- (-2.3,-1.5);
		\draw [thick](-1.908,0.6) arc(180:360:1.908 and 0.3);
		\draw [decorate](-1.908,0.6) arc(180:360:1.908 and 0.3);
		\draw [thick,dashed](1.908,0.6) arc(0:180:1.908 and 0.3);
		
		%nomi
		\node at(2,2){\(\Sp^{2}\)};
		\node at (-2.2,-1.8) {\(x\)};
		\node at (3,-0.3){\(y\)};
		\node at (-0.4,2.8){\(z\)};
		\node at  (0,-1) {\(\gamma_0\)};
		\node at  (-0.5,0) {\(\gamma_s\)};
		
	\end{tikzpicture}
	
	\caption{La funzione \(s \mapsto E(\gamma_s)\) non ha un minimo locale per \(s=0\).}
	
	\label{fig: instabilità geodetica}
	
\end{figure}
\end{es}

L'esempio precedente mostra che, in una generica varietà chiusa semplicemente connessa, è inutile provare a minimizzare l'energia. Nel 1917 G.D. Birkhoff, usando un procedimento di min-max, ha dimostrato l'esistenza di una geodetica chiusa su una qualunque superficie chiusa semplicemente connessa, vale a dire su una sfera \(\Sp^2\) dotata di una metrica Riemanniana qualunque \cite{birkhoff1917dynamical}.
Generalizziamo l'idea di Birkhoff utilizzando il Teorema~\ref{teo: MP con energia}. Un elemento di un gruppo di omotopia di \(M\) genera una famiglia \(\phi\)-invariante \(\mathcal{H}\), su cui possiamo applicare il Teorema~\ref{teo: MP con energia}; pertanto il punto di partenza è trovare un gruppo di omotopia non banale:
\begin{prop*A}
	Se \(M\) è una varietà (topologica) \(n\)-dimensionale, chiusa e semplicemente connessa, allora esiste \(0<k < n\) tale che \(\pi_{k+1}(M) \not = \{0\}\). In particolare, se \(M\) è una varietà differenziabile, esiste una mappa \(f:\Sp^{k+1} \to M\) di classe \(C^\infty\) e omotopicamente non banale, cioè non omotopa relativamente al punto \(e_{k+1} \in \Sp^{k+1}\) a nessuna mappa costante. 
\end{prop*A}
La dimostrazione è rimandata all'Appendice~\ref{appendice}.
La seguente costruzione può essere vista come un legame tra la topologia di \(M\) e la topologia di \(H^1(\Sp^1,M)\). Denotiamo con $\mathcal{F}$ l'insieme delle mappe \(F \in C^\infty(\D^k, C^0(\Sp^1,M))\) tali che
\begin{itemize}
	\item \(F(x) \in C^\infty(\Sp^1,M)\) per ogni \(x \in \D^k\);
	\item \(F(x)\) è costante per ogni \(x \in \de \D^k\).
\end{itemize}

\begin{lemma}\label{lemma: corrispondenza f e F}
	C'è una corrispondenza biunivoca
	\begin{align*}
		C^\infty(\Sp^{k+1},M) &\longleftrightarrow \mathcal{F}\\
		f  & \mapsto f_*\\
		F_\# & \mapsfrom  F .
	\end{align*}
\end{lemma}

\begin{proof}
	La corrispondenza segue dalla seguente costruzione. Consideriamo la sfera \(\Sp^{k+1}\) immersa come la sfera unitaria di \(\R^{k+2}\) e identifichiamo
	\[
	\D^k \equiv \{(x_0,0,z) \in \Sp^{k+1} \ | \ x_0 \geq 0\},
	\]
	come in Figura \ref{fig: costruzione di F}. Per ogni \(x=(x_0,0,z) \in \D^k\), sia \(\gamma_x:[0,1]\to \Sp^{k+1}\) definito da
	\[
	\gamma_x(t) \coloneq (x_0 \cos(2\pi t),x_0 \sin(2\pi t),z), \qquad t \in [0,1].
	\]
	Siccome la mappa 
	\begin{align*}
		\D^k \times \Sp^1 &\to \Sp^{k+1} \\
		(x,t) &\mapsto \gamma_x(t)
	\end{align*}
	è una mappa \(C^\infty\), per il Lemma~\ref{lemma: identificazione curve/omotopie} anche la mappa
	\begin{align*}
		\gamma: \D^k \to C^\infty([0,1],\Sp^{k+1}) \subset C^0([0,1],\Sp^{k+1})
	\end{align*}
	è \(C^\infty\). Osserviamo che se \(x \in \de \D^k\), cioè \(x_0=0\), allora \(\gamma_x \equiv x\) è il cammino costante.
	
	Per \(f \in C^\infty(\Sp^{k+1},M) \), definiamo
	\[
		f_*(x) \coloneq f \circ \gamma_x,
	\]
	che è \(C^\infty\) perché composizione di mappe \(C^\infty\). In particolare, se \(x \in \de \D^k\), allora \(f_*(x) \equiv f(x)\) è un cammino costante. 
	
	%COSTRUZIONE DI F
\begin{figure}[h]
	\centering
	\begin{tikzpicture}[decoration={markings, mark=at position 0.5 with {\arrow[]{>}}}]
		%\draw [help lines] (-3,-4) grid (10,4);
		
		% DOMINIO
		%linee
		\draw [thick] (0,0) circle (2);
		\draw (-2,0) arc (180:360: 2 and 0.55);
		\draw [dashed] (2,0) arc (0:180: 2 and 0.55);
		\draw [very thick] (0,2) arc (90:270: 0.55 and 2);
		\draw[thin,->] (0,0) -- (0,3);
		\draw[thin,->] (0,0) -- (3,0);
		\draw[thin,->] (0,0) -- (-2,-2);
		\fill (0,2) circle (1.5pt)
		(0,-2) circle (1.5pt);
		\draw [](-1.732,1) arc(180:360:1.732 and 0.3);
		\draw [decorate](0,0.7) arc(270:360:1.732 and 0.3);
		\draw [dashed](1.732,1) arc (0:180:1.732 and 0.3);
		\fill[] (-0.52,0.71) circle (1.5pt);
		
		%nomi
		\node at(2,2.5){\(\Sp^{k+1}\)};
		\node at (-2.2,-1.8) {\(x_0\)};
		\node at (3,-0.3){\(x_1\)};
		\node at (-0.4,2.8){\(z\)};
		\node at (0,-1) {\(\D^k\)};
		\node[] at (-0.7,0.9) {\(x\)};
		\node[] at  (1,1) {\(\gamma_x\)};
		
		
		
		
		%freccina
		\draw[thick,->](3,1) to[out=20, in=160](5,1);
		\node at (4,1.5){\(f\)};
		
		
		%CODOMINIO
		%linee
		\draw[thick] (6,0) to [out=250, in=90] (5.1,-2) to [out=270, in=180](7,-2.5)to [out=0,in=290](8.5,-1) to[out=110,in=260] (8.5,0) to [out=80,in=250] (9,1) to[out=70,in=0](7.5,2.2) to[out=180,in=100] (6,1) to[out=280,in=70](6,0);
		\draw(6,0) to[out=290,in=170](7,-0.4) to [out=350,in=250](8.5,0);
		\draw[dashed](8.5,0) to[out=70,in=10] (7.5,0.2) to[out=190,in=60] (6,0);
		\fill (7.5,2) circle (1.5pt)
		(6.5,-2.5) circle (1.5pt);
		\draw [very thick] (6.5,-2.5)to[out=160,in=265](6.2,-2) to[out=85,in=270](6.9,0) to[out=90,in=270](6.8,1) to[out=90,in=215](7.5,2);
		\draw [](6,1) to[out=290,in=170](6.82,0.68) to[out=350, in=210](8,0.8) to [out=30,in=250](9,1);
		\draw[decorate](8,0.8) to [out=30,in=250](9,1);
		\draw[dashed](9,1) to[out=70,in=10] (7.8,1.5) to[out=190,in=60] (6,1);
		\fill[] (6.82,0.68) circle (1.5pt);
		
		%nomi
		\node at (6,2.5) {\(f(\Sp^{k+1})\subset M\)};
		\node at (7,-1.5){\(f(\D^k)\)};
		\node [] at(7.3,1){\(f(x)\)};
		\node [] at (8.5,0.4) {\(f \circ \gamma_x = f_*(x) \)};
		
		
	\end{tikzpicture}
	
	\caption{Costruzione dell'identificazione \(C^\infty(\Sp^{k+1},M) \to \mathcal{F}\).}
	
	\label{fig: costruzione di F}
	
\end{figure}
	
	Viceversa, sia \(F \in \mathcal{F}\). Allora la mappa 
	\[
	h(x,t) = F(x)(t), \qquad x\in \D^k, t \in [0,1]
	\]
	è una mappa liscia dal cilindro \(\D^k \times [0,1] \to M\). Dato che \(F(x)\) è un cammino costante se \(x \in \de \D^k\), possiamo definire \(f=F_\#\) come
	\[
	f(x_0 \cos (2\pi t), x_0 \sin (2 \pi t), z) = h(x,t) = F(x)(t), \qquad t \in [0,1], x \in \D^k,
	\]
	avendo identificato \(\D^k \subset \Sp^{k+1}\) come prima. La definizione è ben posta e \(f = F_\# \in C^\infty(\Sp^{k+1},M)\). 
	
	È evidente dalla costruzione che \(*\) e \(\#\) sono l'una l'inversa dell'altra.
\end{proof}

\begin{lemma}\label{lemma: corrispondenza h e H}
	La corrispondenza del Lemma~\ref{lemma: corrispondenza f e F} si solleva ad una corrispondenza biunivoca tra le omotopie lisce, ovvero la stessa costruzione dà una corrispondenza biunivoca tra l'insieme delle omotopie lisce tra mappe in \(C^\infty(\Sp^{k+1},M)\), cioè
	\[
		C^\infty(\Sp^{k+1} \times [0,1],M),
	\]
	e l'insieme delle omotopie lisce tra mappe in \(\mathcal{F}\), cioè
	\[
		\{H \in C^\infty(\D^k \times [0,1], C^0(\Sp^1,M)) \ | \ H(\cdot,t) \in \mathcal{F} \quad \forall t \in [0,1] \}.
	\]
	
	Inoltre questa costruzione si restringe anche alle omotopie lisce relative al punto \(e_{k+1} = (0,\dots,0,1) \in \D^k \subset \Sp^{k+1}\) (identificato come prima).
\end{lemma}
\begin{proof}
	Sia \(h \in C^\infty(\Sp^{k+1} \times [0,1], M)\) un'omotopia liscia tra \(f_0=h(\cdot,0), f_1 = h(\cdot, 1) \in C^\infty(\Sp^{k+1},M)\). Per il Lemma~\ref{lemma: identificazione curve/omotopie}, la mappa 
	\begin{align*}
		[0,1] &\to C^\infty(\Sp^{k+1},M) \subset C^0(\Sp^{k+1},M)\\
		t &\mapsto h_t=h(\cdot, t)
	\end{align*}
	è \(C^\infty\). Componendo con la corrispondenza del Lemma~\ref{lemma: corrispondenza f e F}, definiamo la mappa \(C^\infty\)
	\[
		h_*: \D^k \times [0,1] \to C^\infty(\Sp^1,M) \subset C^0(\Sp^1,M),
	\]
	definita quindi da
	\[
		h_*(x,t) \coloneq (h_t)_*(x). 
	\]
	Allora \(h_*\) è un'omotopia liscia tra \((f_0)_*\) e \((f_1)_*\). Inoltre, se \(h\) è un'omotopia liscia relativa a \(e_{k+1} \in \Sp^{k+1}\), cioè se
	\[
		h(e_{k+1},t) = p \in M \qquad \forall t \in [0,1]
	\]
	allora anche \(h_*(e_{k+1},t) \equiv p\) per ogni \(t \in [0,1]\).

	L'altra costruzione è analoga, usando \(\#\). Inoltre, come sopra, le costruzioni sono l'una l'inversa dell'altra.
\end{proof}

Diamo adesso la dimostrazione del Teorema di Lusternik-Fet.

\begin{proof}[Dimostrazione del Teorema~\ref{teo: Lusternik-Fet}(Lusternik-Fet)]
	Se \(M\) non è semplicemente connessa, grazie al Teorema \ref{teo: minimizzazione energia} riusciamo a trovare una geodetica chiusa. Dunque supponiamo che \(M\) sia semplicemente connessa. 
	
	Per la Proposizione~\ref{prop: esiste f omotopicamente non banale}, esiste  \(f:\Sp^{k+1} \to M\) di classe \(C^\infty\) e omotopicamente non banale, per un certo \(0<k<n\). Sia \(F=f_*\) la mappa indotta da \(f\), come nel Lemma~\ref{lemma: corrispondenza f e F}. Sia \([f] \in \pi_{k+1}(M,p)\) la classe di \(f\), con \(p = f(e_{k+1})\), e denotiamo
	\[
	\mathcal{F}(F) \coloneq \left\{ G \in \mathcal{F} \ \middle| \ G_\# \in [f] \right\},
	\]
	che, con la corrispondenza del Lemma~\ref{lemma: corrispondenza h e H}, è anche l'insieme delle mappe \(G \in \mathcal{F}\) omotope a \(F\) relativamente a \(e_{k+1}\), con omotopia liscia.
	Consideriamo
	\[
	\mathcal{H} \coloneq \{G(\D^k) \ | \ G \in \mathcal{F}(F)\}.
	\]
	Verifichiamo che \(\mathcal{H}\) sia \(\phi\)-invariante, con \(\phi\) il flusso definito dall'energia come nella Sezione~\ref{sez: MP con energia}. Sia \(T\geq 0\) e \(G(\D^k) \in \mathcal{H}\), con \(G \in \mathcal{F}(F)\); proviamo che \(\phi_T \circ G \in \mathcal{F}(F)\). Chiaramente, componendo con l'inclusione \(H^1(\Sp^1,M) \hookrightarrow C^0(\Sp^1,M)\), abbiamo \(\phi_T \circ G \in C^\infty(\D^k,C^0(\Sp^1,M))\). Per ogni \(x \in \D^k\), denotando \(\overline{G}: \D^k \times \Sp^1 \to M\) la mappa corrispondente a \(G\) attraverso l'isometria del Lemma~\ref{lemma: identificazione curve/omotopie},
	\[
		\frac{\dif \phi_T(G(x)(t))}{\dif t} = \frac{\dif \phi_T(\overline{G}(x,t)}{\dif t} = \dif \phi_T(G(x)(t)) \circ \frac{\de \overline{G}(x,t)}{\de t},
	\]
	e quindi \(\phi_T(G(x)) \in C^\infty(\Sp^1,M)\). Inoltre, chiaramente \(\phi_T \circ G\) è omotopa relativamente a \(e_{k+1}\) a \(G\), con omotopia liscia ottenuta riparametrizzando \(\phi|_{C^\infty(\Sp^1,M) \times [0,T]}\). Quindi \(\phi_T \circ G \in \mathcal{F}(F)\). 
	
	Osserviamo che
	\[
		\lambda \coloneq \inf_{H \in \mathcal{H}} \sup_{\gamma \in H} E(\gamma)
	\]
	è finito, in quanto \(\mathcal{H} \neq \emptyset\) e ogni \(H = G(\D^k) \in \mathcal{H}\) è compatto.
	Per il Teorema~\ref{teo: MP con energia}, \(\lambda \geq 0\) è un valore critico, cioè esiste una geodetica \(\gamma \in H^1(\Sp^1,M)\) con energia \(E(\gamma)=\lambda\). 
	
	Per concludere, dobbiamo mostrare che \(\lambda >0\), perché in tal caso \(\gamma\) sarebbe una curva non costante. Supponiamo per assurdo che \(\lambda=0\). Sia \(\ve>0\) come nel Corollario~\ref{cor: epsilon per compatta}. Sia \(G=g_* \in \mathcal{F}(F)\) tale che \(E(G(x))<\ve^2/2\) per ogni \(x \in \D^k\). Per la (\ref{eq: L^2<2E}), 
	\[
		L(G(x)) < \ve \qquad \forall x \in \D^k
	\]
	e quindi la mappa esponenziale 
	\[
		\exp_{g(x)}: B_\ve(0_{g(x)}) \to M
	\] 
	è un diffeomorfismo con la sua immagine \(B_\ve(g(x))\). Definiamo
	\[
		W(x)(t) = W(x,t) \coloneq \exp_{g(x)}^{-1}(G(x)(t)).
	\]
	Osserviamo che \(W: \D^k \to TM\) è di classe \(C^\infty\) perché è una composizione di mappe \(C^\infty\). Definiamo \(H: \D^k \times [0,1] \to H^1(\Sp^1,M) \) come
	\[
		H(x,s)(t) \coloneq \exp(sW(x,t)).
	\]
	Chiaramente \(H\) è di classe \(C^\infty\). Inoltre
	\begin{itemize}
		\item per ogni \(x \in \D^k, t \in \Sp^1\),
		\[
			H(x,0)(t) =\exp(0_{g(x)}) = g(x),
		\]
		cioè \(H_0=H(\cdot,0)\) manda punti di \(\D^k\) in mappe costanti e \(H_0 \in \mathcal{F}\);
		\item per ogni \(x \in \D^k, t \in \Sp^1\),
		\[
			H(x,1)(t) = \exp(W(x,t)) = G(x)(t);
		\]
		cioè \(H_1=G\);
		\item per ogni \(x \in \de \D^k, s \in [0,1], t \in \Sp^1\), siccome \(W(x) \equiv 0_{g(x)}\)
		\[
			H(x,s)(t) = \exp(0_{g(x)}) = g(x) 
		\]
		cioè \(H\) è un'omotopia liscia relativa a \(\de \D^k\) (e quindi anche relativa a \(e_{k+1}\)) tra \(H_0\) e \(H_1=G\).
	\end{itemize}
	
		%RETRAZIONE SUL DISCO
	
	\begin{figure}[ht]
	\begin{tikzpicture}[decoration={markings, mark=at position 0.5 with {\arrow[]{>}}}]
		%\draw [help lines] (-3,-3) grid (10,3);
		
		%CILINDRO
		\draw[thick](-2,-2) -- (-2,2);
		\draw[thick](2,-2) -- (2,2);
		\draw[very thick](0,-2)--(0,2);
		\draw[thick] (2,2)arc(0:360:2 and 0.5);
		\draw[thick] (-2,-2)arc(180:360:2 and 0.5);
		\draw[thick, dashed] (2,-2)arc(0:180:2 and 0.5);
		\fill[lightgray, opacity=0.4] (2,0)arc(0:360:2 and 0.5);
		\draw (-2,0)arc(180:360:2 and 0.5);
		\draw[dashed] (2,0)arc(0:180:2 and 0.5);
		\fill (0,-2) circle(1.5pt);
		\fill (0,2) circle(1.5pt);
		\fill (0,0) circle(1.5pt);
		\draw (-1.3,0.8) to[out=280, in =160](-0.8,0.2);
		\draw[<->] (0.1,2) to (1.9,2);
		\draw[thick] (0,0)to[out=330,in=180](1,-0.2)to[out=0,in=300](1.6,0)to[out=120,in=0](0.8,0.3)to[out=180,in=90](-0.3,0.2)to[out=270,in=120](0,0);
		\draw[thick, decorate](1,-0.2)to[out=0,in=300](1.6,0);
		\draw[] (0,0)to[out=330,in=180](0.5,-0.1)to[out=0,in=300](0.8,0)to[out=120,in=0](0.4,0.15)to[out=180,in=90](-0.15,0.15)to[out=270,in=120](0,0);
		\draw[decorate](0.5,-0.1)to[out=0,in=300](0.8,0);
		
		%nomi
		\node at (-2.5,2.5){\(TM\)};
		\node at (-0.4,-0.2){\(0_{g(x)}\)};
		\node at (-1.2,1){\(B_\ve(0_{g(x)})\)};
		\node at (1,1.8){\(\ve\)};
		\node at (1.5,-0.7){\(W(x)\)};
		%\node at (1,0.7){\(sW(x)\)};
		
		
		%FRECCINA
		\draw[thick,->](2.5,0) to (4,0);
		\node at (3.25,0.3){\(\exp\)};
		
		
		%IMMAGINE
		\draw[very thick] (6.5,-2)to[out=110,in = 270] (6.4,-1) to[out=90, in =260] (6.5,0)to[in=270,out=80](6.7,1) to[out=90,in =290](6.5,2);
		\draw[thick] (6.5,0)to[out=330,in=180](7.5,-0.3)to[out=0,in=300](8.1,0)to[out=120,in=0](7.45,0.3)to[out=180,in=90](6.2,0.2)to[out=270,in=120](6.5,0);
		\draw[thick, decorate](7.5,-0.3)to[out=0,in=300](8.1,0);
		\draw[] (6.5,0)to[out=330,in=180](6.75,-0.1)to[out=0,in=300](7.3,0)to[out=120,in=0](6.7,0.2)to[out=180,in=90](6.4,0.2)to[out=270,in=120](6.5,0);
		\draw[decorate](6.75,-0.1)to[out=0,in=300](7.3,0);
		\draw[dashed](4.5,2)to[out=280,in=190](6.5,1.6)to[out=10,in=270](8.5,2)to[out=90,in=10](6.5,2.4)to[out=190,in=60](4.5,2);
		\draw[dashed](4.5,-2)to[out=280,in=190](6.5,-2.3)to[out=10,in=270](8.5,-2)to[out=90,in=10](6.5,-1.4)to[out=190,in=60](4.5,-2);
		\draw[dashed](4.5,2)to[out=270,in=90](4.9,0)to[out=270,in=130](4.6,-2.25);
		\draw[dashed](8.5,2)to[out=260,in=90](8.2,-0.5)to[out=270,in=95](8.5,-2);
		\fill (6.5,-2) circle(1.5pt);
		\fill (6.5,0) circle(1.5pt);
		\fill (6.5,2) circle(1.5pt);
		\draw[<->](6.6,2)to[out=10,in=180](8.3,2.3);
		
		%nomi
		\node at (6.1,-0.2){\(g(x)\)};
		\node at (7.6,-0.6){\(G(x)\)};
		%\node at (7.2,0.7){\(H_s(x)\)};
		\node at (5.9,1){\(g(\D^k)\)};
		\node at (7.5,2){\(\ve\)};
		\node at (8.8,2.5){\(M\)};
		
		
		
	\end{tikzpicture}
	
	\caption{Costruzione dell'omotopia \(H\).}
	
	\label{fig: costruzione retrazione}
	
\end{figure}
	
	Osserviamo che \(h=(H_0)_\#: \Sp^{k+1} \to M\) è data da
	\[
	h(x_0 \cos (2\pi t), x_0 \sin (2 \pi t), z) = H_0(x)(t)=g(x), \qquad t \in [0,1], x \in \D^k,
	\]
	dunque se \(\rho:\D^k \to \D^k\) è una retrazione di \(\D^k\) nel punto \(e_{k+1}\), \(h\) è omotopa alla mappa costante \(h \circ \rho \equiv p\) relativamente a \(e_{k+1}\). Per il Lemma~\ref{lemma: corrispondenza h e H}, ci sono le seguenti omotopie lisce relative a \(e_{k+1}\)  (denotate con il simbolo \(\sim\)) tra mappe \(C^\infty(\Sp^{k+1},M)\):
	\[
		h \circ \rho \sim h \sim g \sim f.
	\]
	Assurdo, perché \(f\) è omotopicamente non banale. 
	
	Dunque deve essere \(\lambda>0\), e questo conclude la dimostrazione.
\end{proof}


\section{Dopo il teorema di Lusternik-Fet}

Alla luce del Teorema~\ref{teo: Lusternik-Fet}, è naturale chiedersi:
\begin{enumerate}[label=(\arabic*)]
	\item Quante geodetiche chiuse geometricamente distinte esistono?
	\item Esiste sempre una geodetica chiusa semplice, cioè che non si autointerseca?
	\item Esistono risultati analoghi per sottovarietà chiuse di dimensione maggiore di 1?
\end{enumerate}

Per chiarire la domanda (1), dobbiamo prima precisare cosa vuol dire distinguere geometricamente due curve. 
\begin{defi}
	Due curve chiuse \(\gamma, \widetilde{\gamma} : \Sp^1 \to M\) si dicono \textit{geometricamente distinte} se non esistono \(a,b \in \R\) tale che
	\[
		\gamma(at+b) = \widetilde{\gamma}(t).
	\]
\end{defi}
In questo modo, una geodetica è sempre geometricamente distinta da una curva non geodetica. 
\begin{teo}
	In una varietà Riemanniana chiusa con gruppo fondamentale finito esistono infinite geodetiche chiuse non costanti a due a due geometricamente distinte.
\end{teo}
Per una dimostrazione si veda \cite[Theorem 4.3.5]{klingenberg2012lectures}.

La domanda (2) ha una risposta nel caso di una superficie compatta semplicemente connessa, cioè della sfera \(M=\Sp^2\) dotata di una metrica Riemanniana arbitraria:
\begin{teo}[Lusternik-Schnirelmann, 1929]
	Sulla sfera \(\Sp^2\) con una metrica Riemanniana arbitraria esistono almeno tre geodetiche chiuse geometricamente distinte e senza autointersezioni.
\end{teo}
Per una dimostrazione si veda \cite[Theorem A.3.1]{klingenberg2012lectures}. Il risultato non può essere migliorato, come mostra il seguente esempio ottenuto da Morse.
\begin{es}[Morse]
	Sia \(E=E(a,b,c) \subset \R^3\) l'ellissoide di equazione
	\[
		\frac{x^2}{a^2}+\frac{y^2}{b^2}+\frac{z^2}{c^2}=1
	\] 
	dotato della metrica indotta da quella piatta di \(\R^3\). Esiste \(\ve >0\) tale che, se 
	\[
		1-\ve < a < b < c < 1+\ve,
	\]
	allora le uniche geodetiche chiuse senza autointersezioni sono le tre ellissi principali, ottenute intersecando \(E\) con i piani coordinati. Per una dimostrazione si veda \cite[Proposition 5.1.2]{klingenberg2012lectures}.
\end{es}

La domanda (3) è un problema attuale della geometria Riemanniana: studiare lo spazio delle \(k\)-sottovarietà Riemanniane attraverso il funzionale \(k\)-volume con metodi variazionali. Ad esempio, un'ipersuperficie chiusa \(\Sigma\) di una (\(n+1\))-varietà \(M\) è un'\textit{ipersuperficie minima} se è un punto critico del funzionale \(n\)-volume. Quando \(n=1\), si tratta di geodetiche chiuse di una superficie. Per \(n>1\) non è più possibile sfruttare il teorema di immersione di Sobolev per concludere che le superfici minime sono lisce, e in effetti non è assicurato. Tuttavia, come per il teorema di Lusternik-Fet, i metodi di min-max si sono rivelati degli strumenti molto potenti per dimostrare risultati di esistenza di ipersuperfici minime chiuse lisce. F. Almgren e J. Pitts hanno generalizzato il teorema di Lusternik-Fet delle geodetiche alle ipersuperfici minimie per \(2 \leq n \leq 5\) \cite{pitts1981existence}, che poi R. Schoen e L. Simon hanno esteso al caso \(n=6\) \cite{schoen1981regularity}:
\begin{teo}[Almgren-Pitts-Schoen-Simon, 1981]
	Ogni \((n+1)\)-varietà Riemanniana chiusa con \(2 \leq n \leq 6\) contiene un'ipersuperficie minima chiusa liscia. 
\end{teo}
In realtà hanno mostrato che anche per \(n \geq 7\) esiste un'ipersuperficie minima, ma potrebbe essere singolare (nel senso delle \textit{correnti integrali} \cite{delellis2015size}) in un sottoinsieme con codimensione di Hausdorff almeno 7.

Motivato da questo risultato, S.-T. Yau formulò in \cite{yau1981seminar} la seguente congettura.
\begin{conget}[Yau, 1981]
	Ogni 3-varietà chiusa contiene infinite superfici minime lisce. 
\end{conget}
Usando dei metodi di min-max e i lavori di Marques-Neves \cite{marques2013existence} \cite{marques2016morseindexmultiplicityminmax} \cite{marques2019morseindexmultiplicityminmax}, Irie-Marques-Neves \cite{irie2018densityminimalhypersurfacesgeneric}, Marques-Neves-Song \cite{marques2018equidistributionminimalhypersurfacesgeneric} e Liokumovich-Marques-Neves \cite{liokumovich2018weyllawvolumespectrum}, la congettura di Yau è stata provata in una forma più forte nel 2018 da A. Song in \cite{song2023existence}:
\begin{teo}[Song, 2018]
	Ogni \((n+1)\)-varietà Riemanniana chiusa con \(2 \leq n\leq 6\) contiene infinite ipersuperfici minime chiuse lisce. 
\end{teo}





	
	\appendix
	\chapter{Gruppi di omotopia} \label{appendice}
	
In questa appendice sono raccolte la definizione dei gruppi di omotopia e la dimostrazione della Proposizione~\ref{prop: esiste f omotopicamente non banale}. Tutte le definizioni e i risultati sono tratti da \cite{hatcher2000algebraic}, ad eccezione del Teorema~\ref{teo: approssimazione whitney} (\cite[Theorem~6.19]{lee2012smooth}).

\section{Gruppi di omotopia}

Siano \(X\) e \(Y\) due spazi topolgici e \(A \subset X\). Due mappe continue \(f,g:X \to Y\) si dicono \textit{omotope relativamente ad \(A\)} se esiste una mappa \(H:X \times [0,1] \to Y\), detta \textit{omotopia relativa ad \(A\)}, tale che
\begin{itemize}
	\item \(H(x,0) = f(x)\) per ogni \(x \in X\);
	\item \(H(x,1) = g(x)\) per ogni \(x \in X\);
	\item \(H(a,t) = f(a) = g(a)\) per ogni \(a \in A\), \(t \in [0,1]\).
\end{itemize}
È immediato verificare che questa è una relazione di equivalenza su \(C^0(X,Y)\).

Sia \(x_0 \in X\) un punto fissato. La coppia \((X,x_0)\) è detta \textit{spazio puntato}, e \(x_0\) è detto \textit{punto base}. Sia \((Y,y_0)\) un altro spazio puntato. Una mappa tra spazi puntati \(f:(X,x_0) \to (Y,y_0)\) è una mappa continua \(f : X \to Y\) tale che \(f(x_0)=y_0\).

Sia  \(e_n =(0,\dots,0,1)\) il polo nord della sfera \(\Sp^n \subset \R^{n+1}\). Denotiamo \(\pi_n(X,x_0)\) l'insieme di classi di omotopia relativa a \({x_0}\) di mappe tra spazi puntati \(f:(\Sp^n,e_n) \to (X,x_0)\). L'insieme \(\pi_n(X,x_0)\) eredita una struttura di gruppo se dotato dell'operazione \([f][g]=[f*g]\) (Figura \ref{fig: operazione gruppo di omotopia}), con
\begin{itemize}
	\item  \(f*g=c \circ f \vee g\);
	\item \(c:\Sp^n \to \Sp^n \) la mappa che collassa l'equatore \(\Sp^{n-1} \owns e_n\) in un punto;
	\item \(f \vee g : \Sp^n \to \Sp^n \to X\) definita come \(f\) su \(\Sp^n \vee e_n\) e \(g\) su \(e_n \vee \Sp^n\), avendo ruotato le due sfere in modo che \(e_n\) sia il punto di contatto.
\end{itemize} 

%OPERAZIONE DI GRUPPO
\begin{figure}[h]
	\centering
	\begin{tikzpicture}
		%\draw[help lines](-2,-2) grid (10,2);
		
		%Sfera
		\draw[thick](-1.5,0)arc(180:360:1.5 and 0.5);
		\draw[thick,dashed](1.5,0)arc(0:180:1.5 and 0.5);
		\draw[thick] (0,0)circle(1.5);
		\fill (1.5,0) circle (1.5pt);
		\node at (1,0){\(e_n\)};
		\node at (1.5,1.5){\(\Sp^n\)};
		
		\draw [thick,->] (2,0)--(3.2,0);
		\node at (2.6,0.2){\(c\)};
		
		%Bouqet di sfere
		\draw[thick] (4.5,0.75)circle(0.75)
		(4.5,-0.75)circle(0.75);
		\draw[thick](3.75,0.75)arc(180:360:0.75 and 0.25)
		(3.75,-0.75)arc(180:360:0.75 and 0.25);
		\draw[thick,dashed](5.25,0.75)arc(0:180:0.75 and 0.25)
		(5.25,-0.75)arc(0:180:0.75 and 0.25);
		\fill (4.5,0) circle (1.5pt);	
		\node at (4.5,-0.25){\(e_n\)};
		\node at (6,1.5){\(\Sp^n \vee \Sp^n\)};
		
		\draw[thick,->](5.5,0.5)--(7,0.2);
		\draw[thick,->](5.5,-0.5)--(7,-0.2);
		
		\node at (6.2,0.7){\(f\)};
		\node at (6.2,-0.7){\(g\)};
		
		%X
		\node at(7.5,0){\(X\)};
	\end{tikzpicture}
	
	\caption{Operazione di gruppo su \(\pi_n(X,x_0)\).}
	\label{fig: operazione gruppo di omotopia}
\end{figure}

\begin{defi}
	\(\pi_n(X,x_0)\) è l'\(n\)-esimo \textit{gruppo di omotopia} dello spazio puntato \((X,x_0)\).
\end{defi}

\begin{oss}
	Identificando \(\Sp^1 = [0,1]/\{0,1\}\), è immediato verificare che il gruppo fondamentale è esattamente il primo gruppo di omotopia. 
\end{oss}

\begin{prop}\label{prop: invarianza pt base}
	Una cammino \(\gamma: [0,1] \to X\), con \(x_0 = \gamma(0)\) e \(x_1=\gamma(1)\), induce isomorfismi \(\pi_n \gamma: \pi_n(X,x_1) \to \pi_n(X,x_0)\), definiti come in Figura~\ref{fig: cambio punto base}.
\end{prop}
Per la dimostrazione si veda \cite[p. 341]{hatcher2000algebraic}.


	\begin{figure}[h]
	\centering
	\begin{tikzpicture}[decoration={markings, mark=at position 0.5 with {\arrow{<}}}]
		\draw[thick](-2,-2) rectangle (2,2);
		\draw[thick](-1,-1) rectangle (1,1);
		\draw[thin](-1,-1) to (-2,-2);
		\draw[thin](1,-1) to (2,-2);
		\draw[thin](-1,1) to (-2,2);
		\draw[thin](1,1) to (2,2);
		\draw[thin](-1,0) to (-2,0);
		\draw[thin](0,-1) to (0,-2);
		\draw[thin](1,0) to (2,0);
		\draw[thin](0,1) to (0,2);
		\draw[thin](-0.5,-1) to (-1,-2);
		\draw[thin](0.5,-1) to (1,-2);
		\draw[thin](1,-0.5) to (2,-1);
		\draw[thin](1,0.5) to (2,1);
		\draw[thin](0.5,1) to (1,2);
		\draw[thin](-0.5,1) to (-1,2);
		\draw[thin](-1,0.5) to (-2,1);
		\draw[thin](-1,-0.5) to (-2,-1);
		\draw[thin,decorate](-1,-1) to (-2,-2);
		\draw[thin,decorate](1,-1) to (2,-2);
		\draw[thin,decorate](-1,1) to (-2,2);
		\draw[thin,decorate](1,1) to (2,2);
		\draw[thin,decorate](-1,0) to (-2,0);
		\draw[thin,decorate](0,-1) to (0,-2);
		\draw[thin,decorate](1,0) to (2,0);
		\draw[thin,decorate](0,1) to (0,2);
		\draw[thin,decorate](-0.5,-1) to (-1,-2);
		\draw[thin,decorate](0.5,-1) to (1,-2);
		\draw[thin,decorate](1,-0.5) to (2,-1);
		\draw[thin,decorate](1,0.5) to (2,1);
		\draw[thin,decorate](0.5,1) to (1,2);
		\draw[thin,decorate](-0.5,1) to (-1,2);
		\draw[thin,decorate](-1,0.5) to (-2,1);
		\draw[thin,decorate](-1,-0.5) to (-2,-1);
		
		\node at (-0.7,-0.05){\(x_1\)};
		\node at (-2.3,-0.05){\(x_0\)};
		\node at (0.7,0.05){\(x_1\)};
		\node at (2.3,0.05){\(x_0\)};
		\node at (0,-0.7){\(x_1\)};
		\node at (0,-2.3){\(x_0\)};
		\node at (0,0.7){\(x_1\)};
		\node at (0,2.3){\(x_0\)};
		\node at (-1.5, 0.3){\(\gamma\)};
		\node at (0,0){\(f\)};
		
	\end{tikzpicture}
	
	\caption{Identificando \(\Sp^n = I^n/ \de I^n\), con \(I=[0,1]\), possiamo lavorare sull'ipercubo \(I^n\). Dunque la figura definisce, per ogni \(f : (\Sp^n,e_n) \to (X,x_1)\), una mappa continua \(\gamma f: (\Sp^n,e_n) \to (X,x_0)\) e possiamo definire \(\pi_n\gamma[f] \coloneq [\gamma f]\).}
	\label{fig: cambio punto base}
\end{figure}

Qualora \(X\) sia connesso per archi, la Proposizione~\ref{prop: invarianza pt base} permette di denotare l'\(n\)-esimo gruppo di omotopia \(\pi_n(X)\), senza specificare il punto base. In quanto segue, se \(X\) è connesso per archi, adotteremo questa notazione.

Concludiamo la sezione con il seguente risultato di approssimazione.

\begin{teo}[di approssimazione di Whitney]\label{teo: approssimazione whitney}
	Siano \(N,M\) due varietà differenziabili e sia \(F:N \to M\) una mappa continua. Allora \(F\) è omotopa a una mappa liscia \(\widetilde{F}:N \to M\). Inoltre, se \(F\) è \(C^\infty\) su un sottoinsieme chiuso \(A \subset N\), allora l'omotopia può essere scelta relativa ad \(A\), e in particolare \(\widetilde{F}|_A = F|_A\).
\end{teo}
\begin{proof}
	Si veda \cite[Theorem 6.19]{lee2012smooth}.
\end{proof}

Scegliendo \(N=\Sp^k\) e \(A=\{e_k\}\) otteniamo il seguente corollario.
\begin{cor}\label{cor: rappresentazione liscia}
	Se \(M\) è una varietà differenziabile, ogni classe \(\alpha \in \pi_k(M,p)\) ammette un rappresentante in \(C^\infty(\Sp^k,M)\). 
\end{cor}

\section{Gruppi di omotopia di una varietà semplicemente connessa}

Lo scopo di questa sezione è dimostrare la seguente:

\begin{prop}\label{prop: esiste f omotopicamente non banale}
	Se \(M\) è una varietà (topologica) \(n\)-dimensionale, chiusa e semplicemente connessa, allora esiste \(0<k < n\) tale che \(\pi_{k+1}(M) \not = \{0\}\). In particolare, se \(M\) è una varietà differenziabile, esiste una mappa \(f:\Sp^{k+1} \to M\) di classe \(C^\infty\) e omotopicamente non banale, cioè non omotopa relativamente al punto \(e_{k+1} \in \Sp^{k+1}\) a nessuna mappa costante. 
\end{prop}

Per dimostrarlo, utilizzeremo risultati profondi che connettono tra loro \textit{omotopia} e \textit{omologia}. Poiché per la comprensione dell'enunciato della Proposizione~\ref{prop: esiste f omotopicamente non banale} non è rilevante sapere cosa sono i gruppi di omologia \(H_k(X)\), li tratteremo semplicemente come dei gruppi associati a uno spazio topologico \(X\), rimandando a \cite{hatcher2000algebraic} per tutte le definizioni. 

Riportiamo i seguenti risultati, senza dimostrazione.

\begin{teo} \label{teo: H_n = Z}
	Se \(M\) è una varietà \(n\)-dimensionale chiusa e semplicemente connessa, allora \(H_n(M) \cong \Z\).
\end{teo}
Per la dimostrazione, si veda \cite[Proposition 3.25 e Theorem 3.26]{hatcher2000algebraic}.

\begin{teo}[Hurewicz]\label{teo: Hurewizc}
	Sia \(X\) uno spazio topologico connesso per archi e tale che \(\pi_k(X)=\{0\}\) per ogni \(k \leq n-1\). Allora \(\pi_n(X)\cong H_n(X)\). 
\end{teo}
Per la dimostrazione, si veda \cite[Theorem 4.32]{hatcher2000algebraic}.

\begin{proof}[Dimostrazione della Proposizione~\ref{prop: esiste f omotopicamente non banale}]
	Supponiamo che \(\pi_{k+1}(M)\) siano banali per ogni \(k < n-1\) e mostriamo che \(\pi_n(M) \not= \{0\}\). Per il teorema di Hurewicz, \(\pi_n(M) \cong H_n(M)\), che, per il Teorema~\ref{teo: H_n = Z}, è isomorfo a \(\Z\).
	
	Sia \(M\) una varietà differenziabile \(n\)-dimensionale, chiusa e semplicemente connessa. Abbiamo provato che esiste \(0<k<n\) tale che \(\pi_{k+1}(M) \neq \{0\}\). Concludiamo con il Corollario \ref{cor: rappresentazione liscia} scegliendo un rappresentante liscio di una classe non banale \(\alpha \in \pi_{k+1}(M,p)\). 
\end{proof}
	
	\printbibliography
	\addcontentsline{toc}{chapter}{Bibliografia}
\end{document}
